% !TEX root = thesis.tex

%%
%%
%% Method chapter
%%
%%


\section{Diggle}



\section{Guan}

From \citet{guan2008consistent}

\(N\) is a spatial point process.
\(W \subset \mathbb{R}^2 \) is the study area (window).
Intensity at \(\vec{s} \in \mathbb{R}^2\) is defined as:

\[
    \lambda (\vec{s}) = \lim_{|d\vec{s}| \to 0} \left\{ \frac{\mathbb{E}[N(d\vec{s})]}{|d\vec{s}|} \right\}
\]

And \(\lambda(\vec{s})|d\vec{s}|\) is the approximate probability of \(d\vec{s}\) containing a single incident.

Goal of intensity estimate: approximate \(\lambda(s)\) on the study area.

Kernel intensity estimator (my simplification):
\[
    \hat{\lambda}(\vec{s}; h) = \frac{1}{h} \sum_{\vec{x} \in N \cap W}{k((\vec{x}-\vec{s})/h)}
\]

{
\color{red}
\textbf{The kernel intensity estimator is not consistent.}
This is because, although the kernel only uses local information around the point of interest to compute the estimate.
The resulting estimate has bias that tends to zero.
However it's variance does not diminish because the number of events in any fixed region is of order 1.
}

The paper develops a \textit{new nonparametric intensity estimator} that is consistent under ``some suitable yet reasonable conditions.''

\section{PhD thesis}

