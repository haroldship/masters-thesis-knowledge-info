% !TEX root = thesis.tex

%%
%%
%% Notation - these are definitions that will appear in the Notation section
%%
%%

\newglossaryentry{E}
{% This entry goes in the `notation' glossary:
   type=notation,
   name={\ensuremath{\mathbb{E}}},
   description={expected value},
   sort={E}
}

\newglossaryentry{f}
{% This entry goes in the `notation' glossary:
   type=notation,
   name={\ensuremath{f}},
   description={a probability density function defined on \ensuremath{W}},
   sort={f}
}

\newglossaryentry{f_hat}
{% This entry goes in the `notation' glossary:
   type=notation,
   name={\ensuremath{\hat{f}}},
   description={an estimate of a probability density function},
   sort={f_hat}
}


\newglossaryentry{grad}
{% This entry goes in the `notation' glossary:
   type=notation,
   name={\ensuremath{\nabla}},
   description={the gradient vector of partial derivative operators},
   sort={grad}
}

\newglossaryentry{h}
{% This entry goes in the `notation' glossary:
   type=notation,
   name={\ensuremath{h}},
   description={kernel bandwidth parameter},
   sort={h}
}

\newglossaryentry{h_1}
{% This entry goes in the `notation' glossary:
   type=notation,
   name={\ensuremath{h_1}},
   description={kernel bandwidth parameter in the \ensuremath{x_1} direction},
   sort={h_1}
}

\newglossaryentry{h_2}
{% This entry goes in the `notation' glossary:
   type=notation,
   name={\ensuremath{h_2}},
   description={kernel bandwidth parameter in the \ensuremath{x_2} direction},
   sort={h_2}
}

\newglossaryentry{h_opt}
{% This entry goes in the `notation' glossary:
   type=notation,
   name={\ensuremath{h_{\mbox{opt}}}},
   description={optimal kernel bandwidth parameter for a SPP},
   sort={h_opt}
}

\newglossaryentry{h_o1}
{% This entry goes in the `notation' glossary:
   type=notation,
   name={\ensuremath{h_{o1}}},
   description={optimal kernel bandwidth parameter in the \ensuremath{x_1} direction, approximated using an Oracle},
   sort={h_o1}
}

\newglossaryentry{h_o2}
{% This entry goes in the `notation' glossary:
   type=notation,
   name={\ensuremath{h_{o2}}},
   description={optimal kernel bandwidth parameter in the \ensuremath{x_2} direction, approximated using an Oracle},
   sort={h_o2}
}

\newglossaryentry{I}
{% This entry goes in the `notation' glossary:
   type=notation,
   name={\ensuremath{I}},
   description={the set of points in \ensuremath{W} making up the incidents},
   sort={I}
}

\newglossaryentry{K}
{% This entry goes in the `notation' glossary:
   type=notation,
   name={K},
   description={a kernel function}
}

\newglossaryentry{Lambda}
{% This entry goes in the `notation' glossary:
   type=notation,
   name={\ensuremath{\Lambda}},
   description={the spatial point process that whose realization is the intensity \ensuremath{\lambda}},
   sort={lambda}
}

\newglossaryentry{lambda}
{% This entry goes in the `notation' glossary:
   type=notation,
   name={\ensuremath{\lambda}},
   description={an intensity function in the plane},
   sort={lambda}
}

\newglossaryentry{lambda_hat}
{% This entry goes in the `notation' glossary:
   type=notation,
   name={\ensuremath{\hat{\lambda}}},
   description={an estimate of an intensity function},
   sort={lambda_hat}
}

\newglossaryentry{lambda_hat_o}
{% This entry goes in the `notation' glossary:
   type=notation,
   name={\ensuremath{\hat{\lambda_o}}},
   description={an estimate of an intensity function based on the oracle bandwidths},
   sort={lambda_hat}
}

\newglossaryentry{lambda_I}
{% This entry goes in the `notation' glossary:
   type=notation,
   name={\ensuremath{\lambda_I}},
   description={the intensity function of the incidents},
   sort={lambda_i}
}

\newglossaryentry{lambda_P}
{% This entry goes in the `notation' glossary:
   type=notation,
   name={\ensuremath{\lambda_P}},
   description={the intensity function of the population},
   sort={lambda_p}
}

\newglossaryentry{relative_lambda_hat}
{% This entry goes in the `notation' glossary:
   type=notation,
   name={\ensuremath{\hat{\lambda}^*}},
   description={the estimate of an intensity function relative to the true value; \ensuremath{\hat{\lambda}(x_1, x_2)/\lambda(x_1, x_2)}},
   sort={lambda_hat_star}
}

\newglossaryentry{mu}
{% This entry goes in the `notation' glossary:
   type=notation,
   name={\ensuremath{\mu}},
   description={a constant multiplication factor to scale the intensity},
   sort={mu}
}

\newglossaryentry{N}
{% This entry goes in the `notation' glossary:
   type=notation,
   name={\ensuremath{N}},
   description={a set of events},
   sort={N}
}

\newglossaryentry{P}
{% This entry goes in the `notation' glossary:
   type=notation,
   name={\ensuremath{P}},
   description={the set of points in \ensuremath{W} that make up the population of the experiment},
   sort={P}
}

\newglossaryentry{prob}
{% This entry goes in the `notation' glossary:
   type=notation,
   name={\ensuremath{\mathbb{P}}},
   description={probability},
   sort={P}
}

\newglossaryentry{sigma}
{% This entry goes in the `notation' glossary:
   type=notation,
   name={\ensuremath{\sigma}},
   description={standard deviation},
   sort={sigma}
}

\newglossaryentry{sigma_1}
{% This entry goes in the `notation' glossary:
   type=notation,
   name={\ensuremath{\sigma_1}},
   description={standard deviation in the \ensuremath{x_1} direction},
   sort={sigma_1}
}

\newglossaryentry{sigma_2}
{% This entry goes in the `notation' glossary:
   type=notation,
   name={\ensuremath{\sigma_2}},
   description={standard deviation in the \ensuremath{x_2} direction},
   sort={sigma_2}
}

\newglossaryentry{sigma_i}
{% This entry goes in the `notation' glossary:
   type=notation,
   name={\ensuremath{\sigma_I}},
   description={standard deviation, representing the spread of the incident risk function},
   sort={sigma_i}
}

\newglossaryentry{sigma_p}
{% This entry goes in the `notation' glossary:
   type=notation,
   name={\ensuremath{\sigma_P}},
   description={standard deviation, representing the spread of the population distribution},
   sort={sigma_p}
}

\newglossaryentry{W}
{% This entry goes in the `notation' glossary:
   type=notation,
   name={\ensuremath{W}},
   description={the study area, a subset of \ensuremath{\RS}},
   sort={W}
}

