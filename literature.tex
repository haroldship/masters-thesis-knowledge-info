% !TEX root = thesis.tex

%%
%%
%% Literature chapter
%%
%%

Kernel smoothing is a highly fexible and popular approach for estimation of probability density and intensity functions of continuous spatial data. In this role it also forms an integral part of estimation of functionals such as the density-ratio or ``relative risk surface''. Originally developed with the epidemiological motivation of examining fluctuations in disease risk based on samples of cases and controls collected over a given geographical region, such functions have also been successfully employed across a diverse range of disciplines where a relative comparison of spatial density functions has been of interest. This versatility has demanded ongoing developments and improvements to the relevant methodology, including use spatially adaptive smoothers; tests of significantly elevated risk based on asymptotic theory; extension to the spatiotemporal domain; and novel computational methods for their evaluation. In this tutorial paper we review the current methodology, including the most recent developments in estimation, computation and inference. All techniques are implemented in the new software package sparr, publicly available for the R language, and we illustrate its use with a pair of epidemiological examples. \citep{davies2018tutorial}
