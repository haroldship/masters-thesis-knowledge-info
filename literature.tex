% !TEX root = thesis.tex

%%
%%
%% Literature chapter
%%
%%

Our research is concerned with two-dimensional non-parametric intensity estimation,
that is estimating functions in the plane while making only minimal assumptions about the underlying distribution of the data. 
In particular, we are interested in estimating an incidence rate,
which is the ratio of incidence count and of population which were each individually computed using kernel methods.
This technique has been applied in several areas of spatial analysis including epidemiology.

\section{Spatial pattern analysis}

The estimation of intensity is part of the wider field of spatial pattern analysis.
For event data such as disease incidence,
spatial point processes including Poisson processes are often used to model the statistical and spatial properties of events.
\Citet{diggle1983spatial} is an early publication which extensively covers the spatial-statistical analysis of point patterns such as these.
It covers several stochastic models and statistical methods,
and provides examples of biological and epidemiological analyses.

\section{Kernel density estimation}
 
While there is some theoretical work available for \gls{kernel intensity estimation},
it is usually based on and compared to \acrfull{kde}.
\Gls{kde} is a popular method of non-parametric estimation of probability density functions from observed data.

The primary reference for probability density estimation is \citet{silverman1986density},
which covers both parametric and non-parametric methods.
In particular,
he covers kernel methods,
including the multi-dimensional \gls{kernel intensity estimation} used in this research.
We note that both of our bandwidth selection methods,
the \Gls{silverman} Rule of Thumb and the least-squares \acrlong{cv} are discussed in this book.
The earliest paper to discuss \gls{kde} for probability density functions was by \citet{rosenblatt1956remarks}.
It contains derivations of the bias, variance,
and \acrfull{mise} for general density estimates in one dimension,
and introduces the class of kernel estimates.
\Citet{wand1993comparison} take an in-depth look at smoothing parameters in two dimensions.
In particular,
they find that using a two-dimensional bandwidth vector is adequate for smoothing in each coordinate direction.
In another book, \citet{wand1994kernel}
give a broad overview of kernel methods,
including several techniques for selecting the bandwidth.
In examining how both the \gls{kde} and \gls{kernel intensity estimation} can use data to learn about underlying structure,
they describe how these similar techniques are used in comparison to parametric methods.
In addition to single and multi-variable kernel methods to estimate functions,
they also describe how kernel regression techniques can be used for supervised learning and prediction.

A different perspective can be found in \citet{devroye1985nonparametric},
which analyses non-parametric density estimation from an $L_1$ perspective,
which is in line with our use of the \gls{miae} measure of accuracy.

\section{Kernel intensity estimation}

The idea to use \gls{kernel intensity estimation} to estimate the intensity functions of one-dimensional point processes can be found in \citet{diggle1985kernel}.
Based on the kernel methods described by \citet{rosenblatt1956remarks},
and making use of edge-correction,
Diggle's paper includes estimates for the \gls{mise} as well as a performance evaluation using simulated data.
The equivalence of the bandwidth in kernel density and intensity was shown in \citet{diggle1988equivalence}.
\Citet{brooks1991asymptotic} show that the \gls{cv} bandwidth is asymptotically optimal,
building on similar results for the \gls{kde} \citep{hall1983large,burman1985data,stone1984asymptotically}.

Unlike the \gls{kde} however,
the \gls{kernel intensity estimator} is not consistent,
meaning that the \gls{mise} does not converge to zero as the number of incidents increases.
There is some work in this area,
with \citet{guan2008consistent} and \citet{fuentes2016consistent} developing techniques for \gls{kernel intensity estimation} that are consistent under certain conditions.
The research mentioned above is mostly concerned with one-dimensional data.
However, kernel technieques have successfully been used in two dimensions as well \citep{scott1992multivariate}.

\section{Application to epidemiology}

Current research in kernel estimation continues.
For example, we see kernel estimation techniques used in epidemiology.
Like our research,
the tutorial paper \citet{davies2018tutorial} looks at using the ratio of two kernel estimates for estimating epidemiological risk functions.
However,
they also delve into several additional bandwidth selection techniques,
edge corrections,
and other methods that extend the basic \gls{kde}.
Their paper is based on the technique developed by \citet{bithell1990application,bithell1991estimation,kelsall1995kernel}.
By conditioning on sample size,
they compute two \glspl{kde} instead of \glspl{kernel intensity estimator} in order to compare the relative risk between two samples representing the case and control populations on a common study area.
This relative risk ratio represents a different quantity from the incidence rate of our current research.

In many studies,
incidents of disease are aggregated together instead of using spatial analysis.
This is referred to as ``zonal analysis'' and suffers from several issues,
mainly due to information that is lost when aggregating and computing \glspl{asr}.
One of the motivations of spatial analysis is to provide more informative results than can be done with aggregation.
An example of how the results of studies based on such aggregation can be misleading can be found in \citet{portnov2007ecological}.

\section{The double kernel}

The \acrfull{dkd} has been used in epidemiological studies since at least 2004 \citep{rushton2004analyzing}.
\citet{kloog2009using} is another recent study which uses the \gls{dkd}.
In particular,\
it identifies the \gls{maup},
which occurs when changes to the arbitrary administration zones that are used to aggregate incidents can result in different observations and findings.
\citet{zusman2012residential} is another study which discusses the advantages of the \gls{dkd} over \glspl{asr}.
A recent examination of the properties of the \gls{dkd} can be found in \citet{zusman2016application}.
They find that:
\begin{quotation}
The DKD approach provides reasonably stable and consistent estimates, if the following three preconditions are met: (a) the kernel estimation parameters are properly defined, (b) the number of reference points, used for transformation of continuous DKD surfaces into discrete observations, is sufficiently large, and (c) the spatial dependency of neighboring observations is taken into account.
\end{quotation}


While the \gls{dkd} has been used frequently in epidemiological and non-epidemiological spatial analysis,
its statistical properties have not been adequately studied.
In order to understand these properties,
both a theoretical and an empirical analysis is required.
The scope of this research is to provide a beginning to the empirical analysis by studying the sensitivity of the \gls{dkd} to several factors using monte carlo simulations.

