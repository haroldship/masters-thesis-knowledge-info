% !TEX root = thesis.tex

%%
%%
%% Glossary - these are definitions that will appear in the Glossary section
%%
%%

\newglossaryentry{cross-validation} {
   type=main,
   name={cross-validation},
   description={a method for selecting bandwidths by minimising leave-one-out cross-validation mean squared error}
}

\newglossaryentry{spread} {
   type=main,
   name={spread},
   description={an experimental parameter that controls how wide the population distribution or the risk function is spread out over the study area; equivalent to standard deviation}
}

\newglossaryentry{factor}
{% This entry goes in the `main' glossary:
   type=main,
   name={expected number of incidents},
   description={the expected number of incidents from the stochastic process}
}

\newglossaryentry{oracle} {
   type=main,
   name={oracle},
   description={an imaginary entity that has access to the the unknown truth}
}

\newglossaryentry{silverman} {
   type=main,
   name={Silvermen},
   description={Silverman's rule of thumb for optimal bandwidth estimation}
}

\newglossaryentry{supremum error}
{% This entry goes in the `main' glossary:
   type=main,
   name={supremum error},
   description={the maximum absolute value of the error of the dkd estimate}
}

\newglossaryentry{peak error}
{% This entry goes in the `main' glossary:
   type=main,
   name={peak error},
   description={the difference between the dkd estimate and the highest value of the risk}
}

\newglossaryentry{peak bias}
{% This entry goes in the `main' glossary:
   type=main,
   name={peak bias},
   description={the expected value of the peak error}
}

\newglossaryentry{peak drift}
{% This entry goes in the `main' glossary:
   type=main,
   name={peak drift},
   description={the distance between of the peak of the dkd estimate and the peak of the true risk function}
}

\newglossaryentry{centroid error}
{% This entry goes in the `main' glossary:
   type=main,
   name={centroid error},
   description={the difference between the centroid of the top 5\% of the dkd estimate and the highest value of the risk}
}

\newglossaryentry{centroid bias}
{% This entry goes in the `main' glossary:
   type=main,
   name={centroid bias},
   description={the expected value of the centroid error}
}

\newglossaryentry{centroid drift}
{% This entry goes in the `main' glossary:
   type=main,
   name={peak drift},
   description={the distance between of the centroid of the top 5\% of the dkd estimate and the peak of the true risk function}
}



