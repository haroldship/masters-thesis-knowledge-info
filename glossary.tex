% !TEX root = thesis.tex

%%
%%
%% Glossary - these are definitions that will appear in the Glossary section
%%
%%

\newglossaryentry{centroid bias}
{% This entry goes in the `main' glossary:
   type=main,
   name={centroid bias},
   description={the expected value of the centroid error}
}

\newglossaryentry{centroid drift}
{% This entry goes in the `main' glossary:
   type=main,
   name={centroid drift},
   description={the distance between of the centroid of the top 5\% of the dkd estimate and the centroid of the true risk function}
}

\newglossaryentry{event}
{% This entry goes in the `main' glossary:
   type=main,
   name={event},
   description={a random point in $\glsentryname{W} \subset \RS$ that is of interest}
}

\newglossaryentry{factor}
{% This entry goes in the `main' glossary:
   type=main,
   name={expected number of incidents},
   plural={expected numbers of incidents},
   description={the expected number of incidents from the stochastic process}
}

\newglossaryentry{gls-cv}
{% use \gls{cv} which is found in acronyms.tex
   type=main,
   name={cross-validation},
   description={a method for selecting bandwidths by minimising leave-one-out cross-validation mean squared error}
}

\newglossaryentry{incidence rate} 
{
   type=main,
   name={incidence rate},
   description={the number of incidents divided by person-time}
}

\newglossaryentry{incident}
{
   type=main,
   name={incident},
   description={a new case of the disease during the study period}
}

\newglossaryentry{intensity}
{
   type=main,
   name={intensity},
   description={a function of a point $\xvec \in \RS$ that returns the expected number of events at that point}
}

\newglossaryentry{kernel}
{
   type=main,
   name={kernel},
   description={a probability density function used in kernel estimation}
}

\newglossaryentry{kernel estimate}
{
   type=main,
   name={kernel estimate},
   description={an estimate of a function computed by smoothing event data with a kernel}
}

\newglossaryentry{kernel intensity estimator}
{
   type=main,
   name={kernel intensity estimator},
   description={a rule for estimating an intensity function by smoothing event data using a kernel}
}

\newglossaryentry{kernel intensity estimation}
{
   type=main,
   name={kernel intensity estimation},
   description={the process of estimating an intensity function with a kernel intensity estimator}
}

\newglossaryentry{oracle} 
{
   type=main,
   name={oracle},
   description={an imaginary entity that has access to the the unknown truth}
}

\newglossaryentry{oracle bandwidth} 
{
   type=main,
   name={oracle bandwidth},
   description={the bandwidth that minimizes the empirical \acrshort{mise} for a given \glsentryname{intensity}, computed from the data samples generated from the true intensity function}
}

\newglossaryentry{normalized supremum error}
{% This entry goes in the `main' glossary:
   type=main,
   name={normalized supremum error},
   description={the supremum error, normalized by the expected number of incidents}
}

\newglossaryentry{peak bias}
{% This entry goes in the `main' glossary:
   type=main,
   name={peak bias},
   description={the expected value of the peak error}
}

\newglossaryentry{peak drift}
{% This entry goes in the `main' glossary:
   type=main,
   name={peak drift},
   description={the distance between of the peak of the dkd estimate and the peak of the true risk function}
}

\newglossaryentry{peak error}
{% This entry goes in the `main' glossary:
   type=main,
   name={peak error},
   description={the difference between the dkd estimate and the highest value of the risk}
}

\newglossaryentry{relative centroid bias}
{% This entry goes in the `main' glossary:
   type=main,
   name={relative centroid bias},
   description={the \glsentryname{centroid bias} divided by the value of the true peak}
}

\newglossaryentry{relative centroid drift}
{% This entry goes in the `main' glossary:
   type=main,
   name={relative centroid drift},
   description={the \glsentryname{centroid drift} divided by the range of \ensuremath{x_1} or \ensuremath{x_2}, whichever is larger}
}

\newglossaryentry{risk}
{
   type=main,
   name={risk},
   description={the probability of contracting a disease}
}

\newglossaryentry{relative peak bias}
{% This entry goes in the `main' glossary:
   type=main,
   name={relative peak bias},
   description={the \glsentryname{peak bias} divided by the value of the true peak}
}

\newglossaryentry{relative peak drift}
{% This entry goes in the `main' glossary:
   type=main,
   name={relative peak drift},
   description={the \glsentryname{peak drift} divided by the range of \ensuremath{x_1} or \ensuremath{x_2}, whichever is larger}
}

\newglossaryentry{relative supremum error}
{% This entry goes in the `main' glossary:
   type=main,
   name={relative supremum error},
   description={the supremum error, relative to the maximum value of the true risk function}
}

\newglossaryentry{silverman} 
{
   type=main,
   name={Silverman},
   description={Silverman's rule of thumb for optimal bandwidth estimation}
}

\newglossaryentry{spread}
{
   type=main,
   name={spread},
   description={an experimental parameter that controls how wide the population distribution or the risk function is spread out over the study area; equivalent to standard deviation}
}

\newglossaryentry{supremum error}
{% This entry goes in the `main' glossary:
   type=main,
   name={supremum error},
   description={the maximum absolute value of the error of the dkd estimate}
}


