% !TEX root = scanstatistics_article.tex

\subsection*{What are scan statistics?}

Scan statistics are used to test if a point process is purely random or not, and to detect clusters \citep{naus1965clustering}.
In general, it works by scanning a subregion called a window over the area of interest, and counting how many of the points are contained in the window.
If and when the number of points within the window is large enough that the probability is small, a cluster is detected. \citep{schabenberger2004statistical}

The algorithm of \citep{kulldorff1997spatial} extends this idea, by using a circular window, and allowing the area of the window to vary.
He further extends the idea to allow for inhomogeneous Poisson or Bernoulli processes with \emph{known} intensity function.
It is evaluated using a likelihood ratio test that compares the likelihood detecting that number of points within the window with the likelihood of constant intensity.

\subsection*{How are they computed?}

Given a region $G$ in $\mathbb{R}^2$ containing $N$ points $(x_1,y_1), \ldots, (x_N, y_N)$, and a zone $Z \subset G$ compute the likelihood function $\mathfrak{L}~(Z)$ according to either the Bernoulli or Poisson model.
This function expresses the likelihood that that $k_Z$ events are in $Z$ and $N-k_Z$ events are in $Z^c$, i.e. that the rates inside and outside $Z$ are as per the sample.
The zone $\hat{Z}$ that maximizes the likelihood is the \emph{most likely cluster}.
Take the ratio of this with the likelihood that the rate inside and outside $Z$ is the same.
This is computed using Monte Carlo simulations -- need to figure this bit out.
\citep{schabenberger2004statistical,kulldorff1997spatial}

\subsection*{Is it used for chronic disease epidemiology?}


\citep[Section 6.4]{costa2009applications} describe several studies of cancer that used spatial scan statistics to look for clusters.
Not all of these found statistically significant clusters using this method.
However, \citep{viel2000soft} found a statistically significant cluster of soft-tissue sarcoma and non-Hodgkins lymphoma clusters around a municipal solid waste incinerator with high dioxin emission levels in France.
In this study they used circular windows with some modification to allow for certain geographic features (cantons, presence of pollution source).
The radius was allowed to vary from $0$ to $30\%$ of the \emph{population}.

\subsection*{Other}

\citep{perone2009false} discuss multiple testing and controlling false discovery of clusters.
 

