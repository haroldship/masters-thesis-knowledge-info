% !TEX root = thesis.tex

%%
%%
%% Introduction chapter
%%
%%


There are many examples in epidemiology of data that occurs as points distributed in the plane.
One example is
the home addresses of persons in a neighborhood and,
in particular, whether or not they have a particular medical condition.
These data allow one to compute the \textit{\gls{incidence rate}} of this condition.
The study of point set data in the plane is known as spatial point pattern analysis.

Recently, a technique known as the \gls{dkd} has been used for estimating \gls{incidence rate} functions
$\gls{lambda} := \gls{lambda}\dotdot$
which is the ratio of two other functions \gls{lambda_I} and \gls{lambda_P} \citep{portnov2009studying,kloog2009using,zusman2012residential}.
Each one of \gls{lambda_I} and \gls{lambda_P} is itself estimated by smoothing a set of point observations into a continuous,
smooth function \citep{bithell1990application},
and the value of \gls{lambda} at any point is the quotient of the values of
\gls{lambda_I} and \gls{lambda_P}.

Researchers have recently been motivated to use the \gls{dkd} due to information loss due to data aggregation of incidents into geographical areas such as cities or neighborhoods.
This aggregation has been deemed necessary both
to reduce the effects of variance on the relative errors of the results,
and to preserve the privacy of individuals who have been stricken with disease.

This study examines some of the statistical properties of the \gls{dkd}.
In particular,
we empirically analyse how different factors of the population and incidence distributions
affect the \textit{accuracy} of the \gls{dkd}.
Our methodology is based on monte marlo simulations under different experimental setups.

The contribution of this study will be to answer the following research questions:

\begin{question}
    \label{thm:accuracy-affected}
    How is the accuracy of the \gls{dkd} affected by:
    \begin{subquestions}
        \item the duration of the study, \label[question]{thm:accuracy-affected:duration}%
        \item different rate functions, \label[question]{thm:accuracy-affected:rates}%
        \item the size of the population, \label[question]{thm:accuracy-affected:popsize}%
        \item and different population distributions. \label[question]{thm:accuracy-affected:popdist}%
    \end{subquestions}
\end{question}

\begin{question}
    \label{thm:accuracy-scale}
    When looking at the the factors in \Cref{thm:accuracy-affected},
    how accurate is the \gls{dkd}:
    \begin{subquestions}
        \item globally, over the study area as a whole, \label[question]{thm:accuracy-scale:global}%
        \item and pointwise at the peaks. \label[question]{thm:accuracy-scale:peaks}%
    \end{subquestions}
\end{question}


