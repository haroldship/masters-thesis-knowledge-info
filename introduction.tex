% !TEX root = thesis.tex

%%
%%
%% Introduction chapter
%%
%%

There are many examples in epidemiology of data that occur as points distributed in the plane.
One example is
the home addresses of persons in a neighborhood and,
in particular,
whether or not they have developed a particular medical condition during a specific time period.
These data allow one to compute the \textit{\gls{incidence rate}} of this condition.
The study of point set data in the plane in general is known as spatial point pattern analysis.

Recently, a technique known as the \gls{dkd} has been used for estimating \gls{incidence rate} functions
$\gls{lambda} := \gls{lambda}\dotdot$.
The function \gls{lambda} is the ratio of two other functions \gls{lambda_I} and \gls{lambda_P} \citep{portnov2009studying,kloog2009using,zusman2012residential}.
Each one of the above functions \gls{lambda_I} and \gls{lambda_P}
is itself estimated by smoothing a set of point observations into a continuous,
smooth function \citep{bithell1990application} using \gls{kernel intensity estimation},
and the value of \gls{lambda} at any point is the estimate of \gls{lambda_I} at that point
divided by the estimate of \gls{lambda_P} at the same point.
Thus in order to estimate a function $\gls{lambda} = \gls{lambda_I}/\gls{lambda_P}$,
we compute two kernel intenstiy estimates and take the quotient.
In the special case where the population distribution is a constant,
the \gls{dkd} is the same as \gls{kernel intensity estimation}.

Researchers have recently been motivated to use the \gls{dkd} because of information loss
due to data aggregation of incidents into geographical areas such as cities or neighborhoods.
This aggregation has been deemed necessary both
to reduce the effects of variance on the relative errors of the results,
and to preserve the privacy of individuals who have been stricken with disease.
Use of the \gls{dkd} also allows for the linking of location data with other data sources,
such as welfare levels, air pollution and other attributes not available for individuals.

%%%%%%%%%%%%%%%%%%%%%%%%%%%%%%%%%%%%%%%%%%%%%%%%%%%%%%%%%%%%%%%%%%%%%%%%%%%%%%
%%
%% Section: Goal
%%
%%%%%%%%%%%%%%%%%%%%%%%%%%%%%%%%%%%%%%%%%%%%%%%%%%%%%%%%%%%%%%%%%%%%%%%%%%%%%%
\section{Study goal}
\label{sec:introduction:goal}

As far as we know,
the statistical properties of the \gls{dkd} are not yet fully understood.
We would like to know whether or not the estimates for \gls{risk} or \gls{incidence rate}
obtained using the \gls{dkd} are accurate or misleading in several ways.
It is also not known under what conditions we can expect reasonable results from these estimates.
For example,
we do not know if the \gls{dkd} accurately estimates the true \gls{risk}
for both small and large numbers of incidents.
We would also like to know how the shape of the underlying \gls{risk} function affects this accuracy.
Of particular concern are the peaks of the true risk function.
These peaks indicate a point around which the risk of contracting a disease is at its highest.
While the accuracy of the estimate of the magnitude of these peaks is important,
it is critical that the location estimate be accurate so as not create any misleading associations
with other, nearby points of interest.

The rest of this thesis is organized as follows.
\Cref{ch:theory} gives the theoretical background of the \gls{dkd} as a statistical and epidemiological tool.
\Cref{ch:literature} covers past and current research on the \gls{dkd} and its applications.
In \Cref{ch:method}, we describe the methods used to set up, execute and evaluate our simulation experiments.
\Cref{ch:results} describes the findings of these experiments.
We obtained results that show that increasing the expected number of incidents $N$,
such as what would occur when increasing the duration of the study,
reduces the selected bandwidth by a factor of around $N^{-1/6}$.
It also reduces the estimation error of the double kernel density in the relative sense
for global errors by a factor of around $N^{-3/4}$ and also for point-wise magnitude and peak errors.
However, in this case the absolute errors increased with the expected number of incidents.
We also observed that increases in the spread $\sigma$ of the risk (rate) function also results in reduced estimation error
in both absolute and relative terms of approximately $\sigma^{-1.4}$.
We also observed that increasing the expected number of incidents in tandem with the population size reduced the estimation error,
at a rate of around $N^{-2.7}$
and that having a non-uniform population increased the estimation error slightly, 
except when the population spread was extremely narrow.
\Cref{ch:discussion} contains a discussion of the implications of the results,
while \Cref{ch:further} suggests further research.
Finally, our conclusions and recommendations can be found in \Cref{ch:conclusion}.

