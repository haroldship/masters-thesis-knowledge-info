% !TEX root = thesis.tex

%%
%%
%% results tables appendix
%%
%%

%%
%% Section
\section{Uniform risk on a uniform population}

\begin{table}[H]
\centering
\scriptsize

    \begin{subtable}{0.5\textwidth}
    % latex table generated in R 3.3.3 by xtable 1.8-2 package
% Sun Mar 11 18:27:52 2018
\begin{tabular}{lrrr}
  \hline
 & Oracle & Silverman & CV \\ 
  \hline
MISE & 0.000012 & 0.000013 & 0.000013 \\ 
  Relative MISE & 0.116112 & 0.125979 & 0.128308 \\ 
  Normalized MISE & 0.000000 & 0.000000 & 0.000000 \\ 
  MIAE & 0.002786 & 0.002898 & 0.002938 \\ 
  Relative MIAE & 0.278620 & 0.289846 & 0.293819 \\ 
  Normalized MIAE & 0.000000 & 0.000000 & 0.000000 \\ 
  Max Error & 0.008551 & 0.009000 & 0.008926 \\ 
  Normalized Max Error & 0.000001 & 0.000001 & 0.000001 \\ 
   \hline
\end{tabular}

    \caption[]{Means} 
    \end{subtable}%
    \begin{subtable}{0.5\textwidth}
    % latex table generated in R 3.4.3 by xtable 1.8-2 package
% Tue Apr 03 12:03:44 2018
\begin{tabular}{lrrr}
  \hline
 & Oracle & Silverman & CV \\ 
  \hline
MISE & 0.000003 & 0.000003 & 0.000003 \\ 
  Relative MISE & 0.030194 & 0.032029 & 0.034292 \\ 
  Normalized MISE & 0.301940 & 0.320288 & 0.342922 \\ 
  MIAE & 0.000438 & 0.000418 & 0.000446 \\ 
  Relative MIAE & 0.043781 & 0.041769 & 0.044615 \\ 
  Normalized MIAE & 0.000004 & 0.000004 & 0.000004 \\ 
  Max Error & 0.000581 & 0.000870 & 0.001109 \\ 
  Normalized Max Error & 0.000006 & 0.000009 & 0.000011 \\ 
  Peak bias & 0.001757 & 0.002365 & 0.002891 \\ 
  Relative Peak bias & 0.175734 & 0.236474 & 0.289140 \\ 
  Peak drift & 1.467764 & 1.600954 & 1.558562 \\ 
  Relative Peak drift & 0.209681 & 0.228708 & 0.222652 \\ 
  Centroid bias & 0.002315 & 0.003475 & 0.003480 \\ 
  Relative Centroid bias & 0.231491 & 0.347454 & 0.347951 \\ 
  Centroid drift & 1.204587 & 1.197905 & 1.181971 \\ 
  Relative Centroid drift & 0.172084 & 0.171129 & 0.168853 \\ 
   \hline
\end{tabular}

    \caption[]{Standard deviations} 
    \end{subtable}

\caption[]{Error rates for uniform population of 10,000, uniform intensity of factor 100}
\label{tbl:mean_error_rates:unif_100_unif}
\end{table}


%%
%% Section
\section{Varying the number of cases for fixed population of 10,000}

\subsection{50 cases}
\begin{table}[H]
\centering
\scriptsize

    \begin{subtable}{0.5\textwidth}
    % latex table generated in R 3.3.3 by xtable 1.8-2 package
% Mon Jan 15 21:47:42 2018
\begin{tabular}{lrrr}
  \hline
 & Oracle & Silverman & CV \\ 
  \hline
MISE & 0.000009 & 0.000014 & 0.000014 \\ 
  Relative MISE & 0.005426 & 0.008763 & 0.008447 \\ 
  Normalized MISE & 0.000018 & 0.000029 & 0.000028 \\ 
  MIAE & 0.001587 & 0.001969 & 0.001934 \\ 
  Relative MIAE & 0.039303 & 0.048754 & 0.047876 \\ 
  Max Error & 0.012773 & 0.019068 & 0.018561 \\ 
  Peak bias & -0.006714 & 0.003707 & 0.003094 \\ 
  Relative Peak bias & -0.166236 & 0.091779 & 0.076597 \\ 
  Peak drift & 0.356080 & 0.496424 & 0.490986 \\ 
  Relative Peak drift & 0.050869 & 0.070918 & 0.070141 \\ 
  Centroid bias & -0.006972 & -0.000297 & -0.000533 \\ 
  Relative Centroid bias & -0.172619 & -0.007343 & -0.013205 \\ 
  Centroid drift & 0.276440 & 0.292158 & 0.291318 \\ 
  Relative Centroid drift & 0.039491 & 0.041737 & 0.041617 \\ 
   \hline
\end{tabular}

    \caption[]{Means} 
    \end{subtable}%
    \begin{subtable}{0.5\textwidth}
    % latex table generated in R 3.4.2 by xtable 1.8-2 package
% Sat Feb 17 16:49:42 2018
\begin{tabular}{lrrr}
  \hline
 & Oracle & Silverman & CV \\ 
  \hline
MISE & 0.000005 & 0.000006 & 0.000008 \\ 
  Relative MISE & 0.002899 & 0.003609 & 0.005053 \\ 
  Normalized MISE & 0.000009 & 0.000012 & 0.000016 \\ 
  MIAE & 0.000359 & 0.000326 & 0.000466 \\ 
  Relative MIAE & 0.008885 & 0.008064 & 0.011536 \\ 
  Max Error & 0.004163 & 0.005764 & 0.008681 \\ 
  Peak bias & 0.006126 & 0.010674 & 0.013822 \\ 
  Relative Peak bias & 0.151679 & 0.264272 & 0.342218 \\ 
  Peak drift & 0.191399 & 0.259172 & 0.233991 \\ 
  Relative Peak drift & 0.027343 & 0.037025 & 0.033427 \\ 
  Centroid bias & 0.006257 & 0.010519 & 0.011075 \\ 
  Relative Centroid bias & 0.154905 & 0.260428 & 0.274203 \\ 
  Centroid drift & 0.149444 & 0.157220 & 0.153568 \\ 
  Relative Centroid drift & 0.021349 & 0.022460 & 0.021938 \\ 
   \hline
\end{tabular}

    \caption[]{Standard deviations} 
    \end{subtable}

\caption[]{Error rates for uniform population of 10,000, single peak intensity of factor 50}
\label{tbl:mean_error_rates:unif_50_1.0_1h}
\end{table}

\subsection{100 cases}
\begin{table}[H]
\centering
\scriptsize

    \begin{subtable}{0.5\textwidth}
    % latex table generated in R 3.3.3 by xtable 1.8-2 package
% Mon Jan 15 21:47:29 2018
\begin{tabular}{lrrr}
  \hline
 & Oracle & Silverman & CV \\ 
  \hline
MISE & 0.000022 & 0.000035 & 0.000033 \\ 
  Relative MISE & 0.003402 & 0.005404 & 0.004996 \\ 
  Normalized MISE & 0.000022 & 0.000035 & 0.000033 \\ 
  MIAE & 0.002515 & 0.003126 & 0.003003 \\ 
  Relative MIAE & 0.031137 & 0.038695 & 0.037178 \\ 
  Max Error & 0.021087 & 0.031111 & 0.029319 \\ 
  Peak bias & -0.010671 & 0.006739 & 0.004427 \\ 
  Relative Peak bias & -0.132093 & 0.083426 & 0.054803 \\ 
  Peak drift & 0.288634 & 0.418493 & 0.401299 \\ 
  Relative Peak drift & 0.041233 & 0.059785 & 0.057328 \\ 
  Centroid bias & -0.011216 & -0.000155 & -0.001331 \\ 
  Relative Centroid bias & -0.138849 & -0.001917 & -0.016472 \\ 
  Centroid drift & 0.207657 & 0.222708 & 0.221023 \\ 
  Relative Centroid drift & 0.029665 & 0.031815 & 0.031575 \\ 
   \hline
\end{tabular}

    \caption[]{Means} 
    \end{subtable}%
    \begin{subtable}{0.5\textwidth}
    % latex table generated in R 3.3.3 by xtable 1.8-2 package
% Sun Mar 11 18:24:22 2018
\begin{tabular}{lrrr}
  \hline
 & Oracle & Silverman & CV \\ 
  \hline
MISE & 0.000011 & 0.000012 & 0.000021 \\ 
  Relative MISE & 0.001700 & 0.001896 & 0.003145 \\ 
  Normalized MISE & 0.000000 & 0.000000 & 0.000000 \\ 
  MIAE & 0.000526 & 0.000447 & 0.000719 \\ 
  Relative MIAE & 0.006506 & 0.005534 & 0.008895 \\ 
  Normalized MIAE & 0.000000 & 0.000000 & 0.000000 \\ 
  Max Error & 0.006576 & 0.008019 & 0.013486 \\ 
  Normalized Max Error & 0.000001 & 0.000001 & 0.000001 \\ 
  Peak bias & 0.009684 & 0.015381 & 0.020691 \\ 
  Relative Peak bias & 0.119873 & 0.190401 & 0.256135 \\ 
  Peak drift & 0.154869 & 0.218750 & 0.200306 \\ 
  Relative Peak drift & 0.022124 & 0.031250 & 0.028615 \\ 
  Centroid bias & 0.009891 & 0.015892 & 0.016018 \\ 
  Relative Centroid bias & 0.122436 & 0.196722 & 0.198285 \\ 
  Centroid drift & 0.114681 & 0.121309 & 0.117012 \\ 
  Relative Centroid drift & 0.016383 & 0.017330 & 0.016716 \\ 
   \hline
\end{tabular}

    \caption[]{Standard deviations} 
    \end{subtable}

\caption[]{Error rates for uniform population of 10,000, single peak intensity of factor 100}
\label{tbl:mean_error_rates:unif_100_1.0_1h}
\end{table}

\subsection{200 cases}
\begin{table}[H]
\centering
\scriptsize

    \begin{subtable}{0.5\textwidth}
    % latex table generated in R 3.3.3 by xtable 1.8-2 package
% Sun Mar 11 18:28:48 2018
\begin{tabular}{lrrr}
  \hline
 & Oracle & Silverman & CV \\ 
  \hline
MISE & 0.000053 & 0.000083 & 0.000080 \\ 
  Relative MISE & 0.002028 & 0.003195 & 0.003047 \\ 
  Normalized MISE & 0.000000 & 0.000000 & 0.000000 \\ 
  MIAE & 0.003914 & 0.004863 & 0.004653 \\ 
  Relative MIAE & 0.024227 & 0.030102 & 0.028802 \\ 
  Normalized MIAE & 0.000000 & 0.000000 & 0.000000 \\ 
  Max Error & 0.033855 & 0.049415 & 0.045884 \\ 
  Normalized Max Error & 0.000001 & 0.000001 & 0.000001 \\ 
  Peak bias & -0.016050 & 0.010816 & -0.002953 \\ 
  Relative Peak bias & -0.099339 & 0.066949 & -0.018278 \\ 
  Peak drift & 0.223424 & 0.357624 & 0.284159 \\ 
  Relative Peak drift & 0.031918 & 0.051089 & 0.040594 \\ 
  Centroid bias & -0.017140 & -0.000902 & -0.013918 \\ 
  Relative Centroid bias & -0.106087 & -0.005583 & -0.086149 \\ 
  Centroid drift & 0.149249 & 0.159867 & 0.152694 \\ 
  Relative Centroid drift & 0.021321 & 0.022838 & 0.021813 \\ 
   \hline
\end{tabular}

    \caption[]{Means} 
    \end{subtable}%
    \begin{subtable}{0.5\textwidth}
    % latex table generated in R 3.4.2 by xtable 1.8-2 package
% Thu Feb 15 19:57:40 2018
\begin{tabular}{lrrr}
  \hline
 & Oracle & Silverman & CV \\ 
  \hline
MISE & 0.000089 & 0.000116 & 0.000228 \\ 
  Relative MISE & 0.000131 & 0.000171 & 0.000335 \\ 
  Normalized MISE & 0.000018 & 0.000023 & 0.000046 \\ 
  MIAE & 0.000911 & 0.000820 & 0.001681 \\ 
  Relative MIAE & 0.001105 & 0.000994 & 0.002040 \\ 
  Max Error & 0.024919 & 0.036302 & 0.058950 \\ 
  Peak bias & 0.031372 & 0.045576 & 0.062499 \\ 
  Relative Peak bias & 0.038067 & 0.055302 & 0.075836 \\ 
  Peak drift & 0.060193 & 0.086240 & 0.111909 \\ 
  Relative Peak drift & 0.008599 & 0.012320 & 0.015987 \\ 
  Centroid bias & 0.031457 & 0.047196 & 0.046071 \\ 
  Relative Centroid bias & 0.038170 & 0.057267 & 0.055902 \\ 
  Centroid drift & 0.051758 & 0.054538 & 0.052729 \\ 
  Relative Centroid drift & 0.007394 & 0.007791 & 0.007533 \\ 
   \hline
\end{tabular}

    \caption[]{Standard deviations} 
    \end{subtable}

\caption[]{Error rates for uniform population of 10,000, single peak intensity of factor 200}
\label{tbl:mean_error_rates:unif_200_1.0_1h}
\end{table}

\subsection{500 cases}
\begin{table}[H]
\centering
\scriptsize

    \begin{subtable}{0.5\textwidth}
    % latex table generated in R 3.3.3 by xtable 1.8-2 package
% Mon Jan 15 21:46:56 2018
\begin{tabular}{lrrr}
  \hline
 & Oracle & Silverman & CV \\ 
  \hline
MISE & 0.000170 & 0.000267 & 0.000289 \\ 
  Relative MISE & 0.001041 & 0.001635 & 0.001771 \\ 
  Normalized MISE & 0.000034 & 0.000053 & 0.000058 \\ 
  MIAE & 0.007091 & 0.008789 & 0.008088 \\ 
  Relative MIAE & 0.017555 & 0.021760 & 0.020024 \\ 
  Max Error & 0.063876 & 0.092773 & 0.075377 \\ 
  Peak bias & -0.029045 & 0.019359 & -0.029269 \\ 
  Relative Peak bias & -0.071910 & 0.047930 & -0.072465 \\ 
  Peak drift & 0.162605 & 0.293576 & 0.212179 \\ 
  Relative Peak drift & 0.023229 & 0.041939 & 0.030311 \\ 
  Centroid bias & -0.031287 & -0.003206 & -0.037054 \\ 
  Relative Centroid bias & -0.077460 & -0.007938 & -0.091740 \\ 
  Centroid drift & 0.094390 & 0.103873 & 0.099062 \\ 
  Relative Centroid drift & 0.013484 & 0.014839 & 0.014152 \\ 
   \hline
\end{tabular}

    \caption[]{Means} 
    \end{subtable}%
    \begin{subtable}{0.5\textwidth}
    % latex table generated in R 3.3.3 by xtable 1.8-2 package
% Mon Jan 15 21:46:56 2018
\begin{tabular}{lrrr}
  \hline
 & Oracle & Silverman & CV \\ 
  \hline
MISE & 0.000067 & 0.000067 & 0.000930 \\ 
  Relative MISE & 0.000410 & 0.000409 & 0.005699 \\ 
  Normalized MISE & 0.000013 & 0.000013 & 0.000186 \\ 
  MIAE & 0.001132 & 0.000907 & 0.004417 \\ 
  Relative MIAE & 0.002802 & 0.002246 & 0.010936 \\ 
  Max Error & 0.016888 & 0.019592 & 0.038426 \\ 
  Peak bias & 0.024544 & 0.033904 & 0.051170 \\ 
  Relative Peak bias & 0.060766 & 0.083940 & 0.126688 \\ 
  Peak drift & 0.085657 & 0.145926 & 0.385511 \\ 
  Relative Peak drift & 0.012237 & 0.020847 & 0.055073 \\ 
  Centroid bias & 0.024858 & 0.037166 & 0.045331 \\ 
  Relative Centroid bias & 0.061544 & 0.092017 & 0.112230 \\ 
  Centroid drift & 0.061841 & 0.066143 & 0.092926 \\ 
  Relative Centroid drift & 0.008834 & 0.009449 & 0.013275 \\ 
   \hline
\end{tabular}

    \caption[]{Standard deviations} 
    \end{subtable}

\caption[]{Error rates for uniform population of 10,000, single peak intensity of factor 500}
\label{tbl:mean_error_rates:unif_500_1.0_1h}
\end{table}

\subsection{1000 cases}
\begin{table}[H]
\centering
\scriptsize

    \begin{subtable}{0.5\textwidth}
    % latex table generated in R 3.3.3 by xtable 1.8-2 package
% Mon Jan 15 21:46:18 2018
\begin{tabular}{lrrr}
  \hline
 & Oracle & Silverman & CV \\ 
  \hline
MISE & 0.000382 & 0.000621 & 0.001140 \\ 
  Relative MISE & 0.000586 & 0.000952 & 0.001747 \\ 
  Normalized MISE & 0.000038 & 0.000062 & 0.000114 \\ 
  MIAE & 0.010843 & 0.013613 & 0.014720 \\ 
  Relative MIAE & 0.013423 & 0.016852 & 0.018222 \\ 
  Max Error & 0.105785 & 0.145689 & 0.153793 \\ 
  Peak bias & -0.035487 & 0.022078 & 0.010450 \\ 
  Relative Peak bias & -0.043930 & 0.027330 & 0.012936 \\ 
  Peak drift & 0.128748 & 0.247941 & 0.245312 \\ 
  Relative Peak drift & 0.018393 & 0.035420 & 0.035045 \\ 
  Centroid bias & -0.038838 & -0.003767 & -0.014905 \\ 
  Relative Centroid bias & -0.048077 & -0.004663 & -0.018451 \\ 
  Centroid drift & 0.068777 & 0.075322 & 0.074997 \\ 
  Relative Centroid drift & 0.009825 & 0.010760 & 0.010714 \\ 
   \hline
\end{tabular}

    \caption[]{Means} 
    \end{subtable}%
    \begin{subtable}{0.5\textwidth}
    % latex table generated in R 3.3.3 by xtable 1.8-2 package
% Mon Jan 15 21:46:18 2018
\begin{tabular}{lrrr}
  \hline
 & Oracle & Silverman & CV \\ 
  \hline
MISE & 0.000111 & 0.000115 & 0.004540 \\ 
  Relative MISE & 0.000170 & 0.000177 & 0.006957 \\ 
  Normalized MISE & 0.000011 & 0.000012 & 0.000454 \\ 
  MIAE & 0.001340 & 0.001082 & 0.009854 \\ 
  Relative MIAE & 0.001659 & 0.001339 & 0.012198 \\ 
  Max Error & 0.022920 & 0.025550 & 0.079460 \\ 
  Peak bias & 0.025834 & 0.033150 & 0.099915 \\ 
  Relative Peak bias & 0.031980 & 0.041037 & 0.123686 \\ 
  Peak drift & 0.068326 & 0.114098 & 0.113555 \\ 
  Relative Peak drift & 0.009761 & 0.016300 & 0.016222 \\ 
  Centroid bias & 0.025928 & 0.035946 & 0.097946 \\ 
  Relative Centroid bias & 0.032096 & 0.044498 & 0.121248 \\ 
  Centroid drift & 0.054706 & 0.055042 & 0.055304 \\ 
  Relative Centroid drift & 0.007815 & 0.007863 & 0.007901 \\ 
   \hline
\end{tabular}

    \caption[]{Standard deviations} 
    \end{subtable}

\caption[]{Error rates for uniform population of 10,000, single peak intensity of factor 1000}
\label{tbl:mean_error_rates:unif_1000_1.0_1h}
\end{table}


%%
%% Section
\section{Varying population and cases together}

\subsection{100 cases from 10,000}

See \autoref{tbl:mean_error_rates:unif_100_1.0_1h}.

\begin{table}[H]
\centering
\scriptsize

    \begin{subtable}{0.5\textwidth}
    % latex table generated in R 3.3.3 by xtable 1.8-2 package
% Mon Jan 15 21:47:29 2018
\begin{tabular}{lrrr}
  \hline
 & Oracle & Silverman & CV \\ 
  \hline
MISE & 0.000022 & 0.000035 & 0.000033 \\ 
  Relative MISE & 0.003402 & 0.005404 & 0.004996 \\ 
  Normalized MISE & 0.000022 & 0.000035 & 0.000033 \\ 
  MIAE & 0.002515 & 0.003126 & 0.003003 \\ 
  Relative MIAE & 0.031137 & 0.038695 & 0.037178 \\ 
  Max Error & 0.021087 & 0.031111 & 0.029319 \\ 
  Peak bias & -0.010671 & 0.006739 & 0.004427 \\ 
  Relative Peak bias & -0.132093 & 0.083426 & 0.054803 \\ 
  Peak drift & 0.288634 & 0.418493 & 0.401299 \\ 
  Relative Peak drift & 0.041233 & 0.059785 & 0.057328 \\ 
  Centroid bias & -0.011216 & -0.000155 & -0.001331 \\ 
  Relative Centroid bias & -0.138849 & -0.001917 & -0.016472 \\ 
  Centroid drift & 0.207657 & 0.222708 & 0.221023 \\ 
  Relative Centroid drift & 0.029665 & 0.031815 & 0.031575 \\ 
   \hline
\end{tabular}

    \caption[]{Means} 
    \end{subtable}%
    \begin{subtable}{0.5\textwidth}
    % latex table generated in R 3.3.3 by xtable 1.8-2 package
% Sun Mar 11 18:24:22 2018
\begin{tabular}{lrrr}
  \hline
 & Oracle & Silverman & CV \\ 
  \hline
MISE & 0.000011 & 0.000012 & 0.000021 \\ 
  Relative MISE & 0.001700 & 0.001896 & 0.003145 \\ 
  Normalized MISE & 0.000000 & 0.000000 & 0.000000 \\ 
  MIAE & 0.000526 & 0.000447 & 0.000719 \\ 
  Relative MIAE & 0.006506 & 0.005534 & 0.008895 \\ 
  Normalized MIAE & 0.000000 & 0.000000 & 0.000000 \\ 
  Max Error & 0.006576 & 0.008019 & 0.013486 \\ 
  Normalized Max Error & 0.000001 & 0.000001 & 0.000001 \\ 
  Peak bias & 0.009684 & 0.015381 & 0.020691 \\ 
  Relative Peak bias & 0.119873 & 0.190401 & 0.256135 \\ 
  Peak drift & 0.154869 & 0.218750 & 0.200306 \\ 
  Relative Peak drift & 0.022124 & 0.031250 & 0.028615 \\ 
  Centroid bias & 0.009891 & 0.015892 & 0.016018 \\ 
  Relative Centroid bias & 0.122436 & 0.196722 & 0.198285 \\ 
  Centroid drift & 0.114681 & 0.121309 & 0.117012 \\ 
  Relative Centroid drift & 0.016383 & 0.017330 & 0.016716 \\ 
   \hline
\end{tabular}

    \caption[]{Standard deviations} 
    \end{subtable}

\caption[]{Error rates for uniform population of 10,000, single peak intensity of factor 100}
\label{tbl:mean_error_rates:unif_100_1.0_1h:2}
\end{table}

\subsection{200 cases from 20,000}
\begin{table}[H]
\centering
\scriptsize

    \begin{subtable}{0.5\textwidth}
    % latex table generated in R 3.4.0 by xtable 1.8-2 package
% Sun Aug 13 13:20:48 2017
\begin{table}[ht]
\centering
\begin{tabular}{rrrr}
  \hline
 & Oracle & Silverman & CV \\ 
  \hline
MISE & 0.000015 & 0.000023 & 0.000022 \\ 
  Relative MISE & 0.002372 & 0.003481 & 0.003425 \\ 
  MIAE & 0.002069 & 0.002496 & 0.002476 \\ 
  Relative MIAE & 0.025606 & 0.030901 & 0.030651 \\ 
  Max Error & 0.018984 & 0.026091 & 0.025789 \\ 
  Peak bias & -0.011406 & 0.001175 & 0.000799 \\ 
  Relative Peak bias & -0.141193 & 0.014544 & 0.009891 \\ 
  Peak drift & 0.291837 & 0.422926 & 0.420581 \\ 
  Relative Peak drift & 0.041691 & 0.060418 & 0.060083 \\ 
  Centroid bias & -0.012288 & -0.006236 & -0.006300 \\ 
  Relative Centroid bias & -0.152112 & -0.077190 & -0.077988 \\ 
  Centroid drift & 0.171960 & 0.190351 & 0.190355 \\ 
  Relative Centroid drift & 0.024566 & 0.027193 & 0.027194 \\ 
   \hline
\end{tabular}
\caption{Mean error rates} 
\label{tbl:mean_error_rates}
\end{table}

    \caption[]{Means} 
    \end{subtable}%
    \begin{subtable}{0.5\textwidth}
    % latex table generated in R 3.4.0 by xtable 1.8-2 package
% Sun Aug 13 13:20:48 2017
\begin{tabular}{rrrr}
  \hline
 & Oracle & Silverman & CV \\ 
  \hline
MISE & 0.000007 & 0.000007 & 0.000007 \\ 
  Relative MISE & 0.001051 & 0.001092 & 0.001093 \\ 
  MIAE & 0.000370 & 0.000314 & 0.000318 \\ 
  Relative MIAE & 0.004574 & 0.003892 & 0.003931 \\ 
  Max Error & 0.005609 & 0.005995 & 0.006009 \\ 
  Peak bias & 0.007725 & 0.011211 & 0.011137 \\ 
  Relative Peak bias & 0.095629 & 0.138785 & 0.137868 \\ 
  Peak drift & 0.155419 & 0.211609 & 0.209799 \\ 
  Relative Peak drift & 0.022203 & 0.030230 & 0.029971 \\ 
  Centroid bias & 0.007989 & 0.012125 & 0.012026 \\ 
  Relative Centroid bias & 0.098899 & 0.150096 & 0.148867 \\ 
  Centroid drift & 0.097875 & 0.103910 & 0.104528 \\ 
  Relative Centroid drift & 0.013982 & 0.014844 & 0.014933 \\ 
   \hline
\end{tabular}

    \caption[]{Standard deviations} 
    \end{subtable}

\caption[]{Error rates for uniform population of 20,000, single peak intensity of factor 200}
\label{tbl:mean_error_rates:unif20k_200_1.0_1h}
\end{table}

\subsection{400 cases from 40,000}
\begin{table}[H]
\centering
\scriptsize

    \begin{subtable}{0.5\textwidth}
    % latex table generated in R 3.4.0 by xtable 1.8-2 package
% Sun Aug 13 14:16:06 2017
\begin{table}[ht]
\centering
\begin{tabular}{rrrr}
  \hline
 & Oracle & Silverman & CV \\ 
  \hline
MISE & 0.000010 & 0.000014 & 0.000013 \\ 
  Relative MISE & 0.001506 & 0.002142 & 0.002051 \\ 
  MIAE & 0.001642 & 0.001969 & 0.001926 \\ 
  Relative MIAE & 0.020323 & 0.024376 & 0.023837 \\ 
  Max Error & 0.015691 & 0.021412 & 0.020759 \\ 
  Peak bias & -0.009361 & 0.001818 & 0.000969 \\ 
  Relative Peak bias & -0.115878 & 0.022501 & 0.011994 \\ 
  Peak drift & 0.224460 & 0.348102 & 0.343680 \\ 
  Relative Peak drift & 0.032066 & 0.049729 & 0.049097 \\ 
  Centroid bias & -0.010033 & -0.004142 & -0.004456 \\ 
  Relative Centroid bias & -0.124197 & -0.051270 & -0.055156 \\ 
  Centroid drift & 0.125798 & 0.138387 & 0.137610 \\ 
  Relative Centroid drift & 0.017971 & 0.019770 & 0.019659 \\ 
   \hline
\end{tabular}
\caption{Mean error rates} 
\label{tbl:mean_error_rates}
\end{table}

    \caption[]{Means} 
    \end{subtable}%
    \begin{subtable}{0.5\textwidth}
    % latex table generated in R 3.4.0 by xtable 1.8-2 package
% Sun Aug 13 14:16:06 2017
\begin{table}[ht]
\centering
\begin{tabular}{rrrr}
  \hline
 & Oracle & Silverman & CV \\ 
  \hline
MISE & 0.000004 & 0.000004 & 0.000004 \\ 
  Relative MISE & 0.000659 & 0.000601 & 0.000609 \\ 
  MIAE & 0.000278 & 0.000220 & 0.000229 \\ 
  Relative MIAE & 0.003442 & 0.002723 & 0.002830 \\ 
  Max Error & 0.004435 & 0.004672 & 0.004730 \\ 
  Peak bias & 0.006121 & 0.008856 & 0.008732 \\ 
  Relative Peak bias & 0.075776 & 0.109626 & 0.108095 \\ 
  Peak drift & 0.116624 & 0.171068 & 0.166589 \\ 
  Relative Peak drift & 0.016661 & 0.024438 & 0.023798 \\ 
  Centroid bias & 0.006271 & 0.009763 & 0.009491 \\ 
  Relative Centroid bias & 0.077629 & 0.120854 & 0.117487 \\ 
  Centroid drift & 0.073622 & 0.078391 & 0.079245 \\ 
  Relative Centroid drift & 0.010517 & 0.011199 & 0.011321 \\ 
   \hline
\end{tabular}
\caption{Standard deviation of error rates} 
\label{tbl:stddev_error_rates}
\end{table}

    \caption[]{Standard deviations} 
    \end{subtable}

\caption[]{Error rates for uniform population of 40,000, single peak intensity of factor 400}
\label{tbl:mean_error_rates:unif40k_400_1.0_1h}
\end{table}

\subsection{600 cases from 60,000}
\begin{table}[H]
\centering
\scriptsize

    \begin{subtable}{0.5\textwidth}
    % latex table generated in R 3.4.0 by xtable 1.8-2 package
% Sun Aug 13 15:09:40 2017
\begin{table}[ht]
\centering
\begin{tabular}{rrrr}
  \hline
 & Oracle & Silverman & CV \\ 
  \hline
MISE & 0.000007 & 0.000011 & 0.000010 \\ 
  Relative MISE & 0.001127 & 0.001630 & 0.001509 \\ 
  MIAE & 0.001426 & 0.001718 & 0.001650 \\ 
  Relative MIAE & 0.017653 & 0.021262 & 0.020430 \\ 
  Max Error & 0.013768 & 0.019194 & 0.018128 \\ 
  Peak bias & -0.007019 & 0.002555 & 0.001164 \\ 
  Relative Peak bias & -0.086889 & 0.031633 & 0.014413 \\ 
  Peak drift & 0.183350 & 0.302506 & 0.287383 \\ 
  Relative Peak drift & 0.026193 & 0.043215 & 0.041055 \\ 
  Centroid bias & -0.007706 & -0.003067 & -0.003522 \\ 
  Relative Centroid bias & -0.095392 & -0.037962 & -0.043602 \\ 
  Centroid drift & 0.097719 & 0.112315 & 0.111326 \\ 
  Relative Centroid drift & 0.013960 & 0.016045 & 0.015904 \\ 
   \hline
\end{tabular}
\caption{Mean error rates} 
\label{tbl:mean_error_rates}
\end{table}

    \caption[]{Means} 
    \end{subtable}%
    \begin{subtable}{0.5\textwidth}
    % latex table generated in R 3.4.0 by xtable 1.8-2 package
% Sun Aug 13 15:09:40 2017
\begin{tabular}{rrrr}
  \hline
 & Oracle & Silverman & CV \\ 
  \hline
MISE & 0.000003 & 0.000003 & 0.000003 \\ 
  Relative MISE & 0.000484 & 0.000431 & 0.000453 \\ 
  MIAE & 0.000231 & 0.000177 & 0.000195 \\ 
  Relative MIAE & 0.002861 & 0.002194 & 0.002409 \\ 
  Max Error & 0.003864 & 0.004055 & 0.004263 \\ 
  Peak bias & 0.005921 & 0.008327 & 0.008273 \\ 
  Relative Peak bias & 0.073291 & 0.103080 & 0.102413 \\ 
  Peak drift & 0.106179 & 0.150642 & 0.148199 \\ 
  Relative Peak drift & 0.015168 & 0.021520 & 0.021171 \\ 
  Centroid bias & 0.006052 & 0.008976 & 0.008647 \\ 
  Relative Centroid bias & 0.074912 & 0.111120 & 0.107043 \\ 
  Centroid drift & 0.064924 & 0.069500 & 0.068346 \\ 
  Relative Centroid drift & 0.009275 & 0.009929 & 0.009764 \\ 
   \hline
\end{tabular}

    \caption[]{Standard deviations} 
    \end{subtable}

\caption[]{Error rates for uniform population of 60,000, single peak intensity of factor 600}
\label{tbl:mean_error_rates:unif60k_600_1.0_1h}
\end{table}

\subsection{800 cases from 80,000}
\begin{table}[H]
\centering
\scriptsize

    \begin{subtable}{0.5\textwidth}
    % latex table generated in R 3.4.0 by xtable 1.8-2 package
% Sun Aug 13 16:07:13 2017
\begin{tabular}{rrrr}
  \hline
 & Oracle & Silverman & CV \\ 
  \hline
MISE & 0.000006 & 0.000009 & 0.000008 \\ 
  Relative MISE & 0.000891 & 0.001335 & 0.001173 \\ 
  MIAE & 0.001278 & 0.001556 & 0.001458 \\ 
  Relative MIAE & 0.015821 & 0.019260 & 0.018043 \\ 
  Max Error & 0.012651 & 0.017620 & 0.016067 \\ 
  Peak bias & -0.006430 & 0.002183 & 0.000162 \\ 
  Relative Peak bias & -0.079594 & 0.027022 & 0.002006 \\ 
  Peak drift & 0.189578 & 0.304958 & 0.282499 \\ 
  Relative Peak drift & 0.027083 & 0.043565 & 0.040357 \\ 
  Centroid bias & -0.007286 & -0.003685 & -0.004249 \\ 
  Relative Centroid bias & -0.090199 & -0.045614 & -0.052603 \\ 
  Centroid drift & 0.083142 & 0.094548 & 0.091757 \\ 
  Relative Centroid drift & 0.011877 & 0.013507 & 0.013108 \\ 
   \hline
\end{tabular}

    \caption[]{Means} 
    \end{subtable}%
    \begin{subtable}{0.5\textwidth}
    % latex table generated in R 3.4.0 by xtable 1.8-2 package
% Sun Aug 13 16:07:13 2017
\begin{tabular}{rrrr}
  \hline
 & Oracle & Silverman & CV \\ 
  \hline
MISE & 0.000002 & 0.000002 & 0.000002 \\ 
  Relative MISE & 0.000342 & 0.000328 & 0.000353 \\ 
  MIAE & 0.000186 & 0.000146 & 0.000173 \\ 
  Relative MIAE & 0.002300 & 0.001813 & 0.002141 \\ 
  Max Error & 0.003414 & 0.003799 & 0.003941 \\ 
  Peak bias & 0.005078 & 0.007028 & 0.006866 \\ 
  Relative Peak bias & 0.062864 & 0.086995 & 0.084993 \\ 
  Peak drift & 0.106356 & 0.140820 & 0.133500 \\ 
  Relative Peak drift & 0.015194 & 0.020117 & 0.019071 \\ 
  Centroid bias & 0.005216 & 0.007790 & 0.007253 \\ 
  Relative Centroid bias & 0.064570 & 0.096434 & 0.089789 \\ 
  Centroid drift & 0.061492 & 0.062965 & 0.062084 \\ 
  Relative Centroid drift & 0.008785 & 0.008995 & 0.008869 \\ 
   \hline
\end{tabular}

    \caption[]{Standard deviations} 
    \end{subtable}

\caption[]{Error rates for uniform population of 80,000, single peak intensity of factor 800}
\label{tbl:mean_error_rates:unif80k_800_1.0_1h}
\end{table}

\subsection{1000 cases from 100,000}
\begin{table}[H]
\centering
\scriptsize

    \begin{subtable}{0.5\textwidth}
    % latex table generated in R 3.4.0 by xtable 1.8-2 package
% Sun Jul 30 00:05:45 2017
\begin{table}[ht]
\centering
\begin{tabular}{rrrr}
  \hline
 & Oracle & Silverman & CV \\ 
  \hline
MISE & 0.000002 & 0.000002 & 0.000002 \\ 
  Relative MISE & 0.000296 & 0.000275 & 0.000312 \\ 
  MIAE & 0.000172 & 0.000134 & 0.000169 \\ 
  Relative MIAE & 0.002123 & 0.001663 & 0.002093 \\ 
  Max Error & 0.003190 & 0.003425 & 0.003799 \\ 
  Peak bias & 0.004776 & 0.006378 & 0.006163 \\ 
  Relative Peak bias & 0.059119 & 0.078949 & 0.076294 \\ 
  Peak drift & 0.106483 & 0.145200 & 0.134117 \\ 
  Relative Peak drift & 0.015212 & 0.020743 & 0.019160 \\ 
  Centroid bias & 0.004937 & 0.007348 & 0.006476 \\ 
  Relative Centroid bias & 0.061119 & 0.090964 & 0.080164 \\ 
  Centroid drift & 0.059317 & 0.062619 & 0.061337 \\ 
  Relative Centroid drift & 0.008474 & 0.008946 & 0.008762 \\ 
   \hline
\end{tabular}
\caption{Standard deviation of error rates} 
\label{tbl:stddev_error_rates}
\end{table}

    \caption[]{Means} 
    \end{subtable}%
    \begin{subtable}{0.5\textwidth}
    % latex table generated in R 3.4.2 by xtable 1.8-2 package
% Thu Feb 15 19:57:40 2018
\begin{tabular}{lrrr}
  \hline
 & Oracle & Silverman & CV \\ 
  \hline
MISE & 0.000089 & 0.000116 & 0.000228 \\ 
  Relative MISE & 0.000131 & 0.000171 & 0.000335 \\ 
  Normalized MISE & 0.000018 & 0.000023 & 0.000046 \\ 
  MIAE & 0.000911 & 0.000820 & 0.001681 \\ 
  Relative MIAE & 0.001105 & 0.000994 & 0.002040 \\ 
  Max Error & 0.024919 & 0.036302 & 0.058950 \\ 
  Peak bias & 0.031372 & 0.045576 & 0.062499 \\ 
  Relative Peak bias & 0.038067 & 0.055302 & 0.075836 \\ 
  Peak drift & 0.060193 & 0.086240 & 0.111909 \\ 
  Relative Peak drift & 0.008599 & 0.012320 & 0.015987 \\ 
  Centroid bias & 0.031457 & 0.047196 & 0.046071 \\ 
  Relative Centroid bias & 0.038170 & 0.057267 & 0.055902 \\ 
  Centroid drift & 0.051758 & 0.054538 & 0.052729 \\ 
  Relative Centroid drift & 0.007394 & 0.007791 & 0.007533 \\ 
   \hline
\end{tabular}

    \caption[]{Standard deviations} 
    \end{subtable}

\caption[]{Error rates for uniform population of 100,000, single peak intensity of factor 1000}
\label{tbl:mean_error_rates:unif100k_1000_1.0_1h}
\end{table}


%%
%% Section
\section{Varying the decay of the risk function}


\subsection{100 cases from 10,000 with no decay (uniform)}

See \autoref{tbl:mean_error_rates:unif_100_unif}.

\begin{table}[H]
\centering
\scriptsize

    \begin{subtable}{0.5\textwidth}
    % latex table generated in R 3.3.3 by xtable 1.8-2 package
% Sun Mar 11 18:27:52 2018
\begin{tabular}{lrrr}
  \hline
 & Oracle & Silverman & CV \\ 
  \hline
MISE & 0.000012 & 0.000013 & 0.000013 \\ 
  Relative MISE & 0.116112 & 0.125979 & 0.128308 \\ 
  Normalized MISE & 0.000000 & 0.000000 & 0.000000 \\ 
  MIAE & 0.002786 & 0.002898 & 0.002938 \\ 
  Relative MIAE & 0.278620 & 0.289846 & 0.293819 \\ 
  Normalized MIAE & 0.000000 & 0.000000 & 0.000000 \\ 
  Max Error & 0.008551 & 0.009000 & 0.008926 \\ 
  Normalized Max Error & 0.000001 & 0.000001 & 0.000001 \\ 
   \hline
\end{tabular}

    \caption[]{Means} 
    \end{subtable}%
    \begin{subtable}{0.5\textwidth}
    % latex table generated in R 3.4.3 by xtable 1.8-2 package
% Tue Apr 03 12:03:44 2018
\begin{tabular}{lrrr}
  \hline
 & Oracle & Silverman & CV \\ 
  \hline
MISE & 0.000003 & 0.000003 & 0.000003 \\ 
  Relative MISE & 0.030194 & 0.032029 & 0.034292 \\ 
  Normalized MISE & 0.301940 & 0.320288 & 0.342922 \\ 
  MIAE & 0.000438 & 0.000418 & 0.000446 \\ 
  Relative MIAE & 0.043781 & 0.041769 & 0.044615 \\ 
  Normalized MIAE & 0.000004 & 0.000004 & 0.000004 \\ 
  Max Error & 0.000581 & 0.000870 & 0.001109 \\ 
  Normalized Max Error & 0.000006 & 0.000009 & 0.000011 \\ 
  Peak bias & 0.001757 & 0.002365 & 0.002891 \\ 
  Relative Peak bias & 0.175734 & 0.236474 & 0.289140 \\ 
  Peak drift & 1.467764 & 1.600954 & 1.558562 \\ 
  Relative Peak drift & 0.209681 & 0.228708 & 0.222652 \\ 
  Centroid bias & 0.002315 & 0.003475 & 0.003480 \\ 
  Relative Centroid bias & 0.231491 & 0.347454 & 0.347951 \\ 
  Centroid drift & 1.204587 & 1.197905 & 1.181971 \\ 
  Relative Centroid drift & 0.172084 & 0.171129 & 0.168853 \\ 
   \hline
\end{tabular}

    \caption[]{Standard deviations} 
    \end{subtable}

\caption[]{Error rates for uniform population of 10,000, single peak intensity of factor 100 and no decay (uniform)}
\label{tbl:mean_error_rates:unif_100_unif:2}
\end{table}

\subsection{100 cases from 10,000 with intensity decay rate 2.0}
\begin{table}[H]
\centering
\scriptsize

    \begin{subtable}{0.5\textwidth}
    % latex table generated in R 3.4.2 by xtable 1.8-2 package
% Thu Dec  7 16:55:48 2017
\begin{tabular}{lrrr}
  \hline
 & Oracle & Silverman & CV \\ 
  \hline
MISE & 0.000006 & 0.000012 & 0.000012 \\ 
  Relative MISE & 0.011646 & 0.021523 & 0.021507 \\ 
  MIAE & 0.001892 & 0.002638 & 0.002637 \\ 
  Relative MIAE & 0.080572 & 0.112335 & 0.112292 \\ 
  Max Error & 0.006325 & 0.010554 & 0.010549 \\ 
  Peak bias & -0.003394 & 0.003098 & 0.003092 \\ 
  Relative Peak bias & -0.144492 & 0.131917 & 0.131640 \\ 
  Peak drift & 0.620415 & 0.916341 & 0.916190 \\ 
  Relative Peak drift & 0.088631 & 0.130906 & 0.130884 \\ 
  Centroid bias & -0.003436 & 0.001731 & 0.001728 \\ 
  Relative Centroid bias & -0.146307 & 0.073708 & 0.073569 \\ 
  Centroid drift & 0.541248 & 0.678653 & 0.678470 \\ 
  Relative Centroid drift & 0.077321 & 0.096950 & 0.096924 \\ 
   \hline
\end{tabular}

    \caption[]{Means} 
    \end{subtable}%
    \begin{subtable}{0.5\textwidth}
    % latex table generated in R 3.4.2 by xtable 1.8-2 package
% Thu Dec  7 16:55:48 2017
\begin{tabular}{lrrr}
  \hline
 & Oracle & Silverman & CV \\ 
  \hline
MISE & 0.000004 & 0.000004 & 0.000004 \\ 
  Relative MISE & 0.006636 & 0.007370 & 0.007366 \\ 
  MIAE & 0.000525 & 0.000411 & 0.000411 \\ 
  Relative MIAE & 0.022360 & 0.017479 & 0.017480 \\ 
  Max Error & 0.001951 & 0.002649 & 0.002647 \\ 
  Peak bias & 0.002931 & 0.005085 & 0.005083 \\ 
  Relative Peak bias & 0.124806 & 0.216496 & 0.216407 \\ 
  Peak drift & 0.338067 & 0.505238 & 0.505175 \\ 
  Relative Peak drift & 0.048295 & 0.072177 & 0.072168 \\ 
  Centroid bias & 0.002954 & 0.005691 & 0.005687 \\ 
  Relative Centroid bias & 0.125760 & 0.242311 & 0.242133 \\ 
  Centroid drift & 0.289593 & 0.366093 & 0.366201 \\ 
  Relative Centroid drift & 0.041370 & 0.052299 & 0.052314 \\ 
   \hline
\end{tabular}

    \caption[]{Standard deviations} 
    \end{subtable}

\caption[]{Error rates for uniform population of 10,000, single peak intensity of factor 100 and decay rate 2.0}
\label{tbl:mean_error_rates:unif_100_2_1h}
\end{table}

\subsection{100 cases from 10,000 with intensity decay rate 1.4}
\begin{table}[H]
\centering
\scriptsize

    \begin{subtable}{0.5\textwidth}
    % latex table generated in R 3.4.2 by xtable 1.8-2 package
% Thu Dec  7 17:09:22 2017
\begin{tabular}{lrrr}
  \hline
 & Oracle & Silverman & CV \\ 
  \hline
MISE & 0.000012 & 0.000019 & 0.000019 \\ 
  Relative MISE & 0.006627 & 0.010652 & 0.010635 \\ 
  MIAE & 0.002291 & 0.002949 & 0.002946 \\ 
  Relative MIAE & 0.054683 & 0.070397 & 0.070338 \\ 
  Max Error & 0.011293 & 0.016460 & 0.016441 \\ 
  Peak bias & -0.005739 & 0.003331 & 0.003308 \\ 
  Relative Peak bias & -0.137007 & 0.079526 & 0.078980 \\ 
  Peak drift & 0.407527 & 0.576641 & 0.576732 \\ 
  Relative Peak drift & 0.058218 & 0.082377 & 0.082390 \\ 
  Centroid bias & -0.005876 & 0.001197 & 0.001182 \\ 
  Relative Centroid bias & -0.140265 & 0.028578 & 0.028217 \\ 
  Centroid drift & 0.337723 & 0.384568 & 0.383883 \\ 
  Relative Centroid drift & 0.048246 & 0.054938 & 0.054840 \\ 
   \hline
\end{tabular}

    \caption[]{Means} 
    \end{subtable}%
    \begin{subtable}{0.5\textwidth}
    % latex table generated in R 3.4.2 by xtable 1.8-2 package
% Thu Dec  7 17:09:22 2017
\begin{tabular}{lrrr}
  \hline
 & Oracle & Silverman & CV \\ 
  \hline
MISE & 0.000006 & 0.000007 & 0.000007 \\ 
  Relative MISE & 0.003137 & 0.003786 & 0.003783 \\ 
  MIAE & 0.000459 & 0.000437 & 0.000437 \\ 
  Relative MIAE & 0.010946 & 0.010433 & 0.010433 \\ 
  Max Error & 0.003239 & 0.004242 & 0.004240 \\ 
  Peak bias & 0.005135 & 0.008388 & 0.008383 \\ 
  Relative Peak bias & 0.122587 & 0.200238 & 0.200114 \\ 
  Peak drift & 0.224036 & 0.304266 & 0.304044 \\ 
  Relative Peak drift & 0.032005 & 0.043467 & 0.043435 \\ 
  Centroid bias & 0.005205 & 0.009014 & 0.009006 \\ 
  Relative Centroid bias & 0.124269 & 0.215189 & 0.215006 \\ 
  Centroid drift & 0.178480 & 0.198033 & 0.197719 \\ 
  Relative Centroid drift & 0.025497 & 0.028290 & 0.028246 \\ 
   \hline
\end{tabular}

    \caption[]{Standard deviations} 
    \end{subtable}

\caption[]{Error rates for uniform population of 10,000, single peak intensity of factor 100 and decay rate 1.4}
\label{tbl:mean_error_rates:unif_100_1.4_1h}
\end{table}

\subsection{100 cases from 10,000 with intensity decay rate 1.0}

See \autoref{tbl:mean_error_rates:unif_100_1.0_1h}
\begin{table}[H]
\centering
\scriptsize

    \begin{subtable}{0.5\textwidth}
    % latex table generated in R 3.3.3 by xtable 1.8-2 package
% Mon Jan 15 21:47:29 2018
\begin{tabular}{lrrr}
  \hline
 & Oracle & Silverman & CV \\ 
  \hline
MISE & 0.000022 & 0.000035 & 0.000033 \\ 
  Relative MISE & 0.003402 & 0.005404 & 0.004996 \\ 
  Normalized MISE & 0.000022 & 0.000035 & 0.000033 \\ 
  MIAE & 0.002515 & 0.003126 & 0.003003 \\ 
  Relative MIAE & 0.031137 & 0.038695 & 0.037178 \\ 
  Max Error & 0.021087 & 0.031111 & 0.029319 \\ 
  Peak bias & -0.010671 & 0.006739 & 0.004427 \\ 
  Relative Peak bias & -0.132093 & 0.083426 & 0.054803 \\ 
  Peak drift & 0.288634 & 0.418493 & 0.401299 \\ 
  Relative Peak drift & 0.041233 & 0.059785 & 0.057328 \\ 
  Centroid bias & -0.011216 & -0.000155 & -0.001331 \\ 
  Relative Centroid bias & -0.138849 & -0.001917 & -0.016472 \\ 
  Centroid drift & 0.207657 & 0.222708 & 0.221023 \\ 
  Relative Centroid drift & 0.029665 & 0.031815 & 0.031575 \\ 
   \hline
\end{tabular}

    \caption[]{Means} 
    \end{subtable}%
    \begin{subtable}{0.5\textwidth}
    % latex table generated in R 3.3.3 by xtable 1.8-2 package
% Sun Mar 11 18:24:22 2018
\begin{tabular}{lrrr}
  \hline
 & Oracle & Silverman & CV \\ 
  \hline
MISE & 0.000011 & 0.000012 & 0.000021 \\ 
  Relative MISE & 0.001700 & 0.001896 & 0.003145 \\ 
  Normalized MISE & 0.000000 & 0.000000 & 0.000000 \\ 
  MIAE & 0.000526 & 0.000447 & 0.000719 \\ 
  Relative MIAE & 0.006506 & 0.005534 & 0.008895 \\ 
  Normalized MIAE & 0.000000 & 0.000000 & 0.000000 \\ 
  Max Error & 0.006576 & 0.008019 & 0.013486 \\ 
  Normalized Max Error & 0.000001 & 0.000001 & 0.000001 \\ 
  Peak bias & 0.009684 & 0.015381 & 0.020691 \\ 
  Relative Peak bias & 0.119873 & 0.190401 & 0.256135 \\ 
  Peak drift & 0.154869 & 0.218750 & 0.200306 \\ 
  Relative Peak drift & 0.022124 & 0.031250 & 0.028615 \\ 
  Centroid bias & 0.009891 & 0.015892 & 0.016018 \\ 
  Relative Centroid bias & 0.122436 & 0.196722 & 0.198285 \\ 
  Centroid drift & 0.114681 & 0.121309 & 0.117012 \\ 
  Relative Centroid drift & 0.016383 & 0.017330 & 0.016716 \\ 
   \hline
\end{tabular}

    \caption[]{Standard deviations} 
    \end{subtable}

\caption[]{Error rates for uniform population of 10,000, single peak intensity of factor 100 and decay rate 1.0}
\label{tbl:mean_error_rates:unif_100_1.0_1h:3}
\end{table}

\subsection{100 cases from 10,000 with intensity decay rate 0.7}
\begin{table}[H]
\centering
\scriptsize

    \begin{subtable}{0.5\textwidth}
    % latex table generated in R 3.4.2 by xtable 1.8-2 package
% Thu Dec  7 16:57:46 2017
\begin{tabular}{lrrr}
  \hline
 & Oracle & Silverman & CV \\ 
  \hline
MISE & 0.000042 & 0.000070 & 0.000067 \\ 
  Relative MISE & 0.001528 & 0.002568 & 0.002483 \\ 
  MIAE & 0.002425 & 0.003070 & 0.003019 \\ 
  Relative MIAE & 0.014713 & 0.018625 & 0.018318 \\ 
  Max Error & 0.041189 & 0.062268 & 0.060801 \\ 
  Peak bias & -0.020965 & 0.013460 & 0.011707 \\ 
  Relative Peak bias & -0.127194 & 0.081663 & 0.071024 \\ 
  Peak drift & 0.187770 & 0.306194 & 0.300844 \\ 
  Relative Peak drift & 0.026824 & 0.043742 & 0.042978 \\ 
  Centroid bias & -0.023222 & -0.008659 & -0.009076 \\ 
  Relative Centroid bias & -0.140888 & -0.052531 & -0.055064 \\ 
  Centroid drift & 0.114011 & 0.119838 & 0.119153 \\ 
  Relative Centroid drift & 0.016287 & 0.017120 & 0.017022 \\ 
   \hline
\end{tabular}

    \caption[]{Means} 
    \end{subtable}%
    \begin{subtable}{0.5\textwidth}
    % latex table generated in R 3.4.2 by xtable 1.8-2 package
% Sat Feb 17 16:38:14 2018
\begin{tabular}{lrrr}
  \hline
 & Oracle & Silverman & CV \\ 
  \hline
MISE & 0.000020 & 0.000027 & 0.000039 \\ 
  Relative MISE & 0.000739 & 0.000989 & 0.001432 \\ 
  Normalized MISE & 0.000020 & 0.000027 & 0.000039 \\ 
  MIAE & 0.000484 & 0.000443 & 0.000714 \\ 
  Relative MIAE & 0.002936 & 0.002687 & 0.004331 \\ 
  Max Error & 0.012449 & 0.016956 & 0.024634 \\ 
  Peak bias & 0.019842 & 0.030937 & 0.038487 \\ 
  Relative Peak bias & 0.120383 & 0.187693 & 0.233502 \\ 
  Peak drift & 0.105135 & 0.157797 & 0.148019 \\ 
  Relative Peak drift & 0.015019 & 0.022542 & 0.021146 \\ 
  Centroid bias & 0.020297 & 0.031820 & 0.031438 \\ 
  Relative Centroid bias & 0.123139 & 0.193052 & 0.190736 \\ 
  Centroid drift & 0.067499 & 0.069453 & 0.068564 \\ 
  Relative Centroid drift & 0.009643 & 0.009922 & 0.009795 \\ 
   \hline
\end{tabular}

    \caption[]{Standard deviations} 
    \end{subtable}

\caption[]{Error rates for uniform population of 10,000, single peak intensity of factor 100 and decay rate 0.7}
\label{tbl:mean_error_rates:unif_100_0.7_1h}
\end{table}

\subsection{100 cases from 10,000 with intensity decay rate 0.5}
\begin{table}[H]
\centering
\scriptsize

    \begin{subtable}{0.5\textwidth}
    % latex table generated in R 3.4.0 by xtable 1.8-2 package
% Sat Aug  5 20:27:22 2017
\begin{table}[ht]
\centering
\begin{tabular}{rrrr}
  \hline
 & Oracle & Silverman & CV \\ 
  \hline
MISE & 0.000077 & 0.000131 & 0.000119 \\ 
  Relative MISE & 0.000734 & 0.001254 & 0.001144 \\ 
  MIAE & 0.002409 & 0.003038 & 0.002876 \\ 
  Relative MIAE & 0.007454 & 0.009399 & 0.008899 \\ 
  Max Error & 0.076835 & 0.117578 & 0.108564 \\ 
  Peak bias & -0.046725 & 0.020911 & 0.009059 \\ 
  Relative Peak bias & -0.144576 & 0.064704 & 0.028031 \\ 
  Peak drift & 0.129820 & 0.216138 & 0.203513 \\ 
  Relative Peak drift & 0.018546 & 0.030877 & 0.029073 \\ 
  Centroid bias & -0.054373 & -0.022933 & -0.029030 \\ 
  Relative Centroid bias & -0.168240 & -0.070958 & -0.089824 \\ 
  Centroid drift & 0.064940 & 0.071866 & 0.070546 \\ 
  Relative Centroid drift & 0.009277 & 0.010267 & 0.010078 \\ 
   \hline
\end{tabular}
\caption{Mean error rates} 
\label{tbl:mean_error_rates}
\end{table}

    \caption[]{Means} 
    \end{subtable}%
    \begin{subtable}{0.5\textwidth}
    % latex table generated in R 3.4.0 by xtable 1.8-2 package
% Sat Aug  5 20:27:22 2017
\begin{tabular}{lrrr}
  \hline
 & Oracle & Silverman & CV \\ 
  \hline
MISE & 0.000036 & 0.000046 & 0.000063 \\ 
  Relative MISE & 0.000340 & 0.000438 & 0.000607 \\ 
  MIAE & 0.000466 & 0.000425 & 0.000563 \\ 
  Relative MIAE & 0.001442 & 0.001314 & 0.001741 \\ 
  Max Error & 0.023200 & 0.030300 & 0.039341 \\ 
  Peak bias & 0.032038 & 0.050895 & 0.055081 \\ 
  Relative Peak bias & 0.099132 & 0.157480 & 0.170430 \\ 
  Peak drift & 0.080767 & 0.118526 & 0.114042 \\ 
  Relative Peak drift & 0.011538 & 0.016932 & 0.016292 \\ 
  Centroid bias & 0.033350 & 0.057284 & 0.053665 \\ 
  Relative Centroid bias & 0.103192 & 0.177250 & 0.166049 \\ 
  Centroid drift & 0.056974 & 0.058126 & 0.057936 \\ 
  Relative Centroid drift & 0.008139 & 0.008304 & 0.008277 \\ 
   \hline
\end{tabular}

    \caption[]{Standard deviations} 
    \end{subtable}

\caption[]{Error rates for uniform population of 10,000, single peak intensity of factor 100 and decay rate 0.5}
\label{tbl:mean_error_rates:unif_100_0.5_1h}
\end{table}


%%
%% Section
\section{Varying the decay of the population density}

\subsection{100 cases from 10,000 with no population decay (uniform)}

See \autoref{tbl:mean_error_rates:unif_100_unif}.

\begin{table}[H]
\centering
\scriptsize

    \begin{subtable}{0.5\textwidth}
    % latex table generated in R 3.3.3 by xtable 1.8-2 package
% Sun Mar 11 18:27:52 2018
\begin{tabular}{lrrr}
  \hline
 & Oracle & Silverman & CV \\ 
  \hline
MISE & 0.000012 & 0.000013 & 0.000013 \\ 
  Relative MISE & 0.116112 & 0.125979 & 0.128308 \\ 
  Normalized MISE & 0.000000 & 0.000000 & 0.000000 \\ 
  MIAE & 0.002786 & 0.002898 & 0.002938 \\ 
  Relative MIAE & 0.278620 & 0.289846 & 0.293819 \\ 
  Normalized MIAE & 0.000000 & 0.000000 & 0.000000 \\ 
  Max Error & 0.008551 & 0.009000 & 0.008926 \\ 
  Normalized Max Error & 0.000001 & 0.000001 & 0.000001 \\ 
   \hline
\end{tabular}

    \caption[]{Means} 
    \end{subtable}%
    \begin{subtable}{0.5\textwidth}
    % latex table generated in R 3.4.3 by xtable 1.8-2 package
% Tue Apr 03 12:03:44 2018
\begin{tabular}{lrrr}
  \hline
 & Oracle & Silverman & CV \\ 
  \hline
MISE & 0.000003 & 0.000003 & 0.000003 \\ 
  Relative MISE & 0.030194 & 0.032029 & 0.034292 \\ 
  Normalized MISE & 0.301940 & 0.320288 & 0.342922 \\ 
  MIAE & 0.000438 & 0.000418 & 0.000446 \\ 
  Relative MIAE & 0.043781 & 0.041769 & 0.044615 \\ 
  Normalized MIAE & 0.000004 & 0.000004 & 0.000004 \\ 
  Max Error & 0.000581 & 0.000870 & 0.001109 \\ 
  Normalized Max Error & 0.000006 & 0.000009 & 0.000011 \\ 
  Peak bias & 0.001757 & 0.002365 & 0.002891 \\ 
  Relative Peak bias & 0.175734 & 0.236474 & 0.289140 \\ 
  Peak drift & 1.467764 & 1.600954 & 1.558562 \\ 
  Relative Peak drift & 0.209681 & 0.228708 & 0.222652 \\ 
  Centroid bias & 0.002315 & 0.003475 & 0.003480 \\ 
  Relative Centroid bias & 0.231491 & 0.347454 & 0.347951 \\ 
  Centroid drift & 1.204587 & 1.197905 & 1.181971 \\ 
  Relative Centroid drift & 0.172084 & 0.171129 & 0.168853 \\ 
   \hline
\end{tabular}

    \caption[]{Standard deviations} 
    \end{subtable}

\caption[]{Error rates for uniform population of 10,000, single peak intensity of factor 100 and no population decay (uniform)}
\label{tbl:mean_error_rates:unif_100_unif:3}
\end{table}

% \subsection{50 cases from 10,000 with population decay rate 2.0}
% \begin{table}[H]
% \centering
% \scriptsize

%     \begin{subtable}{0.5\textwidth}
%     % latex table generated in R 3.3.3 by xtable 1.8-2 package
% Mon Jan 15 21:47:29 2018
\begin{tabular}{lrrr}
  \hline
 & Oracle & Silverman & CV \\ 
  \hline
MISE & 0.000022 & 0.000035 & 0.000033 \\ 
  Relative MISE & 0.003402 & 0.005404 & 0.004996 \\ 
  Normalized MISE & 0.000022 & 0.000035 & 0.000033 \\ 
  MIAE & 0.002515 & 0.003126 & 0.003003 \\ 
  Relative MIAE & 0.031137 & 0.038695 & 0.037178 \\ 
  Max Error & 0.021087 & 0.031111 & 0.029319 \\ 
  Peak bias & -0.010671 & 0.006739 & 0.004427 \\ 
  Relative Peak bias & -0.132093 & 0.083426 & 0.054803 \\ 
  Peak drift & 0.288634 & 0.418493 & 0.401299 \\ 
  Relative Peak drift & 0.041233 & 0.059785 & 0.057328 \\ 
  Centroid bias & -0.011216 & -0.000155 & -0.001331 \\ 
  Relative Centroid bias & -0.138849 & -0.001917 & -0.016472 \\ 
  Centroid drift & 0.207657 & 0.222708 & 0.221023 \\ 
  Relative Centroid drift & 0.029665 & 0.031815 & 0.031575 \\ 
   \hline
\end{tabular}

%     \caption[]{Means} 
%     \end{subtable}%
%     \begin{subtable}{0.5\textwidth}
%     % latex table generated in R 3.4.2 by xtable 1.8-2 package
% Thu Feb 15 19:57:40 2018
\begin{tabular}{lrrr}
  \hline
 & Oracle & Silverman & CV \\ 
  \hline
MISE & 0.000089 & 0.000116 & 0.000228 \\ 
  Relative MISE & 0.000131 & 0.000171 & 0.000335 \\ 
  Normalized MISE & 0.000018 & 0.000023 & 0.000046 \\ 
  MIAE & 0.000911 & 0.000820 & 0.001681 \\ 
  Relative MIAE & 0.001105 & 0.000994 & 0.002040 \\ 
  Max Error & 0.024919 & 0.036302 & 0.058950 \\ 
  Peak bias & 0.031372 & 0.045576 & 0.062499 \\ 
  Relative Peak bias & 0.038067 & 0.055302 & 0.075836 \\ 
  Peak drift & 0.060193 & 0.086240 & 0.111909 \\ 
  Relative Peak drift & 0.008599 & 0.012320 & 0.015987 \\ 
  Centroid bias & 0.031457 & 0.047196 & 0.046071 \\ 
  Relative Centroid bias & 0.038170 & 0.057267 & 0.055902 \\ 
  Centroid drift & 0.051758 & 0.054538 & 0.052729 \\ 
  Relative Centroid drift & 0.007394 & 0.007791 & 0.007533 \\ 
   \hline
\end{tabular}

%     \caption[]{Standard deviations} 
%     \end{subtable}

% \caption[]{Error rates for population of 10,000 with decay rate 2.0, single peak intensity of factor 50}
% \label{tbl:mean_error_rates:p2.0_50_1.0_1h}
% \end{table}

\subsection{100 cases from 10,000 with population decay rate 2.0}
\begin{table}[H]
\centering
\scriptsize

    \begin{subtable}{0.5\textwidth}
    % latex table generated in R 3.4.0 by xtable 1.8-2 package
% Sat Aug  5 21:07:06 2017
\begin{table}[H]
\centering
\begin{tabular}{lrrr}
  \hline
 & Oracle & Silverman & CV \\ 
  \hline
MISE & 0.000021 & 0.000034 & 0.000034 \\ 
  Relative MISE & 0.003273 & 0.005274 & 0.005179 \\ 
  MIAE & 0.002642 & 0.003221 & 0.003194 \\ 
  Relative MIAE & 0.032700 & 0.039874 & 0.039542 \\ 
  Max Error & 0.020044 & 0.030183 & 0.029789 \\ 
  Peak bias & -0.009647 & 0.006342 & 0.005880 \\ 
  Relative Peak bias & -0.119422 & 0.078502 & 0.072792 \\ 
  Peak drift & 0.309850 & 0.468400 & 0.464955 \\ 
  Relative Peak drift & 0.044264 & 0.066914 & 0.066422 \\ 
  Centroid bias & -0.010650 & -0.004896 & -0.004979 \\ 
  Relative Centroid bias & -0.131838 & -0.060607 & -0.061636 \\ 
  Centroid drift & 0.192663 & 0.192580 & 0.192641 \\ 
  Relative Centroid drift & 0.027523 & 0.027511 & 0.027520 \\ 
   \hline
\end{tabular}
\caption{Mean error rates} 
\label{tbl:mean_error_rates}
\end{table}

    \caption[]{Means} 
    \end{subtable}%
    \begin{subtable}{0.5\textwidth}
    % latex table generated in R 3.4.2 by xtable 1.8-2 package
% Sun Nov  5 16:38:08 2017
\begin{tabular}{lrrr}
  \hline
 & Oracle & Silverman & CV \\ 
  \hline
MISE & 0.000008 & 0.000009 & 0.000009 \\ 
  Relative MISE & 0.001252 & 0.001437 & 0.001432 \\ 
  MIAE & 0.000465 & 0.000378 & 0.000382 \\ 
  Relative MIAE & 0.005755 & 0.004684 & 0.004725 \\ 
  Max Error & 0.005228 & 0.006768 & 0.006750 \\ 
  Peak bias & 0.007882 & 0.011659 & 0.011571 \\ 
  Relative Peak bias & 0.097568 & 0.144324 & 0.143244 \\ 
  Peak drift & 0.172228 & 0.204336 & 0.204325 \\ 
  Relative Peak drift & 0.024604 & 0.029191 & 0.029189 \\ 
  Centroid bias & 0.008202 & 0.012807 & 0.012658 \\ 
  Relative Centroid bias & 0.101533 & 0.158534 & 0.156701 \\ 
  Centroid drift & 0.106264 & 0.105423 & 0.105643 \\ 
  Relative Centroid drift & 0.015181 & 0.015060 & 0.015092 \\ 
   \hline
\end{tabular}

    \caption[]{Standard deviations} 
    \end{subtable}

\caption[]{Error rates for population of 10,000 with decay rate 2.0, single peak intensity of factor 100}
\label{tbl:mean_error_rates:p2.0_100_1.0_1h}
\end{table}

\subsection{100 cases from 10,000 with population decay rate 1.4}
\begin{table}[H]
\centering
\scriptsize

    \begin{subtable}{0.5\textwidth}
    % latex table generated in R 3.4.0 by xtable 1.8-2 package
% Sat Aug  5 21:37:13 2017
\begin{table}[ht]
\centering
\begin{tabular}{rrrr}
  \hline
 & Oracle & Silverman & CV \\ 
  \hline
MISE & 0.000023 & 0.000036 & 0.000034 \\ 
  Relative MISE & 0.003558 & 0.005476 & 0.005188 \\ 
  MIAE & 0.002836 & 0.003304 & 0.003227 \\ 
  Relative MIAE & 0.035109 & 0.040906 & 0.039946 \\ 
  Max Error & 0.019837 & 0.031477 & 0.030030 \\ 
  Peak bias & -0.008783 & 0.006794 & 0.005493 \\ 
  Relative Peak bias & -0.108731 & 0.084104 & 0.067997 \\ 
  Peak drift & 0.279824 & 0.422472 & 0.412451 \\ 
  Relative Peak drift & 0.039975 & 0.060353 & 0.058922 \\ 
  Centroid bias & -0.009989 & -0.004347 & -0.004578 \\ 
  Relative Centroid bias & -0.123654 & -0.053816 & -0.056672 \\ 
  Centroid drift & 0.154247 & 0.149632 & 0.149878 \\ 
  Relative Centroid drift & 0.022035 & 0.021376 & 0.021411 \\ 
   \hline
\end{tabular}
\caption{Mean error rates} 
\label{tbl:mean_error_rates}
\end{table}

    \caption[]{Means} 
    \end{subtable}%
    \begin{subtable}{0.5\textwidth}
    % latex table generated in R 3.4.0 by xtable 1.8-2 package
% Sat Aug  5 21:37:14 2017
\begin{tabular}{lrrr}
  \hline
 & Oracle & Silverman & CV \\ 
  \hline
MISE & 0.000007 & 0.000008 & 0.000008 \\ 
  Relative MISE & 0.001067 & 0.001212 & 0.001249 \\ 
  MIAE & 0.000460 & 0.000347 & 0.000369 \\ 
  Relative MIAE & 0.005697 & 0.004301 & 0.004568 \\ 
  Max Error & 0.003761 & 0.005748 & 0.005796 \\ 
  Peak bias & 0.006662 & 0.009718 & 0.009774 \\ 
  Relative Peak bias & 0.082474 & 0.120295 & 0.120997 \\ 
  Peak drift & 0.152583 & 0.187811 & 0.187107 \\ 
  Relative Peak drift & 0.021798 & 0.026830 & 0.026730 \\ 
  Centroid bias & 0.006939 & 0.011113 & 0.010738 \\ 
  Relative Centroid bias & 0.085896 & 0.137574 & 0.132922 \\ 
  Centroid drift & 0.085880 & 0.085836 & 0.085521 \\ 
  Relative Centroid drift & 0.012269 & 0.012262 & 0.012217 \\ 
   \hline
\end{tabular}

    \caption[]{Standard deviations} 
    \end{subtable}

\caption[]{Error rates for population of 10,000 with decay rate 1.4, single peak intensity of factor 100}
\label{tbl:mean_error_rates:p1.4_100_1.0_1h}
\end{table}

\subsection{100 cases from 10,000 with population decay rate 1.0}
\begin{table}[H]
\centering
\scriptsize

    \begin{subtable}{0.5\textwidth}
    % latex table generated in R 3.4.0 by xtable 1.8-2 package
% Sat Aug  5 23:00:37 2017
\begin{table}[ht]
\centering
\begin{tabular}{rrrr}
  \hline
 & Oracle & Silverman & CV \\ 
  \hline
MISE & 0.000066 & 0.000087 & 0.000078 \\ 
  Relative MISE & 0.010123 & 0.013299 & 0.011878 \\ 
  MIAE & 0.003812 & 0.004124 & 0.003938 \\ 
  Relative MIAE & 0.047184 & 0.051056 & 0.048743 \\ 
  Max Error & 0.054969 & 0.086132 & 0.074745 \\ 
  Peak bias & 0.002495 & 0.028003 & 0.019537 \\ 
  Relative Peak bias & 0.030881 & 0.346652 & 0.241856 \\ 
  Peak drift & 0.875427 & 1.234994 & 1.090434 \\ 
  Relative Peak drift & 0.125061 & 0.176428 & 0.155776 \\ 
  Centroid bias & -0.011069 & -0.005650 & -0.006984 \\ 
  Relative Centroid bias & -0.137022 & -0.069944 & -0.086449 \\ 
  Centroid drift & 0.238070 & 0.225614 & 0.227857 \\ 
  Relative Centroid drift & 0.034010 & 0.032231 & 0.032551 \\ 
   \hline
\end{tabular}
\caption{Mean error rates} 
\label{tbl:mean_error_rates}
\end{table}

    \caption[]{Means} 
    \end{subtable}%
    \begin{subtable}{0.5\textwidth}
    % latex table generated in R 3.4.0 by xtable 1.8-2 package
% Sat Aug  5 23:00:38 2017
\begin{table}[H]
\centering
\begin{tabular}{lrrr}
  \hline
 & Oracle & Silverman & CV \\ 
  \hline
MISE & 0.000057 & 0.000070 & 0.000066 \\ 
  Relative MISE & 0.008716 & 0.010683 & 0.010096 \\ 
  MIAE & 0.000835 & 0.000624 & 0.000695 \\ 
  Relative MIAE & 0.010332 & 0.007724 & 0.008599 \\ 
  Max Error & 0.039895 & 0.059402 & 0.054207 \\ 
  Peak bias & 0.028142 & 0.048165 & 0.042352 \\ 
  Relative Peak bias & 0.348370 & 0.596241 & 0.524281 \\ 
  Peak drift & 1.182271 & 1.214951 & 1.199142 \\ 
  Relative Peak drift & 0.168896 & 0.173564 & 0.171306 \\ 
  Centroid bias & 0.007869 & 0.010140 & 0.009436 \\ 
  Relative Centroid bias & 0.097406 & 0.125528 & 0.116811 \\ 
  Centroid drift & 0.218588 & 0.158346 & 0.171699 \\ 
  Relative Centroid drift & 0.031227 & 0.022621 & 0.024528 \\ 
   \hline
\end{tabular}
\caption{Standard deviation of error rates} 
\label{tbl:stddev_error_rates}
\end{table}

    \caption[]{Standard deviations} 
    \end{subtable}

\caption[]{Error rates for population of 10,000 with decay rate 1.0, single peak intensity of factor 100}
\label{tbl:mean_error_rates:p1.0_100_1.0_1h}
\end{table}

\subsection{100 cases from 10,000 with population decay rate 0.7}
\begin{table}[H]
\centering
\scriptsize

    \begin{subtable}{0.5\textwidth}
    % latex table generated in R 3.3.3 by xtable 1.8-2 package
% Mon Jan 15 21:47:29 2018
\begin{tabular}{lrrr}
  \hline
 & Oracle & Silverman & CV \\ 
  \hline
MISE & 0.000022 & 0.000035 & 0.000033 \\ 
  Relative MISE & 0.003402 & 0.005404 & 0.004996 \\ 
  Normalized MISE & 0.000022 & 0.000035 & 0.000033 \\ 
  MIAE & 0.002515 & 0.003126 & 0.003003 \\ 
  Relative MIAE & 0.031137 & 0.038695 & 0.037178 \\ 
  Max Error & 0.021087 & 0.031111 & 0.029319 \\ 
  Peak bias & -0.010671 & 0.006739 & 0.004427 \\ 
  Relative Peak bias & -0.132093 & 0.083426 & 0.054803 \\ 
  Peak drift & 0.288634 & 0.418493 & 0.401299 \\ 
  Relative Peak drift & 0.041233 & 0.059785 & 0.057328 \\ 
  Centroid bias & -0.011216 & -0.000155 & -0.001331 \\ 
  Relative Centroid bias & -0.138849 & -0.001917 & -0.016472 \\ 
  Centroid drift & 0.207657 & 0.222708 & 0.221023 \\ 
  Relative Centroid drift & 0.029665 & 0.031815 & 0.031575 \\ 
   \hline
\end{tabular}

    \caption[]{Means} 
    \end{subtable}
    \begin{subtable}{0.5\textwidth}
    % latex table generated in R 3.4.2 by xtable 1.8-2 package
% Thu Feb 15 19:57:40 2018
\begin{tabular}{lrrr}
  \hline
 & Oracle & Silverman & CV \\ 
  \hline
MISE & 0.000089 & 0.000116 & 0.000228 \\ 
  Relative MISE & 0.000131 & 0.000171 & 0.000335 \\ 
  Normalized MISE & 0.000018 & 0.000023 & 0.000046 \\ 
  MIAE & 0.000911 & 0.000820 & 0.001681 \\ 
  Relative MIAE & 0.001105 & 0.000994 & 0.002040 \\ 
  Max Error & 0.024919 & 0.036302 & 0.058950 \\ 
  Peak bias & 0.031372 & 0.045576 & 0.062499 \\ 
  Relative Peak bias & 0.038067 & 0.055302 & 0.075836 \\ 
  Peak drift & 0.060193 & 0.086240 & 0.111909 \\ 
  Relative Peak drift & 0.008599 & 0.012320 & 0.015987 \\ 
  Centroid bias & 0.031457 & 0.047196 & 0.046071 \\ 
  Relative Centroid bias & 0.038170 & 0.057267 & 0.055902 \\ 
  Centroid drift & 0.051758 & 0.054538 & 0.052729 \\ 
  Relative Centroid drift & 0.007394 & 0.007791 & 0.007533 \\ 
   \hline
\end{tabular}

    \caption[]{Standard deviations} 
    \end{subtable}

\caption[]{Error rates for population of 10,000 with decay rate 0.7, single peak intensity of factor 100}
\label{tbl:mean_error_rates:p0.7_100_1.0_1h}
\end{table}


%%
%% Section
\section{Varying the distance between two peaks}

\subsection{100 cases from uniform population of 10,000, 2 hills, 60/40 height ratio, gap of 1}
\begin{table}[H]
\centering
\scriptsize

    \begin{subtable}{0.5\textwidth}
    % latex table generated in R 3.3.3 by xtable 1.8-2 package
% Sun Mar 11 18:25:14 2018
\begin{tabular}{lrrr}
  \hline
 & Oracle & Silverman & CV \\ 
  \hline
MISE & 0.000020 & 0.000028 & 0.000028 \\ 
  Relative MISE & 0.003806 & 0.005434 & 0.005499 \\ 
  Normalized MISE & 0.000000 & 0.000000 & 0.000000 \\ 
  MIAE & 0.002482 & 0.002946 & 0.002911 \\ 
  Relative MIAE & 0.034598 & 0.041069 & 0.040574 \\ 
  Normalized MIAE & 0.000000 & 0.000000 & 0.000000 \\ 
  Max Error & 0.018613 & 0.025324 & 0.024450 \\ 
  Normalized Max Error & 0.000002 & 0.000003 & 0.000002 \\ 
  Peak bias & -0.008775 & 0.002958 & -0.002987 \\ 
  Relative Peak bias & -0.122312 & 0.041234 & -0.041636 \\ 
  Peak drift & 0.308419 & 0.424717 & 0.350763 \\ 
  Relative Peak drift & 0.044060 & 0.060674 & 0.050109 \\ 
  Centroid bias & -0.009281 & -0.001295 & -0.007951 \\ 
  Relative Centroid bias & -0.129358 & -0.018055 & -0.110820 \\ 
  Centroid drift & 0.225096 & 0.241028 & 0.222905 \\ 
  Relative Centroid drift & 0.032157 & 0.034433 & 0.031844 \\ 
   \hline
\end{tabular}

    \caption[]{Means} 
    \end{subtable}%
    \begin{subtable}{0.5\textwidth}
    % latex table generated in R 3.4.2 by xtable 1.8-2 package
% Sat Feb 17 16:39:12 2018
\begin{tabular}{lrrr}
  \hline
 & Oracle & Silverman & CV \\ 
  \hline
MISE & 0.000010 & 0.000010 & 0.000016 \\ 
  Relative MISE & 0.001945 & 0.002011 & 0.003094 \\ 
  Normalized MISE & 0.000010 & 0.000010 & 0.000016 \\ 
  MIAE & 0.000504 & 0.000439 & 0.000652 \\ 
  Relative MIAE & 0.007032 & 0.006119 & 0.009082 \\ 
  Max Error & 0.005700 & 0.006580 & 0.010260 \\ 
  Peak bias & 0.008955 & 0.013103 & 0.016903 \\ 
  Relative Peak bias & 0.124825 & 0.182643 & 0.235609 \\ 
  Peak drift & 0.166058 & 0.224066 & 0.206658 \\ 
  Relative Peak drift & 0.023723 & 0.032009 & 0.029523 \\ 
  Centroid bias & 0.009182 & 0.013554 & 0.013838 \\ 
  Relative Centroid bias & 0.127981 & 0.188920 & 0.192878 \\ 
  Centroid drift & 0.119650 & 0.127521 & 0.120946 \\ 
  Relative Centroid drift & 0.017093 & 0.018217 & 0.017278 \\ 
   \hline
\end{tabular}

    \caption[]{Standard deviations} 
    \end{subtable}

\caption[]{Error rates for uniform population of 10,000, factor of 100 with 2 hills, 60/40 height ratio, gap of 1}
\label{tbl:mean_error_rates:unif_100_1_2h_1}
\end{table}

\subsection{100 cases from uniform population of 10,000, 2 hills, 60/40 height ratio, gap of 2}
\begin{table}[H]
\centering
\scriptsize

    \begin{subtable}{0.5\textwidth}
    % latex table generated in R 3.4.0 by xtable 1.8-2 package
% Sat Aug  5 21:30:32 2017
\begin{table}[ht]
\centering
\begin{tabular}{rrrr}
  \hline
 & Oracle & Silverman & CV \\ 
  \hline
MISE & 0.000016 & 0.000020 & 0.000020 \\ 
  Relative MISE & 0.005500 & 0.006847 & 0.006843 \\ 
  MIAE & 0.002450 & 0.002728 & 0.002727 \\ 
  Relative MIAE & 0.045229 & 0.050364 & 0.050348 \\ 
  Max Error & 0.015217 & 0.018315 & 0.018306 \\ 
  Peak bias & -0.007724 & -0.000374 & -0.000390 \\ 
  Relative Peak bias & -0.142615 & -0.006911 & -0.007195 \\ 
  Peak drift & 0.446284 & 0.521588 & 0.521333 \\ 
  Relative Peak drift & 0.063755 & 0.074513 & 0.074476 \\ 
  Centroid bias & -0.007922 & -0.002078 & -0.002088 \\ 
  Relative Centroid bias & -0.146276 & -0.038374 & -0.038550 \\ 
  Centroid drift & 0.425758 & 0.425942 & 0.425824 \\ 
  Relative Centroid drift & 0.060823 & 0.060849 & 0.060832 \\ 
   \hline
\end{tabular}
\caption{Mean error rates} 
\label{tbl:mean_error_rates}
\end{table}

    \caption[]{Means} 
    \end{subtable}%
    \begin{subtable}{0.5\textwidth}
    % latex table generated in R 3.3.3 by xtable 1.8-2 package
% Sun Mar 11 18:25:42 2018
\begin{tabular}{lrrr}
  \hline
 & Oracle & Silverman & CV \\ 
  \hline
MISE & 0.000008 & 0.000008 & 0.000011 \\ 
  Relative MISE & 0.002755 & 0.002781 & 0.003674 \\ 
  Normalized MISE & 0.000000 & 0.000000 & 0.000000 \\ 
  MIAE & 0.000517 & 0.000473 & 0.000595 \\ 
  Relative MIAE & 0.009538 & 0.008725 & 0.010990 \\ 
  Normalized MIAE & 0.000000 & 0.000000 & 0.000000 \\ 
  Max Error & 0.004458 & 0.004702 & 0.006811 \\ 
  Normalized Max Error & 0.000000 & 0.000000 & 0.000001 \\ 
  Peak bias & 0.006969 & 0.009963 & 0.012617 \\ 
  Relative Peak bias & 0.128684 & 0.183949 & 0.232964 \\ 
  Peak drift & 0.264007 & 0.330445 & 0.305978 \\ 
  Relative Peak drift & 0.037715 & 0.047206 & 0.043711 \\ 
  Centroid bias & 0.007077 & 0.010411 & 0.011572 \\ 
  Relative Centroid bias & 0.130667 & 0.192233 & 0.213671 \\ 
  Centroid drift & 0.228455 & 0.246370 & 0.236918 \\ 
  Relative Centroid drift & 0.032636 & 0.035196 & 0.033845 \\ 
   \hline
\end{tabular}

    \caption[]{Standard deviations} 
    \end{subtable}

\caption[]{Error rates for uniform population of 10,000, factor of 100 with 2 hills, 60/40 height ratio, gap of 2}
\label{tbl:mean_error_rates:unif_100_1_2h_2}
\end{table}

\subsection{100 cases from uniform population of 10,000, 2 hills, 60/40 height ratio, gap of 3}
\begin{table}[H]
\centering
\scriptsize

    \begin{subtable}{0.5\textwidth}
    % latex table generated in R 3.4.0 by xtable 1.8-2 package
% Sat Aug  5 21:31:15 2017
\begin{table}[ht]
\centering
\begin{tabular}{lrrr}
  \hline
 & Oracle & Silverman & CV \\ 
  \hline
MISE & 0.000018 & 0.000019 & 0.000019 \\ 
  Relative MISE & 0.007368 & 0.007591 & 0.007591 \\ 
  MIAE & 0.002675 & 0.002735 & 0.002735 \\ 
  Relative MIAE & 0.053776 & 0.054989 & 0.054991 \\ 
  Max Error & 0.016647 & 0.016798 & 0.016799 \\ 
  Peak bias & -0.010884 & -0.010384 & -0.010381 \\ 
  Relative Peak bias & -0.218843 & -0.208792 & -0.208735 \\ 
  Peak drift & 0.550466 & 0.541528 & 0.541476 \\ 
  Relative Peak drift & 0.078638 & 0.077361 & 0.077354 \\ 
  Centroid bias & -0.011999 & -0.011748 & -0.011746 \\ 
  Relative Centroid bias & -0.241255 & -0.236209 & -0.236168 \\ 
  Centroid drift & 0.581652 & 0.562239 & 0.562221 \\ 
  Relative Centroid drift & 0.083093 & 0.080320 & 0.080317 \\ 
   \hline
\end{tabular}
\caption{Mean error rates} 
\label{tbl:mean_error_rates}
\end{table}

    \caption[]{Means} 
    \end{subtable}%
    \begin{subtable}{0.5\textwidth}
    % latex table generated in R 3.4.0 by xtable 1.8-2 package
% Sat Aug  5 21:31:16 2017
\begin{table}[H]
\centering
\begin{tabular}{lrrr}
  \hline
 & Oracle & Silverman & CV \\ 
  \hline
MISE & 0.000007 & 0.000007 & 0.000007 \\ 
  Relative MISE & 0.002889 & 0.003008 & 0.003008 \\ 
  MIAE & 0.000453 & 0.000470 & 0.000469 \\ 
  Relative MIAE & 0.009104 & 0.009440 & 0.009440 \\ 
  Max Error & 0.004351 & 0.004420 & 0.004420 \\ 
  Peak bias & 0.005799 & 0.006642 & 0.006645 \\ 
  Relative Peak bias & 0.116597 & 0.133539 & 0.133611 \\ 
  Peak drift & 0.595014 & 0.587309 & 0.586974 \\ 
  Relative Peak drift & 0.085002 & 0.083901 & 0.083853 \\ 
  Centroid bias & 0.006730 & 0.007617 & 0.007620 \\ 
  Relative Centroid bias & 0.135322 & 0.153151 & 0.153222 \\ 
  Centroid drift & 0.458335 & 0.456328 & 0.455747 \\ 
  Relative Centroid drift & 0.065476 & 0.065190 & 0.065107 \\ 
   \hline
\end{tabular}
\caption{Standard deviation of error rates} 
\label{tbl:stddev_error_rates}
\end{table}

    \caption[]{Standard deviations} 
    \end{subtable}

\caption[]{Error rates for uniform population of 10,000, factor of 100 with 2 hills, 60/40 height ratio, gap of 3}
\label{tbl:mean_error_rates:unif_100_1_2h_3}
\end{table}
\subsection{100 cases from uniform population of 10,000, 2 hills, 60/40 height ratio, gap of 4}
\begin{table}[H]
\centering
\scriptsize

    \begin{subtable}{0.5\textwidth}
    % latex table generated in R 3.4.2 by xtable 1.8-2 package
% Sat Feb 17 16:40:52 2018
\begin{tabular}{lrrr}
  \hline
 & Oracle & Silverman & CV \\ 
  \hline
MISE & 0.000020 & 0.000024 & 0.000022 \\ 
  Relative MISE & 0.007604 & 0.009208 & 0.008191 \\ 
  Normalized MISE & 0.000020 & 0.000024 & 0.000022 \\ 
  MIAE & 0.002820 & 0.003099 & 0.002921 \\ 
  Relative MIAE & 0.054779 & 0.060198 & 0.056745 \\ 
  Max Error & 0.017580 & 0.019430 & 0.018238 \\ 
  Peak bias & -0.011073 & -0.017197 & -0.011963 \\ 
  Relative Peak bias & -0.215103 & -0.334060 & -0.232374 \\ 
  Peak drift & 0.641297 & 0.420580 & 0.595129 \\ 
  Relative Peak drift & 0.091614 & 0.060083 & 0.085018 \\ 
  Centroid bias & -0.015923 & -0.019127 & -0.016312 \\ 
  Relative Centroid bias & -0.309304 & -0.371538 & -0.316860 \\ 
  Centroid drift & 0.644231 & 0.502452 & 0.625179 \\ 
  Relative Centroid drift & 0.092033 & 0.071779 & 0.089311 \\ 
   \hline
\end{tabular}

    \caption[]{Means} 
    \end{subtable}%
    \begin{subtable}{0.5\textwidth}
    % latex table generated in R 3.4.0 by xtable 1.8-2 package
% Sat Aug  5 21:31:22 2017
\begin{tabular}{lrrr}
  \hline
 & Oracle & Silverman & CV \\ 
  \hline
MISE & 0.000008 & 0.000009 & 0.000009 \\ 
  Relative MISE & 0.003013 & 0.003440 & 0.003441 \\ 
  MIAE & 0.000502 & 0.000489 & 0.000490 \\ 
  Relative MIAE & 0.009756 & 0.009502 & 0.009511 \\ 
  Max Error & 0.004673 & 0.004611 & 0.004613 \\ 
  Peak bias & 0.006316 & 0.005595 & 0.005602 \\ 
  Relative Peak bias & 0.122696 & 0.108690 & 0.108816 \\ 
  Peak drift & 1.106241 & 0.913011 & 0.913039 \\ 
  Relative Peak drift & 0.158034 & 0.130430 & 0.130434 \\ 
  Centroid bias & 0.011014 & 0.007901 & 0.007910 \\ 
  Relative Centroid bias & 0.213943 & 0.153475 & 0.153658 \\ 
  Centroid drift & 0.688671 & 0.620291 & 0.619967 \\ 
  Relative Centroid drift & 0.098382 & 0.088613 & 0.088567 \\ 
   \hline
\end{tabular}

    \caption[]{Standard deviations} 
    \end{subtable}

\caption[]{Error rates for uniform population of 10,000, factor of 100 with 2 hills, 60/40 height ratio, gap of 4}
\label{tbl:mean_error_rates:unif_100_1_2h_4}
\end{table}

%%
%% Section
\section{Varying the distance between the population and risk function peaks}

\subsection{100 cases from 10,000 with population decay rate 1.4, distance of 0}

See \autoref{tbl:mean_error_rates:unif_200_1.0_1h}.

\begin{table}[H]
\centering
\scriptsize

    \begin{subtable}{0.5\textwidth}
    % latex table generated in R 3.4.0 by xtable 1.8-2 package
% Sat Aug  5 21:37:13 2017
\begin{table}[ht]
\centering
\begin{tabular}{rrrr}
  \hline
 & Oracle & Silverman & CV \\ 
  \hline
MISE & 0.000023 & 0.000036 & 0.000034 \\ 
  Relative MISE & 0.003558 & 0.005476 & 0.005188 \\ 
  MIAE & 0.002836 & 0.003304 & 0.003227 \\ 
  Relative MIAE & 0.035109 & 0.040906 & 0.039946 \\ 
  Max Error & 0.019837 & 0.031477 & 0.030030 \\ 
  Peak bias & -0.008783 & 0.006794 & 0.005493 \\ 
  Relative Peak bias & -0.108731 & 0.084104 & 0.067997 \\ 
  Peak drift & 0.279824 & 0.422472 & 0.412451 \\ 
  Relative Peak drift & 0.039975 & 0.060353 & 0.058922 \\ 
  Centroid bias & -0.009989 & -0.004347 & -0.004578 \\ 
  Relative Centroid bias & -0.123654 & -0.053816 & -0.056672 \\ 
  Centroid drift & 0.154247 & 0.149632 & 0.149878 \\ 
  Relative Centroid drift & 0.022035 & 0.021376 & 0.021411 \\ 
   \hline
\end{tabular}
\caption{Mean error rates} 
\label{tbl:mean_error_rates}
\end{table}

    \caption[]{Means} 
    \end{subtable}%
    \begin{subtable}{0.5\textwidth}
    % latex table generated in R 3.4.0 by xtable 1.8-2 package
% Sat Aug  5 21:37:14 2017
\begin{tabular}{lrrr}
  \hline
 & Oracle & Silverman & CV \\ 
  \hline
MISE & 0.000007 & 0.000008 & 0.000008 \\ 
  Relative MISE & 0.001067 & 0.001212 & 0.001249 \\ 
  MIAE & 0.000460 & 0.000347 & 0.000369 \\ 
  Relative MIAE & 0.005697 & 0.004301 & 0.004568 \\ 
  Max Error & 0.003761 & 0.005748 & 0.005796 \\ 
  Peak bias & 0.006662 & 0.009718 & 0.009774 \\ 
  Relative Peak bias & 0.082474 & 0.120295 & 0.120997 \\ 
  Peak drift & 0.152583 & 0.187811 & 0.187107 \\ 
  Relative Peak drift & 0.021798 & 0.026830 & 0.026730 \\ 
  Centroid bias & 0.006939 & 0.011113 & 0.010738 \\ 
  Relative Centroid bias & 0.085896 & 0.137574 & 0.132922 \\ 
  Centroid drift & 0.085880 & 0.085836 & 0.085521 \\ 
  Relative Centroid drift & 0.012269 & 0.012262 & 0.012217 \\ 
   \hline
\end{tabular}

    \caption[]{Standard deviations} 
    \end{subtable}

\caption[]{Error rates for uniform population of 10,000, single peak intensity of factor 100 and decay rate 1.4, distance between population peak and risk peak is 0}
\label{tbl:mean_error_rates:p1.4_100_1.0_1h:2}
\end{table}

\subsection{100 cases from 10,000 with population decay rate 1.4, distance of 1}
\begin{table}[H]
\centering
\scriptsize

    \begin{subtable}{0.5\textwidth}
    % latex table generated in R 3.4.2 by xtable 1.8-2 package
% Sat Feb 17 16:35:26 2018
\begin{tabular}{lrrr}
  \hline
 & Oracle & Silverman & CV \\ 
  \hline
MISE & 0.000040 & 0.000059 & 0.000053 \\ 
  Relative MISE & 0.006157 & 0.008980 & 0.008084 \\ 
  Normalized MISE & 0.000040 & 0.000059 & 0.000053 \\ 
  MIAE & 0.003333 & 0.003782 & 0.003692 \\ 
  Relative MIAE & 0.041247 & 0.046805 & 0.045691 \\ 
  Max Error & 0.032658 & 0.052441 & 0.042040 \\ 
  Peak bias & -0.010635 & 0.006640 & -0.002388 \\ 
  Relative Peak bias & -0.131630 & 0.082180 & -0.029551 \\ 
  Peak drift & 0.503816 & 0.753839 & 0.596590 \\ 
  Relative Peak drift & 0.071974 & 0.107691 & 0.085227 \\ 
  Centroid bias & -0.013864 & -0.009846 & -0.013099 \\ 
  Relative Centroid bias & -0.171596 & -0.121868 & -0.162127 \\ 
  Centroid drift & 0.228592 & 0.242398 & 0.247296 \\ 
  Relative Centroid drift & 0.032656 & 0.034628 & 0.035328 \\ 
   \hline
\end{tabular}

    \caption[]{Means} 
    \end{subtable}%
    \begin{subtable}{0.5\textwidth}
    % latex table generated in R 3.4.2 by xtable 1.8-2 package
% Sat Feb 17 16:35:26 2018
\begin{tabular}{lrrr}
  \hline
 & Oracle & Silverman & CV \\ 
  \hline
MISE & 0.000020 & 0.000028 & 0.000032 \\ 
  Relative MISE & 0.003104 & 0.004242 & 0.004895 \\ 
  Normalized MISE & 0.000020 & 0.000028 & 0.000032 \\ 
  MIAE & 0.000674 & 0.000529 & 0.000780 \\ 
  Relative MIAE & 0.008337 & 0.006543 & 0.009657 \\ 
  Max Error & 0.014262 & 0.025486 & 0.025950 \\ 
  Peak bias & 0.007732 & 0.016208 & 0.020017 \\ 
  Relative Peak bias & 0.095694 & 0.200609 & 0.247752 \\ 
  Peak drift & 0.591447 & 0.774400 & 0.669252 \\ 
  Relative Peak drift & 0.084492 & 0.110629 & 0.095607 \\ 
  Centroid bias & 0.007667 & 0.011869 & 0.010926 \\ 
  Relative Centroid bias & 0.094897 & 0.146900 & 0.135229 \\ 
  Centroid drift & 0.153294 & 0.160808 & 0.161309 \\ 
  Relative Centroid drift & 0.021899 & 0.022973 & 0.023044 \\ 
   \hline
\end{tabular}

    \caption[]{Standard deviations} 
    \end{subtable}

\caption[]{Error rates for uniform population of 10,000, single peak intensity of factor 100 and decay rate 1.4, distance between population peak and risk peak is 1}
\label{tbl:mean_error_rates:p1.4_100_1.0_1h_1s}
\end{table}

\subsection{100 cases from 10,000 with population decay rate 1.4, distance of 2}
\begin{table}[H]
\centering
\scriptsize

    \begin{subtable}{0.5\textwidth}
    % latex table generated in R 3.3.3 by xtable 1.8-2 package
% Sun Mar 11 17:58:57 2018
\begin{tabular}{lrrr}
  \hline
 & Oracle & Silverman & CV \\ 
  \hline
MISE & 0.000128 & 0.000186 & 0.000171 \\ 
  Relative MISE & 0.019452 & 0.028262 & 0.026077 \\ 
  Normalized MISE & 0.000000 & 0.000000 & 0.000000 \\ 
  MIAE & 0.004923 & 0.005142 & 0.005273 \\ 
  Relative MIAE & 0.060708 & 0.063408 & 0.065032 \\ 
  Normalized MIAE & 0.000000 & 0.000001 & 0.000001 \\ 
  Max Error & 0.064541 & 0.110936 & 0.089923 \\ 
  Normalized Max Error & 0.000006 & 0.000011 & 0.000009 \\ 
  Peak bias & 0.007356 & 0.054198 & 0.031429 \\ 
  Relative Peak bias & 0.090715 & 0.668364 & 0.387582 \\ 
  Peak drift & 1.231593 & 1.481287 & 1.435563 \\ 
  Relative Peak drift & 0.175942 & 0.211612 & 0.205080 \\ 
  Centroid bias & -0.015659 & -0.013082 & -0.013226 \\ 
  Relative Centroid bias & -0.193105 & -0.161322 & -0.163101 \\ 
  Centroid drift & 0.681121 & 0.587086 & 0.709678 \\ 
  Relative Centroid drift & 0.097303 & 0.083869 & 0.101383 \\ 
   \hline
\end{tabular}

    \caption[]{Means} 
    \end{subtable}%
    \begin{subtable}{0.5\textwidth}
    % latex table generated in R 3.3.3 by xtable 1.8-2 package
% Sun Mar 11 17:58:57 2018
\begin{tabular}{lrrr}
  \hline
 & Oracle & Silverman & CV \\ 
  \hline
MISE & 0.000149 & 0.000228 & 0.000220 \\ 
  Relative MISE & 0.022731 & 0.034747 & 0.033399 \\ 
  Normalized MISE & 0.000000 & 0.000000 & 0.000000 \\ 
  MIAE & 0.001460 & 0.001193 & 0.001518 \\ 
  Relative MIAE & 0.018006 & 0.014709 & 0.018719 \\ 
  Normalized MIAE & 0.000000 & 0.000000 & 0.000000 \\ 
  Max Error & 0.049284 & 0.089215 & 0.082627 \\ 
  Normalized Max Error & 0.000005 & 0.000009 & 0.000008 \\ 
  Peak bias & 0.040724 & 0.079859 & 0.075417 \\ 
  Relative Peak bias & 0.502201 & 0.984812 & 0.930040 \\ 
  Peak drift & 0.920127 & 0.795791 & 0.894325 \\ 
  Relative Peak drift & 0.131447 & 0.113684 & 0.127761 \\ 
  Centroid bias & 0.015710 & 0.021420 & 0.020204 \\ 
  Relative Centroid bias & 0.193734 & 0.264149 & 0.249158 \\ 
  Centroid drift & 0.425282 & 0.325892 & 0.403521 \\ 
  Relative Centroid drift & 0.060755 & 0.046556 & 0.057646 \\ 
   \hline
\end{tabular}

    \caption[]{Standard deviations} 
    \end{subtable}

\caption[]{Error rates for uniform population of 10,000, single peak intensity of factor 100 and decay rate 1.4, distance between population peak and risk peak is 2}
\label{tbl:mean_error_rates:p1.4_100_1.0_1h_2s}
\end{table}

\subsection{100 cases from 10,000 with population decay rate 1.4, distance of 3}
\begin{table}[H]
\centering
\scriptsize

    \begin{subtable}{0.5\textwidth}
    % latex table generated in R 3.4.0 by xtable 1.8-2 package
% Sun Aug 13 13:13:43 2017
\begin{tabular}{rrrr}
  \hline
 & Oracle & Silverman & CV \\ 
  \hline
MISE & 0.000451 & 0.000655 & 0.000653 \\ 
  Relative MISE & 0.066658 & 0.096783 & 0.096497 \\ 
  MIAE & 0.007659 & 0.007646 & 0.007637 \\ 
  Relative MIAE & 0.093078 & 0.092922 & 0.092817 \\ 
  Max Error & 0.120049 & 0.206498 & 0.205744 \\ 
  Peak bias & 0.059623 & 0.154237 & 0.153454 \\ 
  Relative Peak bias & 0.724613 & 1.874491 & 1.864966 \\ 
  Peak drift & 1.428088 & 1.365948 & 1.367441 \\ 
  Relative Peak drift & 0.204013 & 0.195135 & 0.195349 \\ 
  Centroid bias & 0.010077 & 0.006224 & 0.006442 \\ 
  Relative Centroid bias & 0.122465 & 0.075643 & 0.078287 \\ 
  Centroid drift & 0.907843 & 0.664232 & 0.666366 \\ 
  Relative Centroid drift & 0.129692 & 0.094890 & 0.095195 \\ 
   \hline
\end{tabular}

    \caption[]{Means} 
    \end{subtable}%
    \begin{subtable}{0.5\textwidth}
    % latex table generated in R 3.4.0 by xtable 1.8-2 package
% Sun Aug 13 13:13:43 2017
\begin{tabular}{rrrr}
  \hline
 & Oracle & Silverman & CV \\ 
  \hline
MISE & 0.000604 & 0.000854 & 0.000852 \\ 
  Relative MISE & 0.089183 & 0.126125 & 0.125830 \\ 
  MIAE & 0.003071 & 0.002655 & 0.002659 \\ 
  Relative MIAE & 0.037327 & 0.032271 & 0.032320 \\ 
  Max Error & 0.099582 & 0.171298 & 0.170786 \\ 
  Peak bias & 0.091706 & 0.159725 & 0.159222 \\ 
  Relative Peak bias & 1.114535 & 1.941180 & 1.935064 \\ 
  Peak drift & 0.854263 & 0.691748 & 0.691435 \\ 
  Relative Peak drift & 0.122038 & 0.098821 & 0.098776 \\ 
  Centroid bias & 0.049834 & 0.048907 & 0.049239 \\ 
  Relative Centroid bias & 0.605644 & 0.594377 & 0.598415 \\ 
  Centroid drift & 0.534955 & 0.378376 & 0.379461 \\ 
  Relative Centroid drift & 0.076422 & 0.054054 & 0.054209 \\ 
   \hline
\end{tabular}

    \caption[]{Standard deviations} 
    \end{subtable}

\caption[]{Error rates for uniform population of 10,000, single peak intensity of factor 100 and decay rate 1.4, distance between population peak and risk peak is 3}
\label{tbl:mean_error_rates:p1.4_100_1.0_1h_3s}
\end{table}

\subsection{100 cases from 10,000 with population decay rate 1.4, distance of 4}
\begin{table}[H]
\centering
\scriptsize

    \begin{subtable}{0.5\textwidth}
    % latex table generated in R 3.3.3 by xtable 1.8-2 package
% Sun Mar 11 17:59:33 2018
\begin{tabular}{lrrr}
  \hline
 & Oracle & Silverman & CV \\ 
  \hline
MISE & 0.001774 & 0.002091 & 0.003161 \\ 
  Relative MISE & 0.240960 & 0.284080 & 0.429375 \\ 
  Normalized MISE & 0.000000 & 0.000000 & 0.000000 \\ 
  MIAE & 0.011174 & 0.011352 & 0.014069 \\ 
  Relative MIAE & 0.130234 & 0.132302 & 0.163978 \\ 
  Normalized MIAE & 0.000001 & 0.000001 & 0.000001 \\ 
  Max Error & 0.210715 & 0.325465 & 0.379102 \\ 
  Normalized Max Error & 0.000021 & 0.000033 & 0.000038 \\ 
  Peak bias & 0.144665 & 0.280264 & 0.322637 \\ 
  Relative Peak bias & 1.686072 & 3.266464 & 3.760320 \\ 
  Peak drift & 1.458091 & 1.115466 & 1.362937 \\ 
  Relative Peak drift & 0.208299 & 0.159352 & 0.194705 \\ 
  Centroid bias & 0.065153 & 0.069904 & 0.108514 \\ 
  Relative Centroid bias & 0.759357 & 0.814731 & 1.264726 \\ 
  Centroid drift & 0.902143 & 0.662575 & 0.876723 \\ 
  Relative Centroid drift & 0.128878 & 0.094654 & 0.125246 \\ 
   \hline
\end{tabular}

    \caption[]{Means} 
    \end{subtable}%
    \begin{subtable}{0.5\textwidth}
    % latex table generated in R 3.4.2 by xtable 1.8-2 package
% Sat Feb 17 16:37:02 2018
\begin{tabular}{lrrr}
  \hline
 & Oracle & Silverman & CV \\ 
  \hline
MISE & 0.006326 & 0.004932 & 0.006532 \\ 
  Relative MISE & 0.859258 & 0.669959 & 0.887313 \\ 
  Normalized MISE & 0.006326 & 0.004932 & 0.006532 \\ 
  MIAE & 0.006350 & 0.005399 & 0.008434 \\ 
  Relative MIAE & 0.074011 & 0.062929 & 0.098298 \\ 
  Max Error & 0.332694 & 0.388475 & 0.446557 \\ 
  Peak bias & 0.332958 & 0.384721 & 0.444374 \\ 
  Relative Peak bias & 3.880619 & 4.483908 & 5.179166 \\ 
  Peak drift & 0.831842 & 0.527184 & 0.608822 \\ 
  Relative Peak drift & 0.118835 & 0.075312 & 0.086975 \\ 
  Centroid bias & 0.122374 & 0.137106 & 0.153321 \\ 
  Relative Centroid bias & 1.426264 & 1.597966 & 1.786947 \\ 
  Centroid drift & 0.472368 & 0.317499 & 0.443602 \\ 
  Relative Centroid drift & 0.067481 & 0.045357 & 0.063372 \\ 
   \hline
\end{tabular}

    \caption[]{Standard deviations} 
    \end{subtable}

\caption[]{Error rates for uniform population of 10,000, single peak intensity of factor 100 and decay rate 1.4, distance between population peak and risk peak is 4}
\label{tbl:mean_error_rates:p1.4_100_1.0_1h_4s}
\end{table}

