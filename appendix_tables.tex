% !TEX root = appendix_tables_only.tex

%%
%%
%% results tables appendix
%%
%%

%%
%% Section
\section{Uniform risk on a uniform population}

\begin{table}[H]
\centering
\scriptsize

    \begin{subtable}{0.5\textwidth}
    % latex table generated in R 3.4.0 by xtable 1.8-2 package
% Sat Aug  5 20:02:24 2017
\begin{tabular}{lrrr}
  \hline
 & Oracle & Silverman & CV \\ 
  \hline
MISE & 0.000012 & 0.000013 & 0.000013 \\ 
  Relative MISE & 0.117027 & 0.127810 & 0.127803 \\ 
  MIAE & 0.002801 & 0.002922 & 0.002922 \\ 
  Relative MIAE & 0.280127 & 0.292163 & 0.292156 \\ 
  Max Error & 0.008572 & 0.009040 & 0.009040 \\ 
  Peak bias & 0.003361 & 0.005616 & 0.005615 \\ 
  Relative Peak bias & 0.336092 & 0.561583 & 0.561492 \\ 
  Peak drift & 5.163912 & 5.190480 & 5.190393 \\ 
  Relative Peak drift & 0.737702 & 0.741497 & 0.741485 \\ 
  Centroid bias & 0.002819 & 0.003733 & 0.003733 \\ 
  Relative Centroid bias & 0.281890 & 0.373345 & 0.373340 \\ 
  Centroid drift & 5.093476 & 5.086491 & 5.086412 \\ 
  Relative Centroid drift & 0.727639 & 0.726642 & 0.726630 \\ 
   \hline
\end{tabular}

    \caption{Means} 
    \end{subtable}%
    \begin{subtable}{0.5\textwidth}
    % latex table generated in R 3.3.3 by xtable 1.8-2 package
% Sun Mar 11 18:27:52 2018
\begin{tabular}{lrrr}
  \hline
 & Oracle & Silverman & CV \\ 
  \hline
MISE & 0.000003 & 0.000003 & 0.000003 \\ 
  Relative MISE & 0.030194 & 0.032029 & 0.034292 \\ 
  Normalized MISE & 0.000000 & 0.000000 & 0.000000 \\ 
  MIAE & 0.000438 & 0.000418 & 0.000446 \\ 
  Relative MIAE & 0.043781 & 0.041769 & 0.044615 \\ 
  Normalized MIAE & 0.000000 & 0.000000 & 0.000000 \\ 
  Max Error & 0.000581 & 0.000870 & 0.001109 \\ 
  Normalized Max Error & 0.000000 & 0.000000 & 0.000000 \\ 
   \hline
\end{tabular}

    \caption{Standard deviations} 
    \end{subtable}

\caption{Error rates for uniform population of 10,000, uniform intensity of factor 100}
\label{tbl:mean_error_rates:unif_100_unif}
\end{table}


%%
%% Section
\section{Varying the number of cases for fixed population of 10,000}

\subsection{50 cases}
\begin{table}[H]
\centering
\scriptsize

    \begin{subtable}{0.5\textwidth}
    % latex table generated in R 3.4.0 by xtable 1.8-2 package
% Sat Aug  5 19:49:51 2017
\begin{table}[ht]
\centering
\begin{tabular}{rrrr}
  \hline
 & Oracle & Silverman & CV \\ 
  \hline
MISE & 0.000008 & 0.000014 & 0.000014 \\ 
  Relative MISE & 0.005208 & 0.008600 & 0.008580 \\ 
  MIAE & 0.001568 & 0.001955 & 0.001953 \\ 
  Relative MIAE & 0.038833 & 0.048413 & 0.048363 \\ 
  Max Error & 0.012435 & 0.018787 & 0.018756 \\ 
  Peak bias & -0.007124 & 0.004056 & 0.004020 \\ 
  Relative Peak bias & -0.176378 & 0.100410 & 0.099522 \\ 
  Peak drift & 0.322404 & 0.476198 & 0.475222 \\ 
  Relative Peak drift & 0.046058 & 0.068028 & 0.067889 \\ 
  Centroid bias & -0.007304 & 0.000496 & 0.000487 \\ 
  Relative Centroid bias & -0.180839 & 0.012284 & 0.012057 \\ 
  Centroid drift & 0.262196 & 0.280772 & 0.281105 \\ 
  Relative Centroid drift & 0.037457 & 0.040110 & 0.040158 \\ 
   \hline
\end{tabular}
\caption{Mean error rates} 
\label{tbl:mean_error_rates}
\end{table}

    \caption{Means} 
    \end{subtable}%
    \begin{subtable}{0.5\textwidth}
    % latex table generated in R 3.4.0 by xtable 1.8-2 package
% Sat Aug  5 19:49:52 2017
\begin{tabular}{lrrr}
  \hline
 & Oracle & Silverman & CV \\ 
  \hline
MISE & 0.000005 & 0.000006 & 0.000006 \\ 
  Relative MISE & 0.002759 & 0.003630 & 0.003613 \\ 
  MIAE & 0.000349 & 0.000321 & 0.000320 \\ 
  Relative MIAE & 0.008640 & 0.007935 & 0.007920 \\ 
  Max Error & 0.004095 & 0.005908 & 0.005887 \\ 
  Peak bias & 0.006126 & 0.010857 & 0.010834 \\ 
  Relative Peak bias & 0.151664 & 0.268788 & 0.268237 \\ 
  Peak drift & 0.179354 & 0.252110 & 0.252185 \\ 
  Relative Peak drift & 0.025622 & 0.036016 & 0.036026 \\ 
  Centroid bias & 0.006226 & 0.011137 & 0.011125 \\ 
  Relative Centroid bias & 0.154153 & 0.275729 & 0.275424 \\ 
  Centroid drift & 0.138102 & 0.145182 & 0.145503 \\ 
  Relative Centroid drift & 0.019729 & 0.020740 & 0.020786 \\ 
   \hline
\end{tabular}

    \caption{Standard deviations} 
    \end{subtable}

\caption{Error rates for uniform population of 10,000, single peak intensity of factor 50}
\label{tbl:mean_error_rates:unif_50_1_1h}
\end{table}

\subsection{100 cases}
\begin{table}[H]
\centering
\scriptsize

    \begin{subtable}{0.5\textwidth}
    % latex table generated in R 3.4.0 by xtable 1.8-2 package
% Sat Aug  5 21:15:22 2017
\begin{tabular}{lrrr}
  \hline
 & Oracle & Silverman & CV \\ 
  \hline
MISE & 0.000022 & 0.000035 & 0.000034 \\ 
  Relative MISE & 0.003358 & 0.005293 & 0.005266 \\ 
  MIAE & 0.002505 & 0.003100 & 0.003092 \\ 
  Relative MIAE & 0.031010 & 0.038378 & 0.038279 \\ 
  Max Error & 0.020705 & 0.030617 & 0.030498 \\ 
  Peak bias & -0.012292 & 0.006167 & 0.006026 \\ 
  Relative Peak bias & -0.152166 & 0.076337 & 0.074592 \\ 
  Peak drift & 0.265067 & 0.409888 & 0.408051 \\ 
  Relative Peak drift & 0.037867 & 0.058555 & 0.058293 \\ 
  Centroid bias & -0.012639 & -0.000422 & -0.000492 \\ 
  Relative Centroid bias & -0.156464 & -0.005226 & -0.006090 \\ 
  Centroid drift & 0.199485 & 0.216521 & 0.216495 \\ 
  Relative Centroid drift & 0.028498 & 0.030932 & 0.030928 \\ 
   \hline
\end{tabular}

    \caption{Means} 
    \end{subtable}%
    \begin{subtable}{0.5\textwidth}
    % latex table generated in R 3.4.0 by xtable 1.8-2 package
% Sat Aug  5 21:15:23 2017
\begin{table}[ht]
\centering
\begin{tabular}{rrrr}
  \hline
 & Oracle & Silverman & CV \\ 
  \hline
MISE & 0.000012 & 0.000013 & 0.000013 \\ 
  Relative MISE & 0.001764 & 0.001964 & 0.001962 \\ 
  MIAE & 0.000525 & 0.000440 & 0.000441 \\ 
  Relative MIAE & 0.006496 & 0.005449 & 0.005460 \\ 
  Max Error & 0.006748 & 0.008658 & 0.008654 \\ 
  Peak bias & 0.009900 & 0.016313 & 0.016285 \\ 
  Relative Peak bias & 0.122549 & 0.201943 & 0.201595 \\ 
  Peak drift & 0.140602 & 0.212548 & 0.211161 \\ 
  Relative Peak drift & 0.020086 & 0.030364 & 0.030166 \\ 
  Centroid bias & 0.010031 & 0.016949 & 0.016909 \\ 
  Relative Centroid bias & 0.124173 & 0.209809 & 0.209321 \\ 
  Centroid drift & 0.106052 & 0.113541 & 0.114205 \\ 
  Relative Centroid drift & 0.015150 & 0.016220 & 0.016315 \\ 
   \hline
\end{tabular}
\caption{Standard deviation of error rates} 
\label{tbl:stddev_error_rates}
\end{table}

    \caption{Standard deviations} 
    \end{subtable}

\caption{Error rates for uniform population of 10,000, single peak intensity of factor 100}
\label{tbl:mean_error_rates:unif_100_1_1h}
\end{table}

\subsection{200 cases}
\begin{table}[H]
\centering
\scriptsize

    \begin{subtable}{0.5\textwidth}
    % latex table generated in R 3.4.0 by xtable 1.8-2 package
% Sat Aug  5 21:44:37 2017
\begin{table}[H]
\centering
\begin{tabular}{lrrr}
  \hline
 & Oracle & Silverman & CV \\ 
  \hline
MISE & 0.000053 & 0.000083 & 0.000082 \\ 
  Relative MISE & 0.002029 & 0.003184 & 0.003129 \\ 
  MIAE & 0.003909 & 0.004852 & 0.004810 \\ 
  Relative MIAE & 0.024196 & 0.030032 & 0.029772 \\ 
  Max Error & 0.033936 & 0.049447 & 0.048810 \\ 
  Peak bias & -0.016473 & 0.010215 & 0.009458 \\ 
  Relative Peak bias & -0.101961 & 0.063224 & 0.058539 \\ 
  Peak drift & 0.219372 & 0.357137 & 0.353058 \\ 
  Relative Peak drift & 0.031339 & 0.051020 & 0.050437 \\ 
  Centroid bias & -0.017532 & -0.001587 & -0.001859 \\ 
  Relative Centroid bias & -0.108518 & -0.009821 & -0.011508 \\ 
  Centroid drift & 0.146327 & 0.158930 & 0.158287 \\ 
  Relative Centroid drift & 0.020904 & 0.022704 & 0.022612 \\ 
   \hline
\end{tabular}
\caption{Mean error rates} 
\label{tbl:mean_error_rates}
\end{table}

    \caption{Means} 
    \end{subtable}%
    \begin{subtable}{0.5\textwidth}
    % latex table generated in R 3.4.0 by xtable 1.8-2 package
% Sat Aug  5 21:44:37 2017
\begin{table}[H]
\centering
\begin{tabular}{lrrr}
  \hline
 & Oracle & Silverman & CV \\ 
  \hline
MISE & 0.000025 & 0.000026 & 0.000026 \\ 
  Relative MISE & 0.000965 & 0.000979 & 0.000979 \\ 
  MIAE & 0.000739 & 0.000595 & 0.000603 \\ 
  Relative MIAE & 0.004575 & 0.003682 & 0.003729 \\ 
  Max Error & 0.010130 & 0.012017 & 0.011950 \\ 
  Peak bias & 0.015575 & 0.023178 & 0.023003 \\ 
  Relative Peak bias & 0.096400 & 0.143460 & 0.142376 \\ 
  Peak drift & 0.121164 & 0.182007 & 0.180423 \\ 
  Relative Peak drift & 0.017309 & 0.026001 & 0.025775 \\ 
  Centroid bias & 0.015861 & 0.024553 & 0.024299 \\ 
  Relative Centroid bias & 0.098172 & 0.151969 & 0.150398 \\ 
  Centroid drift & 0.084649 & 0.089605 & 0.089907 \\ 
  Relative Centroid drift & 0.012093 & 0.012801 & 0.012844 \\ 
   \hline
\end{tabular}
\caption{Standard deviation of error rates} 
\label{tbl:stddev_error_rates}
\end{table}

    \caption{Standard deviations} 
    \end{subtable}

\caption{Error rates for uniform population of 10,000, single peak intensity of factor 200}
\label{tbl:mean_error_rates:unif_200_1_1h}
\end{table}

\subsection{500 cases}
\begin{table}[H]
\centering
\scriptsize

    \begin{subtable}{0.5\textwidth}
    % latex table generated in R 3.4.0 by xtable 1.8-2 package
% Sat Aug  5 22:43:32 2017
\begin{table}[ht]
\centering
\begin{tabular}{rrrr}
  \hline
 & Oracle & Silverman & CV \\ 
  \hline
MISE & 0.000170 & 0.000265 & 0.000244 \\ 
  Relative MISE & 0.001041 & 0.001626 & 0.001496 \\ 
  MIAE & 0.007096 & 0.008775 & 0.008421 \\ 
  Relative MIAE & 0.017567 & 0.021725 & 0.020849 \\ 
  Max Error & 0.065158 & 0.092681 & 0.087429 \\ 
  Peak bias & -0.023250 & 0.018688 & 0.012450 \\ 
  Relative Peak bias & -0.057562 & 0.046269 & 0.030824 \\ 
  Peak drift & 0.176919 & 0.293212 & 0.276800 \\ 
  Relative Peak drift & 0.025274 & 0.041887 & 0.039543 \\ 
  Centroid bias & -0.026539 & -0.003605 & -0.005705 \\ 
  Relative Centroid bias & -0.065707 & -0.008925 & -0.014124 \\ 
  Centroid drift & 0.096196 & 0.104184 & 0.103456 \\ 
  Relative Centroid drift & 0.013742 & 0.014883 & 0.014779 \\ 
   \hline
\end{tabular}
\caption{Mean error rates} 
\label{tbl:mean_error_rates}
\end{table}

    \caption{Means} 
    \end{subtable}%
    \begin{subtable}{0.5\textwidth}
    % latex table generated in R 3.4.0 by xtable 1.8-2 package
% Sat Aug  5 22:43:32 2017
\begin{table}[H]
\centering
\begin{tabular}{lrrr}
  \hline
 & Oracle & Silverman & CV \\ 
  \hline
MISE & 0.000065 & 0.000066 & 0.000068 \\ 
  Relative MISE & 0.000401 & 0.000404 & 0.000417 \\ 
  MIAE & 0.001108 & 0.000892 & 0.000963 \\ 
  Relative MIAE & 0.002744 & 0.002209 & 0.002384 \\ 
  Max Error & 0.016494 & 0.018880 & 0.019274 \\ 
  Peak bias & 0.025669 & 0.034254 & 0.033769 \\ 
  Relative Peak bias & 0.063552 & 0.084807 & 0.083606 \\ 
  Peak drift & 0.096380 & 0.147575 & 0.143744 \\ 
  Relative Peak drift & 0.013769 & 0.021082 & 0.020535 \\ 
  Centroid bias & 0.026146 & 0.037366 & 0.035905 \\ 
  Relative Centroid bias & 0.064733 & 0.092512 & 0.088895 \\ 
  Centroid drift & 0.061402 & 0.064030 & 0.063331 \\ 
  Relative Centroid drift & 0.008772 & 0.009147 & 0.009047 \\ 
   \hline
\end{tabular}
\caption{Standard deviation of error rates} 
\label{tbl:stddev_error_rates}
\end{table}

    \caption{Standard deviations} 
    \end{subtable}

\caption{Error rates for uniform population of 10,000, single peak intensity of factor 500}
\label{tbl:mean_error_rates:unif_500_1_1h}
\end{table}

\subsection{1000 cases}
\begin{table}[H]
\centering
\scriptsize

    \begin{subtable}{0.5\textwidth}
    % latex table generated in R 3.4.0 by xtable 1.8-2 package
% Sun Aug  6 00:15:05 2017
\begin{table}[ht]
\centering
\begin{tabular}{lrrr}
  \hline
 & Oracle & Silverman & CV \\ 
  \hline
MISE & 0.000379 & 0.000619 & 0.000484 \\ 
  Relative MISE & 0.000581 & 0.000948 & 0.000742 \\ 
  MIAE & 0.010804 & 0.013570 & 0.012061 \\ 
  Relative MIAE & 0.013375 & 0.016798 & 0.014931 \\ 
  Max Error & 0.105433 & 0.145677 & 0.126060 \\ 
  Peak bias & -0.034129 & 0.024518 & -0.000092 \\ 
  Relative Peak bias & -0.042248 & 0.030351 & -0.000114 \\ 
  Peak drift & 0.129044 & 0.251876 & 0.185739 \\ 
  Relative Peak drift & 0.018435 & 0.035982 & 0.026534 \\ 
  Centroid bias & -0.037516 & -0.001969 & -0.014152 \\ 
  Relative Centroid bias & -0.046441 & -0.002438 & -0.017519 \\ 
  Centroid drift & 0.068889 & 0.076451 & 0.074548 \\ 
  Relative Centroid drift & 0.009841 & 0.010922 & 0.010650 \\ 
   \hline
\end{tabular}
\caption{Mean error rates} 
\label{tbl:mean_error_rates}
\end{table}

    \caption{Means} 
    \end{subtable}%
    \begin{subtable}{0.5\textwidth}
    % latex table generated in R 3.4.0 by xtable 1.8-2 package
% Sun Aug  6 00:15:05 2017
\begin{table}[ht]
\centering
\begin{tabular}{lrrr}
  \hline
 & Oracle & Silverman & CV \\ 
  \hline
MISE & 0.000111 & 0.000119 & 0.000128 \\ 
  Relative MISE & 0.000170 & 0.000182 & 0.000196 \\ 
  MIAE & 0.001345 & 0.001111 & 0.001360 \\ 
  Relative MIAE & 0.001665 & 0.001375 & 0.001684 \\ 
  Max Error & 0.022746 & 0.025000 & 0.026488 \\ 
  Peak bias & 0.025524 & 0.032193 & 0.031474 \\ 
  Relative Peak bias & 0.031597 & 0.039853 & 0.038962 \\ 
  Peak drift & 0.070303 & 0.116284 & 0.103061 \\ 
  Relative Peak drift & 0.010043 & 0.016612 & 0.014723 \\ 
  Centroid bias & 0.025868 & 0.035816 & 0.031993 \\ 
  Relative Centroid bias & 0.032023 & 0.044337 & 0.039604 \\ 
  Centroid drift & 0.054288 & 0.054756 & 0.054273 \\ 
  Relative Centroid drift & 0.007755 & 0.007822 & 0.007753 \\ 
   \hline
\end{tabular}
\caption{Standard deviation of error rates} 
\label{tbl:stddev_error_rates}
\end{table}

    \caption{Standard deviations} 
    \end{subtable}

\caption{Error rates for uniform population of 10,000, single peak intensity of factor 1000}
\label{tbl:mean_error_rates:unif_1000_1_1h}
\end{table}


%%
%% Section
\section{Varying population and cases together}

\subsection{100 cases from 10,000}

See \autoref{tbl:mean_error_rates:unif_100_1_1h}.

\begin{table}[H]
\centering
\scriptsize

    \begin{subtable}{0.5\textwidth}
    % latex table generated in R 3.4.0 by xtable 1.8-2 package
% Sat Aug  5 21:15:22 2017
\begin{tabular}{lrrr}
  \hline
 & Oracle & Silverman & CV \\ 
  \hline
MISE & 0.000022 & 0.000035 & 0.000034 \\ 
  Relative MISE & 0.003358 & 0.005293 & 0.005266 \\ 
  MIAE & 0.002505 & 0.003100 & 0.003092 \\ 
  Relative MIAE & 0.031010 & 0.038378 & 0.038279 \\ 
  Max Error & 0.020705 & 0.030617 & 0.030498 \\ 
  Peak bias & -0.012292 & 0.006167 & 0.006026 \\ 
  Relative Peak bias & -0.152166 & 0.076337 & 0.074592 \\ 
  Peak drift & 0.265067 & 0.409888 & 0.408051 \\ 
  Relative Peak drift & 0.037867 & 0.058555 & 0.058293 \\ 
  Centroid bias & -0.012639 & -0.000422 & -0.000492 \\ 
  Relative Centroid bias & -0.156464 & -0.005226 & -0.006090 \\ 
  Centroid drift & 0.199485 & 0.216521 & 0.216495 \\ 
  Relative Centroid drift & 0.028498 & 0.030932 & 0.030928 \\ 
   \hline
\end{tabular}

    \caption{Means} 
    \end{subtable}%
    \begin{subtable}{0.5\textwidth}
    % latex table generated in R 3.4.0 by xtable 1.8-2 package
% Sat Aug  5 21:15:23 2017
\begin{table}[ht]
\centering
\begin{tabular}{rrrr}
  \hline
 & Oracle & Silverman & CV \\ 
  \hline
MISE & 0.000012 & 0.000013 & 0.000013 \\ 
  Relative MISE & 0.001764 & 0.001964 & 0.001962 \\ 
  MIAE & 0.000525 & 0.000440 & 0.000441 \\ 
  Relative MIAE & 0.006496 & 0.005449 & 0.005460 \\ 
  Max Error & 0.006748 & 0.008658 & 0.008654 \\ 
  Peak bias & 0.009900 & 0.016313 & 0.016285 \\ 
  Relative Peak bias & 0.122549 & 0.201943 & 0.201595 \\ 
  Peak drift & 0.140602 & 0.212548 & 0.211161 \\ 
  Relative Peak drift & 0.020086 & 0.030364 & 0.030166 \\ 
  Centroid bias & 0.010031 & 0.016949 & 0.016909 \\ 
  Relative Centroid bias & 0.124173 & 0.209809 & 0.209321 \\ 
  Centroid drift & 0.106052 & 0.113541 & 0.114205 \\ 
  Relative Centroid drift & 0.015150 & 0.016220 & 0.016315 \\ 
   \hline
\end{tabular}
\caption{Standard deviation of error rates} 
\label{tbl:stddev_error_rates}
\end{table}

    \caption{Standard deviations} 
    \end{subtable}

\caption{Error rates for uniform population of 10,000, single peak intensity of factor 100}
\label{tbl:mean_error_rates:unif_100_1_1h:2}
\end{table}

\subsection{200 cases from 20,000}
\begin{table}[H]
\centering
\scriptsize

    \begin{subtable}{0.5\textwidth}
    % latex table generated in R 3.4.0 by xtable 1.8-2 package
% Sun Aug 13 13:20:48 2017
\begin{table}[ht]
\centering
\begin{tabular}{rrrr}
  \hline
 & Oracle & Silverman & CV \\ 
  \hline
MISE & 0.000015 & 0.000023 & 0.000022 \\ 
  Relative MISE & 0.002372 & 0.003481 & 0.003425 \\ 
  MIAE & 0.002069 & 0.002496 & 0.002476 \\ 
  Relative MIAE & 0.025606 & 0.030901 & 0.030651 \\ 
  Max Error & 0.018984 & 0.026091 & 0.025789 \\ 
  Peak bias & -0.011406 & 0.001175 & 0.000799 \\ 
  Relative Peak bias & -0.141193 & 0.014544 & 0.009891 \\ 
  Peak drift & 0.291837 & 0.422926 & 0.420581 \\ 
  Relative Peak drift & 0.041691 & 0.060418 & 0.060083 \\ 
  Centroid bias & -0.012288 & -0.006236 & -0.006300 \\ 
  Relative Centroid bias & -0.152112 & -0.077190 & -0.077988 \\ 
  Centroid drift & 0.171960 & 0.190351 & 0.190355 \\ 
  Relative Centroid drift & 0.024566 & 0.027193 & 0.027194 \\ 
   \hline
\end{tabular}
\caption{Mean error rates} 
\label{tbl:mean_error_rates}
\end{table}

    \caption{Means} 
    \end{subtable}%
    \begin{subtable}{0.5\textwidth}
    % latex table generated in R 3.4.0 by xtable 1.8-2 package
% Sun Aug 13 13:20:48 2017
\begin{table}[ht]
\centering
\begin{tabular}{rrrr}
  \hline
 & Oracle & Silverman & CV \\ 
  \hline
MISE & 0.000007 & 0.000007 & 0.000007 \\ 
  Relative MISE & 0.001051 & 0.001092 & 0.001093 \\ 
  MIAE & 0.000370 & 0.000314 & 0.000318 \\ 
  Relative MIAE & 0.004574 & 0.003892 & 0.003931 \\ 
  Max Error & 0.005609 & 0.005995 & 0.006009 \\ 
  Peak bias & 0.007725 & 0.011211 & 0.011137 \\ 
  Relative Peak bias & 0.095629 & 0.138785 & 0.137868 \\ 
  Peak drift & 0.155419 & 0.211609 & 0.209799 \\ 
  Relative Peak drift & 0.022203 & 0.030230 & 0.029971 \\ 
  Centroid bias & 0.007989 & 0.012125 & 0.012026 \\ 
  Relative Centroid bias & 0.098899 & 0.150096 & 0.148867 \\ 
  Centroid drift & 0.097875 & 0.103910 & 0.104528 \\ 
  Relative Centroid drift & 0.013982 & 0.014844 & 0.014933 \\ 
   \hline
\end{tabular}
\caption{Standard deviation of error rates} 
\label{tbl:stddev_error_rates}
\end{table}

    \caption{Standard deviations} 
    \end{subtable}

\caption{Error rates for uniform population of 20,000, single peak intensity of factor 200}
\label{tbl:mean_error_rates:unif20k_200_1_1h}
\end{table}

\subsection{400 cases from 40,000}
\begin{table}[H]
\centering
\scriptsize

    \begin{subtable}{0.5\textwidth}
    % latex table generated in R 3.4.0 by xtable 1.8-2 package
% Sun Aug 13 14:16:06 2017
\begin{table}[ht]
\centering
\begin{tabular}{rrrr}
  \hline
 & Oracle & Silverman & CV \\ 
  \hline
MISE & 0.000010 & 0.000014 & 0.000013 \\ 
  Relative MISE & 0.001506 & 0.002142 & 0.002051 \\ 
  MIAE & 0.001642 & 0.001969 & 0.001926 \\ 
  Relative MIAE & 0.020323 & 0.024376 & 0.023837 \\ 
  Max Error & 0.015691 & 0.021412 & 0.020759 \\ 
  Peak bias & -0.009361 & 0.001818 & 0.000969 \\ 
  Relative Peak bias & -0.115878 & 0.022501 & 0.011994 \\ 
  Peak drift & 0.224460 & 0.348102 & 0.343680 \\ 
  Relative Peak drift & 0.032066 & 0.049729 & 0.049097 \\ 
  Centroid bias & -0.010033 & -0.004142 & -0.004456 \\ 
  Relative Centroid bias & -0.124197 & -0.051270 & -0.055156 \\ 
  Centroid drift & 0.125798 & 0.138387 & 0.137610 \\ 
  Relative Centroid drift & 0.017971 & 0.019770 & 0.019659 \\ 
   \hline
\end{tabular}
\caption{Mean error rates} 
\label{tbl:mean_error_rates}
\end{table}

    \caption{Means} 
    \end{subtable}%
    \begin{subtable}{0.5\textwidth}
    % latex table generated in R 3.4.0 by xtable 1.8-2 package
% Sun Aug 13 14:16:06 2017
\begin{table}[ht]
\centering
\begin{tabular}{rrrr}
  \hline
 & Oracle & Silverman & CV \\ 
  \hline
MISE & 0.000004 & 0.000004 & 0.000004 \\ 
  Relative MISE & 0.000659 & 0.000601 & 0.000609 \\ 
  MIAE & 0.000278 & 0.000220 & 0.000229 \\ 
  Relative MIAE & 0.003442 & 0.002723 & 0.002830 \\ 
  Max Error & 0.004435 & 0.004672 & 0.004730 \\ 
  Peak bias & 0.006121 & 0.008856 & 0.008732 \\ 
  Relative Peak bias & 0.075776 & 0.109626 & 0.108095 \\ 
  Peak drift & 0.116624 & 0.171068 & 0.166589 \\ 
  Relative Peak drift & 0.016661 & 0.024438 & 0.023798 \\ 
  Centroid bias & 0.006271 & 0.009763 & 0.009491 \\ 
  Relative Centroid bias & 0.077629 & 0.120854 & 0.117487 \\ 
  Centroid drift & 0.073622 & 0.078391 & 0.079245 \\ 
  Relative Centroid drift & 0.010517 & 0.011199 & 0.011321 \\ 
   \hline
\end{tabular}
\caption{Standard deviation of error rates} 
\label{tbl:stddev_error_rates}
\end{table}

    \caption{Standard deviations} 
    \end{subtable}

\caption{Error rates for uniform population of 40,000, single peak intensity of factor 400}
\label{tbl:mean_error_rates:unif40k_400_1_1h}
\end{table}

\subsection{600 cases from 60,000}
\begin{table}[H]
\centering
\scriptsize

    \begin{subtable}{0.5\textwidth}
    % latex table generated in R 3.4.0 by xtable 1.8-2 package
% Sun Aug 13 15:09:40 2017
\begin{table}[ht]
\centering
\begin{tabular}{rrrr}
  \hline
 & Oracle & Silverman & CV \\ 
  \hline
MISE & 0.000007 & 0.000011 & 0.000010 \\ 
  Relative MISE & 0.001127 & 0.001630 & 0.001509 \\ 
  MIAE & 0.001426 & 0.001718 & 0.001650 \\ 
  Relative MIAE & 0.017653 & 0.021262 & 0.020430 \\ 
  Max Error & 0.013768 & 0.019194 & 0.018128 \\ 
  Peak bias & -0.007019 & 0.002555 & 0.001164 \\ 
  Relative Peak bias & -0.086889 & 0.031633 & 0.014413 \\ 
  Peak drift & 0.183350 & 0.302506 & 0.287383 \\ 
  Relative Peak drift & 0.026193 & 0.043215 & 0.041055 \\ 
  Centroid bias & -0.007706 & -0.003067 & -0.003522 \\ 
  Relative Centroid bias & -0.095392 & -0.037962 & -0.043602 \\ 
  Centroid drift & 0.097719 & 0.112315 & 0.111326 \\ 
  Relative Centroid drift & 0.013960 & 0.016045 & 0.015904 \\ 
   \hline
\end{tabular}
\caption{Mean error rates} 
\label{tbl:mean_error_rates}
\end{table}

    \caption{Means} 
    \end{subtable}%
    \begin{subtable}{0.5\textwidth}
    % latex table generated in R 3.4.0 by xtable 1.8-2 package
% Sun Aug 13 15:09:40 2017
\begin{table}[ht]
\centering
\begin{tabular}{rrrr}
  \hline
 & Oracle & Silverman & CV \\ 
  \hline
MISE & 0.000003 & 0.000003 & 0.000003 \\ 
  Relative MISE & 0.000484 & 0.000431 & 0.000453 \\ 
  MIAE & 0.000231 & 0.000177 & 0.000195 \\ 
  Relative MIAE & 0.002861 & 0.002194 & 0.002409 \\ 
  Max Error & 0.003864 & 0.004055 & 0.004263 \\ 
  Peak bias & 0.005921 & 0.008327 & 0.008273 \\ 
  Relative Peak bias & 0.073291 & 0.103080 & 0.102413 \\ 
  Peak drift & 0.106179 & 0.150642 & 0.148199 \\ 
  Relative Peak drift & 0.015168 & 0.021520 & 0.021171 \\ 
  Centroid bias & 0.006052 & 0.008976 & 0.008647 \\ 
  Relative Centroid bias & 0.074912 & 0.111120 & 0.107043 \\ 
  Centroid drift & 0.064924 & 0.069500 & 0.068346 \\ 
  Relative Centroid drift & 0.009275 & 0.009929 & 0.009764 \\ 
   \hline
\end{tabular}
\caption{Standard deviation of error rates} 
\label{tbl:stddev_error_rates}
\end{table}

    \caption{Standard deviations} 
    \end{subtable}

\caption{Error rates for uniform population of 60,000, single peak intensity of factor 600}
\label{tbl:mean_error_rates:unif60k_600_1_1h}
\end{table}

\subsection{800 cases from 80,000}
\begin{table}[H]
\centering
\scriptsize

    \begin{subtable}{0.5\textwidth}
    % latex table generated in R 3.4.0 by xtable 1.8-2 package
% Sun Aug 13 16:07:13 2017
\begin{tabular}{rrrr}
  \hline
 & Oracle & Silverman & CV \\ 
  \hline
MISE & 0.000006 & 0.000009 & 0.000008 \\ 
  Relative MISE & 0.000891 & 0.001335 & 0.001173 \\ 
  MIAE & 0.001278 & 0.001556 & 0.001458 \\ 
  Relative MIAE & 0.015821 & 0.019260 & 0.018043 \\ 
  Max Error & 0.012651 & 0.017620 & 0.016067 \\ 
  Peak bias & -0.006430 & 0.002183 & 0.000162 \\ 
  Relative Peak bias & -0.079594 & 0.027022 & 0.002006 \\ 
  Peak drift & 0.189578 & 0.304958 & 0.282499 \\ 
  Relative Peak drift & 0.027083 & 0.043565 & 0.040357 \\ 
  Centroid bias & -0.007286 & -0.003685 & -0.004249 \\ 
  Relative Centroid bias & -0.090199 & -0.045614 & -0.052603 \\ 
  Centroid drift & 0.083142 & 0.094548 & 0.091757 \\ 
  Relative Centroid drift & 0.011877 & 0.013507 & 0.013108 \\ 
   \hline
\end{tabular}

    \caption{Means} 
    \end{subtable}%
    \begin{subtable}{0.5\textwidth}
    % latex table generated in R 3.4.0 by xtable 1.8-2 package
% Sun Aug 13 16:07:13 2017
\begin{tabular}{rrrr}
  \hline
 & Oracle & Silverman & CV \\ 
  \hline
MISE & 0.000002 & 0.000002 & 0.000002 \\ 
  Relative MISE & 0.000342 & 0.000328 & 0.000353 \\ 
  MIAE & 0.000186 & 0.000146 & 0.000173 \\ 
  Relative MIAE & 0.002300 & 0.001813 & 0.002141 \\ 
  Max Error & 0.003414 & 0.003799 & 0.003941 \\ 
  Peak bias & 0.005078 & 0.007028 & 0.006866 \\ 
  Relative Peak bias & 0.062864 & 0.086995 & 0.084993 \\ 
  Peak drift & 0.106356 & 0.140820 & 0.133500 \\ 
  Relative Peak drift & 0.015194 & 0.020117 & 0.019071 \\ 
  Centroid bias & 0.005216 & 0.007790 & 0.007253 \\ 
  Relative Centroid bias & 0.064570 & 0.096434 & 0.089789 \\ 
  Centroid drift & 0.061492 & 0.062965 & 0.062084 \\ 
  Relative Centroid drift & 0.008785 & 0.008995 & 0.008869 \\ 
   \hline
\end{tabular}

    \caption{Standard deviations} 
    \end{subtable}

\caption{Error rates for uniform population of 80,000, single peak intensity of factor 800}
\label{tbl:mean_error_rates:unif80k_800_1_1h}
\end{table}

\subsection{1000 cases from 100,000}
\begin{table}[H]
\centering
\scriptsize

    \begin{subtable}{0.5\textwidth}
    % latex table generated in R 3.4.0 by xtable 1.8-2 package
% Sun Aug  6 00:13:35 2017
\begin{table}[ht]
\centering
\begin{tabular}{lrrr}
  \hline
 & Oracle & Silverman & CV \\ 
  \hline
MISE & 0.000005 & 0.000007 & 0.000006 \\ 
  Relative MISE & 0.000780 & 0.001138 & 0.000957 \\ 
  MIAE & 0.001194 & 0.001443 & 0.001318 \\ 
  Relative MIAE & 0.014783 & 0.017860 & 0.016319 \\ 
  Max Error & 0.012033 & 0.016294 & 0.014388 \\ 
  Peak bias & -0.006071 & 0.001338 & -0.001420 \\ 
  Relative Peak bias & -0.075150 & 0.016561 & -0.017584 \\ 
  Peak drift & 0.191971 & 0.283546 & 0.257819 \\ 
  Relative Peak drift & 0.027424 & 0.040507 & 0.036831 \\ 
  Centroid bias & -0.007017 & -0.003590 & -0.004719 \\ 
  Relative Centroid bias & -0.086858 & -0.044443 & -0.058412 \\ 
  Centroid drift & 0.077427 & 0.087651 & 0.084033 \\ 
  Relative Centroid drift & 0.011061 & 0.012522 & 0.012005 \\ 
   \hline
\end{tabular}
\caption{Mean error rates} 
\label{tbl:mean_error_rates}
\end{table}

    \caption{Means} 
    \end{subtable}%
    \begin{subtable}{0.5\textwidth}
    % latex table generated in R 3.4.0 by xtable 1.8-2 package
% Sun Aug  6 00:13:36 2017
\begin{tabular}{lrrr}
  \hline
 & Oracle & Silverman & CV \\ 
  \hline
MISE & 0.000002 & 0.000002 & 0.000002 \\ 
  Relative MISE & 0.000296 & 0.000275 & 0.000312 \\ 
  MIAE & 0.000172 & 0.000134 & 0.000169 \\ 
  Relative MIAE & 0.002123 & 0.001663 & 0.002093 \\ 
  Max Error & 0.003190 & 0.003425 & 0.003799 \\ 
  Peak bias & 0.004776 & 0.006378 & 0.006163 \\ 
  Relative Peak bias & 0.059119 & 0.078949 & 0.076294 \\ 
  Peak drift & 0.106483 & 0.145200 & 0.134117 \\ 
  Relative Peak drift & 0.015212 & 0.020743 & 0.019160 \\ 
  Centroid bias & 0.004937 & 0.007348 & 0.006476 \\ 
  Relative Centroid bias & 0.061119 & 0.090964 & 0.080164 \\ 
  Centroid drift & 0.059317 & 0.062619 & 0.061337 \\ 
  Relative Centroid drift & 0.008474 & 0.008946 & 0.008762 \\ 
   \hline
\end{tabular}

    \caption{Standard deviations} 
    \end{subtable}

\caption{Error rates for uniform population of 100,000, single peak intensity of factor 1000}
\label{tbl:mean_error_rates:unif100k_1000_1_1h}
\end{table}


%%
%% Section
\section{Varying the decay of the risk function}


\subsection{100 cases from 10,000 with no decay (uniform)}

See \autoref{tbl:mean_error_rates:unif_100_unif}.

\begin{table}[H]
\centering
\scriptsize

    \begin{subtable}{0.5\textwidth}
    % latex table generated in R 3.4.0 by xtable 1.8-2 package
% Sat Aug  5 20:02:24 2017
\begin{tabular}{lrrr}
  \hline
 & Oracle & Silverman & CV \\ 
  \hline
MISE & 0.000012 & 0.000013 & 0.000013 \\ 
  Relative MISE & 0.117027 & 0.127810 & 0.127803 \\ 
  MIAE & 0.002801 & 0.002922 & 0.002922 \\ 
  Relative MIAE & 0.280127 & 0.292163 & 0.292156 \\ 
  Max Error & 0.008572 & 0.009040 & 0.009040 \\ 
  Peak bias & 0.003361 & 0.005616 & 0.005615 \\ 
  Relative Peak bias & 0.336092 & 0.561583 & 0.561492 \\ 
  Peak drift & 5.163912 & 5.190480 & 5.190393 \\ 
  Relative Peak drift & 0.737702 & 0.741497 & 0.741485 \\ 
  Centroid bias & 0.002819 & 0.003733 & 0.003733 \\ 
  Relative Centroid bias & 0.281890 & 0.373345 & 0.373340 \\ 
  Centroid drift & 5.093476 & 5.086491 & 5.086412 \\ 
  Relative Centroid drift & 0.727639 & 0.726642 & 0.726630 \\ 
   \hline
\end{tabular}

    \caption{Means} 
    \end{subtable}%
    \begin{subtable}{0.5\textwidth}
    % latex table generated in R 3.3.3 by xtable 1.8-2 package
% Sun Mar 11 18:27:52 2018
\begin{tabular}{lrrr}
  \hline
 & Oracle & Silverman & CV \\ 
  \hline
MISE & 0.000003 & 0.000003 & 0.000003 \\ 
  Relative MISE & 0.030194 & 0.032029 & 0.034292 \\ 
  Normalized MISE & 0.000000 & 0.000000 & 0.000000 \\ 
  MIAE & 0.000438 & 0.000418 & 0.000446 \\ 
  Relative MIAE & 0.043781 & 0.041769 & 0.044615 \\ 
  Normalized MIAE & 0.000000 & 0.000000 & 0.000000 \\ 
  Max Error & 0.000581 & 0.000870 & 0.001109 \\ 
  Normalized Max Error & 0.000000 & 0.000000 & 0.000000 \\ 
   \hline
\end{tabular}

    \caption{Standard deviations} 
    \end{subtable}

\caption{Error rates for uniform population of 10,000, single peak intensity of factor 100 and no decay (uniform)}
\label{tbl:mean_error_rates:unif_100_unif:2}
\end{table}

\subsection{100 cases from 10,000 with intensity decay rate 2.0}
\begin{table}[H]
\centering
\scriptsize

    \begin{subtable}{0.5\textwidth}
    % latex table generated in R 3.4.0 by xtable 1.8-2 package
% Sat Aug  5 21:31:40 2017
\begin{table}[H]
\centering
\begin{tabular}{lrrr}
  \hline
 & Oracle & Silverman & CV \\ 
  \hline
MISE & 0.000006 & 0.000012 & 0.000012 \\ 
  Relative MISE & 0.011308 & 0.021333 & 0.021316 \\ 
  MIAE & 0.001867 & 0.002638 & 0.002637 \\ 
  Relative MIAE & 0.079514 & 0.112328 & 0.112282 \\ 
  Max Error & 0.006153 & 0.010407 & 0.010401 \\ 
  Peak bias & -0.003225 & 0.003258 & 0.003251 \\ 
  Relative Peak bias & -0.137322 & 0.138713 & 0.138426 \\ 
  Peak drift & 0.615800 & 0.900437 & 0.900381 \\ 
  Relative Peak drift & 0.087971 & 0.128634 & 0.128626 \\ 
  Centroid bias & -0.003271 & 0.001927 & 0.001923 \\ 
  Relative Centroid bias & -0.139292 & 0.082052 & 0.081892 \\ 
  Centroid drift & 0.534199 & 0.668770 & 0.668461 \\ 
  Relative Centroid drift & 0.076314 & 0.095539 & 0.095494 \\ 
   \hline
\end{tabular}
\caption{Mean error rates} 
\label{tbl:mean_error_rates}
\end{table}

    \caption{Means} 
    \end{subtable}%
    \begin{subtable}{0.5\textwidth}
    % latex table generated in R 3.4.0 by xtable 1.8-2 package
% Sat Aug  5 21:31:40 2017
\begin{table}[H]
\centering
\begin{tabular}{lrrr}
  \hline
 & Oracle & Silverman & CV \\ 
  \hline
MISE & 0.000004 & 0.000004 & 0.000004 \\ 
  Relative MISE & 0.007039 & 0.007532 & 0.007529 \\ 
  MIAE & 0.000542 & 0.000431 & 0.000431 \\ 
  Relative MIAE & 0.023067 & 0.018346 & 0.018348 \\ 
  Max Error & 0.001995 & 0.002548 & 0.002547 \\ 
  Peak bias & 0.002817 & 0.004825 & 0.004823 \\ 
  Relative Peak bias & 0.119960 & 0.205429 & 0.205337 \\ 
  Peak drift & 0.333560 & 0.493778 & 0.493829 \\ 
  Relative Peak drift & 0.047651 & 0.070540 & 0.070547 \\ 
  Centroid bias & 0.002842 & 0.005453 & 0.005452 \\ 
  Relative Centroid bias & 0.121027 & 0.232194 & 0.232131 \\ 
  Centroid drift & 0.287704 & 0.353528 & 0.353490 \\ 
  Relative Centroid drift & 0.041101 & 0.050504 & 0.050499 \\ 
   \hline
\end{tabular}
\caption{Standard deviation of error rates} 
\label{tbl:stddev_error_rates}
\end{table}

    \caption{Standard deviations} 
    \end{subtable}

\caption{Error rates for uniform population of 10,000, single peak intensity of factor 100 and decay rate 2.0}
\label{tbl:mean_error_rates:unif_100_2_1h}
\end{table}

\subsection{100 cases from 10,000 with intensity decay rate 1.4}
\begin{table}[H]
\centering
\scriptsize

    \begin{subtable}{0.5\textwidth}
    % latex table generated in R 3.4.3 by xtable 1.8-2 package
% Sun Apr 15 09:02:16 2018
\begin{tabular}{lrrr}
  \toprule
 & Oracle & Silverman & CV \\ 
  \midrule
MISE & 0.000012 & 0.000019 & 0.000016 \\ 
  Relative MISE & 0.006627 & 0.010652 & 0.009328 \\ 
  Normalized MISE & 1.162791 & 1.869138 & 1.636752 \\ 
  MIAE & 0.002291 & 0.002949 & 0.002662 \\ 
  Relative MIAE & 0.054683 & 0.070397 & 0.063559 \\ 
  Normalized MIAE & 0.000023 & 0.000029 & 0.000027 \\ 
  Supremum error & 0.011293 & 0.016460 & 0.014595 \\ 
  Normalized Sup error & 0.000113 & 0.000165 & 0.000146 \\ 
  Peak bias & -0.005739 & 0.003331 & -0.001423 \\ 
  Relative Peak bias & -0.137007 & 0.079526 & -0.033973 \\ 
  Peak drift & 0.407527 & 0.576641 & 0.466724 \\ 
  Relative Peak drift & 0.058218 & 0.082377 & 0.066675 \\ 
  Centroid bias & -0.005876 & 0.001197 & -0.003106 \\ 
  Relative Centroid bias & -0.140265 & 0.028578 & -0.074149 \\ 
  Centroid drift & 0.337723 & 0.384568 & 0.351105 \\ 
  Relative Centroid drift & 0.048246 & 0.054938 & 0.050158 \\ 
   \bottomrule
\end{tabular}

    \caption{Means} 
    \end{subtable}%
    \begin{subtable}{0.5\textwidth}
    % latex table generated in R 3.4.2 by xtable 1.8-2 package
% Sat Feb 17 16:41:31 2018
\begin{tabular}{lrrr}
  \hline
 & Oracle & Silverman & CV \\ 
  \hline
MISE & 0.000006 & 0.000007 & 0.000010 \\ 
  Relative MISE & 0.003137 & 0.003786 & 0.005457 \\ 
  Normalized MISE & 0.000006 & 0.000007 & 0.000010 \\ 
  MIAE & 0.000459 & 0.000437 & 0.000642 \\ 
  Relative MIAE & 0.010946 & 0.010433 & 0.015328 \\ 
  Max Error & 0.003239 & 0.004242 & 0.006490 \\ 
  Peak bias & 0.005135 & 0.008388 & 0.010433 \\ 
  Relative Peak bias & 0.122587 & 0.200238 & 0.249069 \\ 
  Peak drift & 0.224036 & 0.304266 & 0.273448 \\ 
  Relative Peak drift & 0.032005 & 0.043467 & 0.039064 \\ 
  Centroid bias & 0.005205 & 0.009014 & 0.008951 \\ 
  Relative Centroid bias & 0.124269 & 0.215189 & 0.213674 \\ 
  Centroid drift & 0.178480 & 0.198033 & 0.184361 \\ 
  Relative Centroid drift & 0.025497 & 0.028290 & 0.026337 \\ 
   \hline
\end{tabular}

    \caption{Standard deviations} 
    \end{subtable}

\caption{Error rates for uniform population of 10,000, single peak intensity of factor 100 and decay rate 1.4}
\label{tbl:mean_error_rates:unif_100_1.4_1h}
\end{table}

\subsection{100 cases from 10,000 with intensity decay rate 1.0}

See \autoref{tbl:mean_error_rates:unif_100_1_1h}
\begin{table}[H]
\centering
\scriptsize

    \begin{subtable}{0.5\textwidth}
    % latex table generated in R 3.4.0 by xtable 1.8-2 package
% Sat Aug  5 21:15:22 2017
\begin{tabular}{lrrr}
  \hline
 & Oracle & Silverman & CV \\ 
  \hline
MISE & 0.000022 & 0.000035 & 0.000034 \\ 
  Relative MISE & 0.003358 & 0.005293 & 0.005266 \\ 
  MIAE & 0.002505 & 0.003100 & 0.003092 \\ 
  Relative MIAE & 0.031010 & 0.038378 & 0.038279 \\ 
  Max Error & 0.020705 & 0.030617 & 0.030498 \\ 
  Peak bias & -0.012292 & 0.006167 & 0.006026 \\ 
  Relative Peak bias & -0.152166 & 0.076337 & 0.074592 \\ 
  Peak drift & 0.265067 & 0.409888 & 0.408051 \\ 
  Relative Peak drift & 0.037867 & 0.058555 & 0.058293 \\ 
  Centroid bias & -0.012639 & -0.000422 & -0.000492 \\ 
  Relative Centroid bias & -0.156464 & -0.005226 & -0.006090 \\ 
  Centroid drift & 0.199485 & 0.216521 & 0.216495 \\ 
  Relative Centroid drift & 0.028498 & 0.030932 & 0.030928 \\ 
   \hline
\end{tabular}

    \caption{Means} 
    \end{subtable}%
    \begin{subtable}{0.5\textwidth}
    % latex table generated in R 3.4.0 by xtable 1.8-2 package
% Sat Aug  5 21:15:23 2017
\begin{table}[ht]
\centering
\begin{tabular}{rrrr}
  \hline
 & Oracle & Silverman & CV \\ 
  \hline
MISE & 0.000012 & 0.000013 & 0.000013 \\ 
  Relative MISE & 0.001764 & 0.001964 & 0.001962 \\ 
  MIAE & 0.000525 & 0.000440 & 0.000441 \\ 
  Relative MIAE & 0.006496 & 0.005449 & 0.005460 \\ 
  Max Error & 0.006748 & 0.008658 & 0.008654 \\ 
  Peak bias & 0.009900 & 0.016313 & 0.016285 \\ 
  Relative Peak bias & 0.122549 & 0.201943 & 0.201595 \\ 
  Peak drift & 0.140602 & 0.212548 & 0.211161 \\ 
  Relative Peak drift & 0.020086 & 0.030364 & 0.030166 \\ 
  Centroid bias & 0.010031 & 0.016949 & 0.016909 \\ 
  Relative Centroid bias & 0.124173 & 0.209809 & 0.209321 \\ 
  Centroid drift & 0.106052 & 0.113541 & 0.114205 \\ 
  Relative Centroid drift & 0.015150 & 0.016220 & 0.016315 \\ 
   \hline
\end{tabular}
\caption{Standard deviation of error rates} 
\label{tbl:stddev_error_rates}
\end{table}

    \caption{Standard deviations} 
    \end{subtable}

\caption{Error rates for uniform population of 10,000, single peak intensity of factor 100 and decay rate 1.0}
\label{tbl:mean_error_rates:unif_100_1_1h:3}
\end{table}

\subsection{100 cases from 10,000 with intensity decay rate 0.7}
\begin{table}[H]
\centering
\scriptsize

    \begin{subtable}{0.5\textwidth}
    % latex table generated in R 3.4.0 by xtable 1.8-2 package
% Sat Aug  5 20:24:52 2017
\begin{table}[ht]
\centering
\begin{tabular}{lrrr}
  \hline
 & Oracle & Silverman & CV \\ 
  \hline
MISE & 0.000041 & 0.000070 & 0.000068 \\ 
  Relative MISE & 0.001509 & 0.002587 & 0.002508 \\ 
  MIAE & 0.002424 & 0.003082 & 0.003036 \\ 
  Relative MIAE & 0.014708 & 0.018700 & 0.018421 \\ 
  Max Error & 0.040419 & 0.063188 & 0.061764 \\ 
  Peak bias & -0.024625 & 0.014782 & 0.013157 \\ 
  Relative Peak bias & -0.149402 & 0.089681 & 0.079821 \\ 
  Peak drift & 0.178185 & 0.309841 & 0.306002 \\ 
  Relative Peak drift & 0.025455 & 0.044263 & 0.043715 \\ 
  Centroid bias & -0.026327 & -0.008016 & -0.008467 \\ 
  Relative Centroid bias & -0.159727 & -0.048632 & -0.051370 \\ 
  Centroid drift & 0.110594 & 0.115124 & 0.114988 \\ 
  Relative Centroid drift & 0.015799 & 0.016446 & 0.016427 \\ 
   \hline
\end{tabular}
\caption{Mean error rates} 
\label{tbl:mean_error_rates}
\end{table}

    \caption{Means} 
    \end{subtable}%
    \begin{subtable}{0.5\textwidth}
    % latex table generated in R 3.4.2 by xtable 1.8-2 package
% Thu Dec  7 16:57:46 2017
\begin{tabular}{lrrr}
  \hline
 & Oracle & Silverman & CV \\ 
  \hline
MISE & 0.000020 & 0.000027 & 0.000027 \\ 
  Relative MISE & 0.000739 & 0.000989 & 0.000982 \\ 
  MIAE & 0.000484 & 0.000443 & 0.000451 \\ 
  Relative MIAE & 0.002936 & 0.002687 & 0.002739 \\ 
  Max Error & 0.012449 & 0.016956 & 0.017061 \\ 
  Peak bias & 0.019842 & 0.030937 & 0.030728 \\ 
  Relative Peak bias & 0.120383 & 0.187693 & 0.186424 \\ 
  Peak drift & 0.105135 & 0.157797 & 0.155232 \\ 
  Relative Peak drift & 0.015019 & 0.022542 & 0.022176 \\ 
  Centroid bias & 0.020297 & 0.031820 & 0.031366 \\ 
  Relative Centroid bias & 0.123139 & 0.193052 & 0.190296 \\ 
  Centroid drift & 0.067499 & 0.069453 & 0.069624 \\ 
  Relative Centroid drift & 0.009643 & 0.009922 & 0.009946 \\ 
   \hline
\end{tabular}

    \caption{Standard deviations} 
    \end{subtable}

\caption{Error rates for uniform population of 10,000, single peak intensity of factor 100 and decay rate 0.7}
\label{tbl:mean_error_rates:unif_100_0.7_1h}
\end{table}

\subsection{100 cases from 10,000 with intensity decay rate 0.5}
\begin{table}[H]
\centering
\scriptsize

    \begin{subtable}{0.5\textwidth}
    % latex table generated in R 3.4.0 by xtable 1.8-2 package
% Sun Jul 30 10:54:52 2017
\begin{table}[ht]
\centering
\begin{tabular}{rrrr}
  \hline
 & Oracle & Silverman & CV \\ 
  \hline
MISE & 0.000036 & 0.000046 & 0.000063 \\ 
  Relative MISE & 0.000340 & 0.000438 & 0.000607 \\ 
  MIAE & 0.000466 & 0.000425 & 0.000563 \\ 
  Relative MIAE & 0.001442 & 0.001314 & 0.001741 \\ 
  Max Error & 0.023200 & 0.030300 & 0.039341 \\ 
  Peak bias & 0.032038 & 0.050895 & 0.055081 \\ 
  Relative Peak bias & 0.099132 & 0.157480 & 0.170430 \\ 
  Peak drift & 0.080767 & 0.118526 & 0.114042 \\ 
  Relative Peak drift & 0.011538 & 0.016932 & 0.016292 \\ 
  Centroid bias & 0.033350 & 0.057284 & 0.053665 \\ 
  Relative Centroid bias & 0.103192 & 0.177250 & 0.166049 \\ 
  Centroid drift & 0.056974 & 0.058126 & 0.057936 \\ 
  Relative Centroid drift & 0.008139 & 0.008304 & 0.008277 \\ 
   \hline
\end{tabular}
\caption{Standard deviation of error rates} 
\label{tbl:stddev_error_rates}
\end{table}

    \caption{Means} 
    \end{subtable}%
    \begin{subtable}{0.5\textwidth}
    % latex table generated in R 3.4.0 by xtable 1.8-2 package
% Sat Aug  5 20:27:22 2017
\begin{table}[ht]
\centering
\begin{tabular}{rrrr}
  \hline
 & Oracle & Silverman & CV \\ 
  \hline
MISE & 0.000036 & 0.000046 & 0.000063 \\ 
  Relative MISE & 0.000340 & 0.000438 & 0.000607 \\ 
  MIAE & 0.000466 & 0.000425 & 0.000563 \\ 
  Relative MIAE & 0.001442 & 0.001314 & 0.001741 \\ 
  Max Error & 0.023200 & 0.030300 & 0.039341 \\ 
  Peak bias & 0.032038 & 0.050895 & 0.055081 \\ 
  Relative Peak bias & 0.099132 & 0.157480 & 0.170430 \\ 
  Peak drift & 0.080767 & 0.118526 & 0.114042 \\ 
  Relative Peak drift & 0.011538 & 0.016932 & 0.016292 \\ 
  Centroid bias & 0.033350 & 0.057284 & 0.053665 \\ 
  Relative Centroid bias & 0.103192 & 0.177250 & 0.166049 \\ 
  Centroid drift & 0.056974 & 0.058126 & 0.057936 \\ 
  Relative Centroid drift & 0.008139 & 0.008304 & 0.008277 \\ 
   \hline
\end{tabular}
\caption{Standard deviation of error rates} 
\label{tbl:stddev_error_rates}
\end{table}

    \caption{Standard deviations} 
    \end{subtable}

\caption{Error rates for uniform population of 10,000, single peak intensity of factor 100 and decay rate 0.5}
\label{tbl:mean_error_rates:unif_100_0.5_1h}
\end{table}


%%
%% Section
\section{Varying the decay of the population density}

\subsection{100 cases from 10,000 with no population decay (uniform)}

See \autoref{tbl:mean_error_rates:unif_100_unif}.

\begin{table}[H]
\centering
\scriptsize

    \begin{subtable}{0.5\textwidth}
    % latex table generated in R 3.4.0 by xtable 1.8-2 package
% Sat Aug  5 20:02:24 2017
\begin{tabular}{lrrr}
  \hline
 & Oracle & Silverman & CV \\ 
  \hline
MISE & 0.000012 & 0.000013 & 0.000013 \\ 
  Relative MISE & 0.117027 & 0.127810 & 0.127803 \\ 
  MIAE & 0.002801 & 0.002922 & 0.002922 \\ 
  Relative MIAE & 0.280127 & 0.292163 & 0.292156 \\ 
  Max Error & 0.008572 & 0.009040 & 0.009040 \\ 
  Peak bias & 0.003361 & 0.005616 & 0.005615 \\ 
  Relative Peak bias & 0.336092 & 0.561583 & 0.561492 \\ 
  Peak drift & 5.163912 & 5.190480 & 5.190393 \\ 
  Relative Peak drift & 0.737702 & 0.741497 & 0.741485 \\ 
  Centroid bias & 0.002819 & 0.003733 & 0.003733 \\ 
  Relative Centroid bias & 0.281890 & 0.373345 & 0.373340 \\ 
  Centroid drift & 5.093476 & 5.086491 & 5.086412 \\ 
  Relative Centroid drift & 0.727639 & 0.726642 & 0.726630 \\ 
   \hline
\end{tabular}

    \caption{Means} 
    \end{subtable}%
    \begin{subtable}{0.5\textwidth}
    % latex table generated in R 3.3.3 by xtable 1.8-2 package
% Sun Mar 11 18:27:52 2018
\begin{tabular}{lrrr}
  \hline
 & Oracle & Silverman & CV \\ 
  \hline
MISE & 0.000003 & 0.000003 & 0.000003 \\ 
  Relative MISE & 0.030194 & 0.032029 & 0.034292 \\ 
  Normalized MISE & 0.000000 & 0.000000 & 0.000000 \\ 
  MIAE & 0.000438 & 0.000418 & 0.000446 \\ 
  Relative MIAE & 0.043781 & 0.041769 & 0.044615 \\ 
  Normalized MIAE & 0.000000 & 0.000000 & 0.000000 \\ 
  Max Error & 0.000581 & 0.000870 & 0.001109 \\ 
  Normalized Max Error & 0.000000 & 0.000000 & 0.000000 \\ 
   \hline
\end{tabular}

    \caption{Standard deviations} 
    \end{subtable}

\caption{Error rates for uniform population of 10,000, single peak intensity of factor 100 and no population decay (uniform)}
\label{tbl:mean_error_rates:unif_100_unif:3}
\end{table}

% \subsection{50 cases from 10,000 with population decay rate 2.0}
% \begin{table}[H]
% \centering
% \scriptsize

%     \begin{subtable}{0.5\textwidth}
%     % latex table generated in R 3.4.2 by xtable 1.8-2 package
% Thu Nov  9 07:46:34 2017
\begin{tabular}{lrrr}
  \hline
 & Oracle & Silverman & CV \\ 
  \hline
MISE & 0.000009 & 0.000014 & 0.000014 \\ 
  Relative MISE & 0.005379 & 0.008735 & 0.008681 \\ 
  MIAE & 0.001692 & 0.002034 & 0.002029 \\ 
  Relative MIAE & 0.041882 & 0.050367 & 0.050233 \\ 
  Max Error & 0.012200 & 0.019064 & 0.018975 \\ 
  Peak bias & -0.006023 & 0.004394 & 0.004294 \\ 
  Relative Peak bias & -0.149126 & 0.108791 & 0.106314 \\ 
  Peak drift & 0.376496 & 0.540355 & 0.539024 \\ 
  Relative Peak drift & 0.053785 & 0.077194 & 0.077003 \\ 
  Centroid bias & -0.006463 & -0.001896 & -0.001922 \\ 
  Relative Centroid bias & -0.160023 & -0.046949 & -0.047585 \\ 
  Centroid drift & 0.262820 & 0.258754 & 0.258222 \\ 
  Relative Centroid drift & 0.037546 & 0.036965 & 0.036889 \\ 
   \hline
\end{tabular}

%     \caption{Means} 
%     \end{subtable}%
%     \begin{subtable}{0.5\textwidth}
%     % latex table generated in R 3.4.2 by xtable 1.8-2 package
% Thu Nov  9 07:46:34 2017
\begin{tabular}{lrrr}
  \hline
 & Oracle & Silverman & CV \\ 
  \hline
MISE & 0.000003 & 0.000004 & 0.000004 \\ 
  Relative MISE & 0.002110 & 0.002745 & 0.002734 \\ 
  MIAE & 0.000327 & 0.000276 & 0.000276 \\ 
  Relative MIAE & 0.008087 & 0.006821 & 0.006828 \\ 
  Max Error & 0.003177 & 0.004987 & 0.004967 \\ 
  Peak bias & 0.004952 & 0.008343 & 0.008316 \\ 
  Relative Peak bias & 0.122614 & 0.206555 & 0.205885 \\ 
  Peak drift & 0.203868 & 0.254285 & 0.254043 \\ 
  Relative Peak drift & 0.029124 & 0.036326 & 0.036292 \\ 
  Centroid bias & 0.005108 & 0.008440 & 0.008398 \\ 
  Relative Centroid bias & 0.126469 & 0.208963 & 0.207921 \\ 
  Centroid drift & 0.145798 & 0.139949 & 0.140716 \\ 
  Relative Centroid drift & 0.020828 & 0.019993 & 0.020102 \\ 
   \hline
\end{tabular}

%     \caption{Standard deviations} 
%     \end{subtable}

% \caption{Error rates for population of 10,000 with decay rate 2.0, single peak intensity of factor 50}
% \label{tbl:mean_error_rates:p2.0_50_1_1h}
% \end{table}

\subsection{100 cases from 10,000 with population decay rate 2.0}
\begin{table}[H]
\centering
\scriptsize

    \begin{subtable}{0.5\textwidth}
    % latex table generated in R 3.4.0 by xtable 1.8-2 package
% Sat Aug  5 21:07:06 2017
\begin{tabular}{lrrr}
  \hline
 & Oracle & Silverman & CV \\ 
  \hline
MISE & 0.000021 & 0.000034 & 0.000034 \\ 
  Relative MISE & 0.003273 & 0.005274 & 0.005179 \\ 
  MIAE & 0.002642 & 0.003221 & 0.003194 \\ 
  Relative MIAE & 0.032700 & 0.039874 & 0.039542 \\ 
  Max Error & 0.020044 & 0.030183 & 0.029789 \\ 
  Peak bias & -0.009647 & 0.006342 & 0.005880 \\ 
  Relative Peak bias & -0.119422 & 0.078502 & 0.072792 \\ 
  Peak drift & 0.309850 & 0.468400 & 0.464955 \\ 
  Relative Peak drift & 0.044264 & 0.066914 & 0.066422 \\ 
  Centroid bias & -0.010650 & -0.004896 & -0.004979 \\ 
  Relative Centroid bias & -0.131838 & -0.060607 & -0.061636 \\ 
  Centroid drift & 0.192663 & 0.192580 & 0.192641 \\ 
  Relative Centroid drift & 0.027523 & 0.027511 & 0.027520 \\ 
   \hline
\end{tabular}

    \caption{Means} 
    \end{subtable}%
    \begin{subtable}{0.5\textwidth}
    % latex table generated in R 3.4.0 by xtable 1.8-2 package
% Sat Aug  5 21:07:07 2017
\begin{table}[ht]
\centering
\begin{tabular}{rrrr}
  \hline
 & Oracle & Silverman & CV \\ 
  \hline
MISE & 0.000008 & 0.000009 & 0.000009 \\ 
  Relative MISE & 0.001252 & 0.001437 & 0.001432 \\ 
  MIAE & 0.000465 & 0.000378 & 0.000382 \\ 
  Relative MIAE & 0.005755 & 0.004684 & 0.004725 \\ 
  Max Error & 0.005228 & 0.006768 & 0.006750 \\ 
  Peak bias & 0.007882 & 0.011659 & 0.011571 \\ 
  Relative Peak bias & 0.097568 & 0.144324 & 0.143244 \\ 
  Peak drift & 0.172228 & 0.204336 & 0.204325 \\ 
  Relative Peak drift & 0.024604 & 0.029191 & 0.029189 \\ 
  Centroid bias & 0.008202 & 0.012807 & 0.012658 \\ 
  Relative Centroid bias & 0.101533 & 0.158534 & 0.156701 \\ 
  Centroid drift & 0.106264 & 0.105423 & 0.105643 \\ 
  Relative Centroid drift & 0.015181 & 0.015060 & 0.015092 \\ 
   \hline
\end{tabular}
\caption{Standard deviation of error rates} 
\label{tbl:stddev_error_rates}
\end{table}

    \caption{Standard deviations} 
    \end{subtable}

\caption{Error rates for population of 10,000 with decay rate 2.0, single peak intensity of factor 100}
\label{tbl:mean_error_rates:p2.0_100_1_1h}
\end{table}

\subsection{100 cases from 10,000 with population decay rate 1.4}
\begin{table}[H]
\centering
\scriptsize

    \begin{subtable}{0.5\textwidth}
    % latex table generated in R 3.4.0 by xtable 1.8-2 package
% Sat Aug  5 21:37:13 2017
\begin{table}[ht]
\centering
\begin{tabular}{lrrr}
  \hline
 & Oracle & Silverman & CV \\ 
  \hline
MISE & 0.000023 & 0.000036 & 0.000034 \\ 
  Relative MISE & 0.003558 & 0.005476 & 0.005188 \\ 
  MIAE & 0.002836 & 0.003304 & 0.003227 \\ 
  Relative MIAE & 0.035109 & 0.040906 & 0.039946 \\ 
  Max Error & 0.019837 & 0.031477 & 0.030030 \\ 
  Peak bias & -0.008783 & 0.006794 & 0.005493 \\ 
  Relative Peak bias & -0.108731 & 0.084104 & 0.067997 \\ 
  Peak drift & 0.279824 & 0.422472 & 0.412451 \\ 
  Relative Peak drift & 0.039975 & 0.060353 & 0.058922 \\ 
  Centroid bias & -0.009989 & -0.004347 & -0.004578 \\ 
  Relative Centroid bias & -0.123654 & -0.053816 & -0.056672 \\ 
  Centroid drift & 0.154247 & 0.149632 & 0.149878 \\ 
  Relative Centroid drift & 0.022035 & 0.021376 & 0.021411 \\ 
   \hline
\end{tabular}
\caption{Mean error rates} 
\label{tbl:mean_error_rates}
\end{table}

    \caption{Means} 
    \end{subtable}%
    \begin{subtable}{0.5\textwidth}
    % latex table generated in R 3.4.0 by xtable 1.8-2 package
% Sat Aug  5 21:37:14 2017
\begin{table}[H]
\centering
\begin{tabular}{lrrr}
  \hline
 & Oracle & Silverman & CV \\ 
  \hline
MISE & 0.000007 & 0.000008 & 0.000008 \\ 
  Relative MISE & 0.001067 & 0.001212 & 0.001249 \\ 
  MIAE & 0.000460 & 0.000347 & 0.000369 \\ 
  Relative MIAE & 0.005697 & 0.004301 & 0.004568 \\ 
  Max Error & 0.003761 & 0.005748 & 0.005796 \\ 
  Peak bias & 0.006662 & 0.009718 & 0.009774 \\ 
  Relative Peak bias & 0.082474 & 0.120295 & 0.120997 \\ 
  Peak drift & 0.152583 & 0.187811 & 0.187107 \\ 
  Relative Peak drift & 0.021798 & 0.026830 & 0.026730 \\ 
  Centroid bias & 0.006939 & 0.011113 & 0.010738 \\ 
  Relative Centroid bias & 0.085896 & 0.137574 & 0.132922 \\ 
  Centroid drift & 0.085880 & 0.085836 & 0.085521 \\ 
  Relative Centroid drift & 0.012269 & 0.012262 & 0.012217 \\ 
   \hline
\end{tabular}
\caption{Standard deviation of error rates} 
\label{tbl:stddev_error_rates}
\end{table}

    \caption{Standard deviations} 
    \end{subtable}

\caption{Error rates for population of 10,000 with decay rate 1.4, single peak intensity of factor 100}
\label{tbl:mean_error_rates:p1.4_100_1_1h}
\end{table}

\subsection{100 cases from 10,000 with population decay rate 1.0}
\begin{table}[H]
\centering
\scriptsize

    \begin{subtable}{0.5\textwidth}
    % latex table generated in R 3.4.0 by xtable 1.8-2 package
% Sun Jul 30 15:44:25 2017
\begin{table}[ht]
\centering
\begin{tabular}{rrrr}
  \hline
 & Oracle & Silverman & CV \\ 
  \hline
MISE & 0.000057 & 0.000070 & 0.000066 \\ 
  Relative MISE & 0.008716 & 0.010683 & 0.010096 \\ 
  MIAE & 0.000835 & 0.000624 & 0.000695 \\ 
  Relative MIAE & 0.010332 & 0.007724 & 0.008599 \\ 
  Max Error & 0.039895 & 0.059402 & 0.054207 \\ 
  Peak bias & 0.028142 & 0.048165 & 0.042352 \\ 
  Relative Peak bias & 0.348370 & 0.596241 & 0.524281 \\ 
  Peak drift & 1.182271 & 1.214951 & 1.199142 \\ 
  Relative Peak drift & 0.168896 & 0.173564 & 0.171306 \\ 
  Centroid bias & 0.007869 & 0.010140 & 0.009436 \\ 
  Relative Centroid bias & 0.097406 & 0.125528 & 0.116811 \\ 
  Centroid drift & 0.218588 & 0.158346 & 0.171699 \\ 
  Relative Centroid drift & 0.031227 & 0.022621 & 0.024528 \\ 
   \hline
\end{tabular}
\caption{Standard deviation of error rates} 
\label{tbl:stddev_error_rates}
\end{table}

    \caption{Means} 
    \end{subtable}%
    \begin{subtable}{0.5\textwidth}
    % latex table generated in R 3.4.0 by xtable 1.8-2 package
% Sat Aug  5 23:00:38 2017
\begin{table}[ht]
\centering
\begin{tabular}{lrrr}
  \hline
 & Oracle & Silverman & CV \\ 
  \hline
MISE & 0.000057 & 0.000070 & 0.000066 \\ 
  Relative MISE & 0.008716 & 0.010683 & 0.010096 \\ 
  MIAE & 0.000835 & 0.000624 & 0.000695 \\ 
  Relative MIAE & 0.010332 & 0.007724 & 0.008599 \\ 
  Max Error & 0.039895 & 0.059402 & 0.054207 \\ 
  Peak bias & 0.028142 & 0.048165 & 0.042352 \\ 
  Relative Peak bias & 0.348370 & 0.596241 & 0.524281 \\ 
  Peak drift & 1.182271 & 1.214951 & 1.199142 \\ 
  Relative Peak drift & 0.168896 & 0.173564 & 0.171306 \\ 
  Centroid bias & 0.007869 & 0.010140 & 0.009436 \\ 
  Relative Centroid bias & 0.097406 & 0.125528 & 0.116811 \\ 
  Centroid drift & 0.218588 & 0.158346 & 0.171699 \\ 
  Relative Centroid drift & 0.031227 & 0.022621 & 0.024528 \\ 
   \hline
\end{tabular}
\caption{Standard deviation of error rates} 
\label{tbl:stddev_error_rates}
\end{table}

    \caption{Standard deviations} 
    \end{subtable}

\caption{Error rates for population of 10,000 with decay rate 1.0, single peak intensity of factor 100}
\label{tbl:mean_error_rates:p1.0_100_1_1h}
\end{table}

\subsection{100 cases from 10,000 with population decay rate 0.7}
\begin{table}[H]
\centering
\scriptsize

    \begin{subtable}{0.5\textwidth}
    % latex table generated in R 3.4.0 by xtable 1.8-2 package
% Sat Aug  5 23:40:16 2017
\begin{table}[H]
\centering
\begin{tabular}{lrrr}
  \hline
 & Oracle & Silverman & CV \\ 
  \hline
MISE & 0.001551 & 0.001488 & 13455866197.588720 \\ 
  Relative MISE & 0.237723 & 0.227961 & 2062001845964.798828 \\ 
  MIAE & 0.006198 & 0.006264 & 516.211152 \\ 
  Relative MIAE & 0.076728 & 0.077543 & 6390.223929 \\ 
  Max Error & 0.401935 & 0.425882 & 148124.498061 \\ 
  Peak bias & 0.328253 & 0.352447 & 0.302431 \\ 
  Relative Peak bias & 4.063469 & 4.362974 & 3.743822 \\ 
  Peak drift & 2.208545 & 2.202533 & 2.260982 \\ 
  Relative Peak drift & 0.315506 & 0.314648 & 0.322997 \\ 
  Centroid bias & -0.008139 & -0.006532 & -0.013600 \\ 
  Relative Centroid bias & -0.100757 & -0.080862 & -0.168350 \\ 
  Centroid drift & 0.325652 & 0.299875 & 0.438476 \\ 
  Relative Centroid drift & 0.046522 & 0.042839 & 0.062639 \\ 
   \hline
\end{tabular}
\caption{Mean error rates} 
\label{tbl:mean_error_rates}
\end{table}

    \caption{Means} 
    \end{subtable}
    \begin{subtable}{0.5\textwidth}
    % latex table generated in R 3.4.0 by xtable 1.8-2 package
% Sat Aug  5 23:40:16 2017
\begin{tabular}{lrrr}
  \hline
 & Oracle & Silverman & CV \\ 
  \hline
MISE & 0.011102 & 0.010329 & 296878170527.312012 \\ 
  Relative MISE & 1.701292 & 1.582834 & 45494160440126.023438 \\ 
  MIAE & 0.002760 & 0.002529 & 10596.705276 \\ 
  Relative MIAE & 0.034164 & 0.031310 & 131177.560609 \\ 
  Max Error & 0.770533 & 0.791358 & 3065119.449329 \\ 
  Peak bias & 0.768882 & 0.789719 & 0.837181 \\ 
  Relative Peak bias & 9.518054 & 9.776008 & 10.363538 \\ 
  Peak drift & 0.491653 & 0.423992 & 0.610823 \\ 
  Relative Peak drift & 0.070236 & 0.060570 & 0.087260 \\ 
  Centroid bias & 0.009546 & 0.010202 & 0.010786 \\ 
  Relative Centroid bias & 0.118169 & 0.126295 & 0.133516 \\ 
  Centroid drift & 0.181917 & 0.166592 & 0.279225 \\ 
  Relative Centroid drift & 0.025988 & 0.023799 & 0.039889 \\ 
   \hline
\end{tabular}

    \caption{Standard deviations} 
    \end{subtable}

\caption{Error rates for population of 10,000 with decay rate 0.7, single peak intensity of factor 100}
\label{tbl:mean_error_rates:p0.7_100_1_1h}
\end{table}


%%
%% Section
\section{Varying the distance between two peaks}

\subsection{100 cases from uniform population of 10,000, 2 hills, 60/40 height ratio, gap of 1}
\begin{table}[H]
\centering
\scriptsize

    \begin{subtable}{0.5\textwidth}
    % latex table generated in R 3.4.2 by xtable 1.8-2 package
% Sat Feb 17 16:39:11 2018
\begin{tabular}{lrrr}
  \hline
 & Oracle & Silverman & CV \\ 
  \hline
MISE & 0.000020 & 0.000028 & 0.000028 \\ 
  Relative MISE & 0.003806 & 0.005434 & 0.005499 \\ 
  Normalized MISE & 0.000020 & 0.000028 & 0.000028 \\ 
  MIAE & 0.002482 & 0.002946 & 0.002911 \\ 
  Relative MIAE & 0.034598 & 0.041069 & 0.040574 \\ 
  Max Error & 0.018613 & 0.025324 & 0.024450 \\ 
  Peak bias & -0.008775 & 0.002958 & -0.002987 \\ 
  Relative Peak bias & -0.122312 & 0.041234 & -0.041636 \\ 
  Peak drift & 0.308419 & 0.424717 & 0.350763 \\ 
  Relative Peak drift & 0.044060 & 0.060674 & 0.050109 \\ 
  Centroid bias & -0.009281 & -0.001295 & -0.007951 \\ 
  Relative Centroid bias & -0.129358 & -0.018055 & -0.110820 \\ 
  Centroid drift & 0.225096 & 0.241028 & 0.222905 \\ 
  Relative Centroid drift & 0.032157 & 0.034433 & 0.031844 \\ 
   \hline
\end{tabular}

    \caption{Means} 
    \end{subtable}%
    \begin{subtable}{0.5\textwidth}
    % latex table generated in R 3.4.2 by xtable 1.8-2 package
% Sat Feb 17 16:39:12 2018
\begin{tabular}{lrrr}
  \hline
 & Oracle & Silverman & CV \\ 
  \hline
MISE & 0.000010 & 0.000010 & 0.000016 \\ 
  Relative MISE & 0.001945 & 0.002011 & 0.003094 \\ 
  Normalized MISE & 0.000010 & 0.000010 & 0.000016 \\ 
  MIAE & 0.000504 & 0.000439 & 0.000652 \\ 
  Relative MIAE & 0.007032 & 0.006119 & 0.009082 \\ 
  Max Error & 0.005700 & 0.006580 & 0.010260 \\ 
  Peak bias & 0.008955 & 0.013103 & 0.016903 \\ 
  Relative Peak bias & 0.124825 & 0.182643 & 0.235609 \\ 
  Peak drift & 0.166058 & 0.224066 & 0.206658 \\ 
  Relative Peak drift & 0.023723 & 0.032009 & 0.029523 \\ 
  Centroid bias & 0.009182 & 0.013554 & 0.013838 \\ 
  Relative Centroid bias & 0.127981 & 0.188920 & 0.192878 \\ 
  Centroid drift & 0.119650 & 0.127521 & 0.120946 \\ 
  Relative Centroid drift & 0.017093 & 0.018217 & 0.017278 \\ 
   \hline
\end{tabular}

    \caption{Standard deviations} 
    \end{subtable}

\caption{Error rates for uniform population of 10,000, factor of 100 with 2 hills, 60/40 height ratio, gap of 1}
\label{tbl:mean_error_rates:unif_100_1_2h_1}
\end{table}

\subsection{100 cases from uniform population of 10,000, 2 hills, 60/40 height ratio, gap of 2}
\begin{table}[H]
\centering
\scriptsize

    \begin{subtable}{0.5\textwidth}
    % latex table generated in R 3.3.3 by xtable 1.8-2 package
% Tue Apr 17 10:55:12 2018
\begin{tabular}{lrrr}
  \toprule
 & Oracle & Silverman & CV \\ 
  \midrule
MISE & 0.000016 & 0.000020 & 0.000022 \\ 
  Relative MISE & 0.005511 & 0.006846 & 0.007379 \\ 
  Normalized MISE & 1.616635 & 2.007979 & 2.164501 \\ 
  MIAE & 0.002448 & 0.002717 & 0.002794 \\ 
  Relative MIAE & 0.045194 & 0.050158 & 0.051596 \\ 
  Normalized MIAE & 0.000024 & 0.000027 & 0.000028 \\ 
  Supremum error & 0.015121 & 0.018267 & 0.018826 \\ 
  Normalized Sup error & 0.000151 & 0.000183 & 0.000188 \\ 
  Peak bias & -0.007114 & -0.000308 & -0.002112 \\ 
  Relative Peak bias & -0.131350 & -0.005688 & -0.038995 \\ 
  Peak drift & 0.435316 & 0.501214 & 0.475627 \\ 
  Relative Peak drift & 0.062188 & 0.071602 & 0.067947 \\ 
  Centroid bias & -0.007333 & -0.001889 & -0.004326 \\ 
  Relative Centroid bias & -0.135394 & -0.034870 & -0.079868 \\ 
  Centroid drift & 0.411449 & 0.405429 & 0.408394 \\ 
  Relative Centroid drift & 0.058778 & 0.057918 & 0.058342 \\ 
   \bottomrule
\end{tabular}

    \caption{Means} 
    \end{subtable}%
    \begin{subtable}{0.5\textwidth}
    % latex table generated in R 3.3.3 by xtable 1.8-2 package
% Sun Mar 11 18:25:42 2018
\begin{tabular}{lrrr}
  \hline
 & Oracle & Silverman & CV \\ 
  \hline
MISE & 0.000008 & 0.000008 & 0.000011 \\ 
  Relative MISE & 0.002755 & 0.002781 & 0.003674 \\ 
  Normalized MISE & 0.000000 & 0.000000 & 0.000000 \\ 
  MIAE & 0.000517 & 0.000473 & 0.000595 \\ 
  Relative MIAE & 0.009538 & 0.008725 & 0.010990 \\ 
  Normalized MIAE & 0.000000 & 0.000000 & 0.000000 \\ 
  Max Error & 0.004458 & 0.004702 & 0.006811 \\ 
  Normalized Max Error & 0.000000 & 0.000000 & 0.000001 \\ 
  Peak bias & 0.006969 & 0.009963 & 0.012617 \\ 
  Relative Peak bias & 0.128684 & 0.183949 & 0.232964 \\ 
  Peak drift & 0.264007 & 0.330445 & 0.305978 \\ 
  Relative Peak drift & 0.037715 & 0.047206 & 0.043711 \\ 
  Centroid bias & 0.007077 & 0.010411 & 0.011572 \\ 
  Relative Centroid bias & 0.130667 & 0.192233 & 0.213671 \\ 
  Centroid drift & 0.228455 & 0.246370 & 0.236918 \\ 
  Relative Centroid drift & 0.032636 & 0.035196 & 0.033845 \\ 
   \hline
\end{tabular}

    \caption{Standard deviations} 
    \end{subtable}

\caption{Error rates for uniform population of 10,000, factor of 100 with 2 hills, 60/40 height ratio, gap of 2}
\label{tbl:mean_error_rates:unif_100_1_2h_2}
\end{table}

\subsection{100 cases from uniform population of 10,000, 2 hills, 60/40 height ratio, gap of 3}
\begin{table}[H]
\centering
\scriptsize

    \begin{subtable}{0.5\textwidth}
    % latex table generated in R 3.4.0 by xtable 1.8-2 package
% Sat Aug  5 21:31:15 2017
\begin{table}[ht]
\centering
\begin{tabular}{lrrr}
  \hline
 & Oracle & Silverman & CV \\ 
  \hline
MISE & 0.000018 & 0.000019 & 0.000019 \\ 
  Relative MISE & 0.007368 & 0.007591 & 0.007591 \\ 
  MIAE & 0.002675 & 0.002735 & 0.002735 \\ 
  Relative MIAE & 0.053776 & 0.054989 & 0.054991 \\ 
  Max Error & 0.016647 & 0.016798 & 0.016799 \\ 
  Peak bias & -0.010884 & -0.010384 & -0.010381 \\ 
  Relative Peak bias & -0.218843 & -0.208792 & -0.208735 \\ 
  Peak drift & 0.550466 & 0.541528 & 0.541476 \\ 
  Relative Peak drift & 0.078638 & 0.077361 & 0.077354 \\ 
  Centroid bias & -0.011999 & -0.011748 & -0.011746 \\ 
  Relative Centroid bias & -0.241255 & -0.236209 & -0.236168 \\ 
  Centroid drift & 0.581652 & 0.562239 & 0.562221 \\ 
  Relative Centroid drift & 0.083093 & 0.080320 & 0.080317 \\ 
   \hline
\end{tabular}
\caption{Mean error rates} 
\label{tbl:mean_error_rates}
\end{table}

    \caption{Means} 
    \end{subtable}%
    \begin{subtable}{0.5\textwidth}
    % latex table generated in R 3.4.3 by xtable 1.8-2 package
% Tue Apr 03 11:55:04 2018
\begin{tabular}{lrrr}
  \hline
 & Oracle & Silverman & CV \\ 
  \hline
MISE & 0.000007 & 0.000008 & 0.000008 \\ 
  Relative MISE & 0.002957 & 0.003127 & 0.003371 \\ 
  Normalized MISE & 0.731484 & 0.773458 & 0.833855 \\ 
  MIAE & 0.000479 & 0.000502 & 0.000519 \\ 
  Relative MIAE & 0.009638 & 0.010103 & 0.010427 \\ 
  Normalized MIAE & 0.000005 & 0.000005 & 0.000005 \\ 
  Max Error & 0.004288 & 0.004405 & 0.004589 \\ 
  Normalized Max Error & 0.000043 & 0.000044 & 0.000046 \\ 
  Peak bias & 0.005729 & 0.006384 & 0.008716 \\ 
  Relative Peak bias & 0.115197 & 0.128359 & 0.175251 \\ 
  Peak drift & 0.601920 & 0.611482 & 0.641342 \\ 
  Relative Peak drift & 0.085989 & 0.087355 & 0.091620 \\ 
  Centroid bias & 0.006743 & 0.007365 & 0.008973 \\ 
  Relative Centroid bias & 0.135585 & 0.148081 & 0.180412 \\ 
  Centroid drift & 0.467306 & 0.465004 & 0.463923 \\ 
  Relative Centroid drift & 0.066758 & 0.066429 & 0.066275 \\ 
   \hline
\end{tabular}

    \caption{Standard deviations} 
    \end{subtable}

\caption{Error rates for uniform population of 10,000, factor of 100 with 2 hills, 60/40 height ratio, gap of 3}
\label{tbl:mean_error_rates:unif_100_1_2h_3}
\end{table}
\subsection{100 cases from uniform population of 10,000, 2 hills, 60/40 height ratio, gap of 4}
\begin{table}[H]
\centering
\scriptsize

    \begin{subtable}{0.5\textwidth}
    % latex table generated in R 3.4.0 by xtable 1.8-2 package
% Sat Aug  5 21:31:21 2017
\begin{tabular}{lrrr}
  \hline
 & Oracle & Silverman & CV \\ 
  \hline
MISE & 0.000020 & 0.000024 & 0.000024 \\ 
  Relative MISE & 0.007580 & 0.009231 & 0.009226 \\ 
  MIAE & 0.002817 & 0.003101 & 0.003100 \\ 
  Relative MIAE & 0.054725 & 0.060231 & 0.060211 \\ 
  Max Error & 0.017437 & 0.019484 & 0.019477 \\ 
  Peak bias & -0.011002 & -0.017317 & -0.017303 \\ 
  Relative Peak bias & -0.213721 & -0.336389 & -0.336120 \\ 
  Peak drift & 0.709475 & 0.472244 & 0.472127 \\ 
  Relative Peak drift & 0.101354 & 0.067463 & 0.067447 \\ 
  Centroid bias & -0.016026 & -0.019387 & -0.019377 \\ 
  Relative Centroid bias & -0.311302 & -0.376586 & -0.376399 \\ 
  Centroid drift & 0.642990 & 0.505662 & 0.505713 \\ 
  Relative Centroid drift & 0.091856 & 0.072237 & 0.072245 \\ 
   \hline
\end{tabular}

    \caption{Means} 
    \end{subtable}%
    \begin{subtable}{0.5\textwidth}
    % latex table generated in R 3.4.0 by xtable 1.8-2 package
% Sat Aug  5 21:31:22 2017
\begin{table}[ht]
\centering
\begin{tabular}{lrrr}
  \hline
 & Oracle & Silverman & CV \\ 
  \hline
MISE & 0.000008 & 0.000009 & 0.000009 \\ 
  Relative MISE & 0.003013 & 0.003440 & 0.003441 \\ 
  MIAE & 0.000502 & 0.000489 & 0.000490 \\ 
  Relative MIAE & 0.009756 & 0.009502 & 0.009511 \\ 
  Max Error & 0.004673 & 0.004611 & 0.004613 \\ 
  Peak bias & 0.006316 & 0.005595 & 0.005602 \\ 
  Relative Peak bias & 0.122696 & 0.108690 & 0.108816 \\ 
  Peak drift & 1.106241 & 0.913011 & 0.913039 \\ 
  Relative Peak drift & 0.158034 & 0.130430 & 0.130434 \\ 
  Centroid bias & 0.011014 & 0.007901 & 0.007910 \\ 
  Relative Centroid bias & 0.213943 & 0.153475 & 0.153658 \\ 
  Centroid drift & 0.688671 & 0.620291 & 0.619967 \\ 
  Relative Centroid drift & 0.098382 & 0.088613 & 0.088567 \\ 
   \hline
\end{tabular}
\caption{Standard deviation of error rates} 
\label{tbl:stddev_error_rates}
\end{table}

    \caption{Standard deviations} 
    \end{subtable}

\caption{Error rates for uniform population of 10,000, factor of 100 with 2 hills, 60/40 height ratio, gap of 4}
\label{tbl:mean_error_rates:unif_100_1_2h_4}
\end{table}

%%
%% Section
\section{Varying the distance between the population and risk function peaks}

\subsection{100 cases from 10,000 with population decay rate 1.4, distance of 0}

See \autoref{tbl:mean_error_rates:unif_200_1_1h}.

\begin{table}[H]
\centering
\scriptsize

    \begin{subtable}{0.5\textwidth}
    % latex table generated in R 3.4.0 by xtable 1.8-2 package
% Sat Aug  5 21:37:13 2017
\begin{table}[ht]
\centering
\begin{tabular}{lrrr}
  \hline
 & Oracle & Silverman & CV \\ 
  \hline
MISE & 0.000023 & 0.000036 & 0.000034 \\ 
  Relative MISE & 0.003558 & 0.005476 & 0.005188 \\ 
  MIAE & 0.002836 & 0.003304 & 0.003227 \\ 
  Relative MIAE & 0.035109 & 0.040906 & 0.039946 \\ 
  Max Error & 0.019837 & 0.031477 & 0.030030 \\ 
  Peak bias & -0.008783 & 0.006794 & 0.005493 \\ 
  Relative Peak bias & -0.108731 & 0.084104 & 0.067997 \\ 
  Peak drift & 0.279824 & 0.422472 & 0.412451 \\ 
  Relative Peak drift & 0.039975 & 0.060353 & 0.058922 \\ 
  Centroid bias & -0.009989 & -0.004347 & -0.004578 \\ 
  Relative Centroid bias & -0.123654 & -0.053816 & -0.056672 \\ 
  Centroid drift & 0.154247 & 0.149632 & 0.149878 \\ 
  Relative Centroid drift & 0.022035 & 0.021376 & 0.021411 \\ 
   \hline
\end{tabular}
\caption{Mean error rates} 
\label{tbl:mean_error_rates}
\end{table}

    \caption{Means} 
    \end{subtable}%
    \begin{subtable}{0.5\textwidth}
    % latex table generated in R 3.4.0 by xtable 1.8-2 package
% Sat Aug  5 21:37:14 2017
\begin{table}[H]
\centering
\begin{tabular}{lrrr}
  \hline
 & Oracle & Silverman & CV \\ 
  \hline
MISE & 0.000007 & 0.000008 & 0.000008 \\ 
  Relative MISE & 0.001067 & 0.001212 & 0.001249 \\ 
  MIAE & 0.000460 & 0.000347 & 0.000369 \\ 
  Relative MIAE & 0.005697 & 0.004301 & 0.004568 \\ 
  Max Error & 0.003761 & 0.005748 & 0.005796 \\ 
  Peak bias & 0.006662 & 0.009718 & 0.009774 \\ 
  Relative Peak bias & 0.082474 & 0.120295 & 0.120997 \\ 
  Peak drift & 0.152583 & 0.187811 & 0.187107 \\ 
  Relative Peak drift & 0.021798 & 0.026830 & 0.026730 \\ 
  Centroid bias & 0.006939 & 0.011113 & 0.010738 \\ 
  Relative Centroid bias & 0.085896 & 0.137574 & 0.132922 \\ 
  Centroid drift & 0.085880 & 0.085836 & 0.085521 \\ 
  Relative Centroid drift & 0.012269 & 0.012262 & 0.012217 \\ 
   \hline
\end{tabular}
\caption{Standard deviation of error rates} 
\label{tbl:stddev_error_rates}
\end{table}

    \caption{Standard deviations} 
    \end{subtable}

\caption{Error rates for uniform population of 10,000, single peak intensity of factor 100 and decay rate 1.4, distance between population peak and risk peak is 0}
\label{tbl:mean_error_rates:p1.4_100_1_1h:2}
\end{table}

\subsection{100 cases from 10,000 with population decay rate 1.4, distance of 1}
\begin{table}[H]
\centering
\scriptsize

    \begin{subtable}{0.5\textwidth}
    % latex table generated in R 3.4.3 by xtable 1.8-2 package
% Sun Apr 15 08:37:43 2018
\begin{tabular}{lrrr}
  \toprule
 & Oracle & Silverman & CV \\ 
  \midrule
MISE & 0.000040 & 0.000059 & 0.000053 \\ 
  Relative MISE & 0.006157 & 0.008980 & 0.008084 \\ 
  Normalized MISE & 4.019238 & 5.861684 & 5.277137 \\ 
  MIAE & 0.003333 & 0.003782 & 0.003692 \\ 
  Relative MIAE & 0.041247 & 0.046805 & 0.045691 \\ 
  Normalized MIAE & 0.000033 & 0.000038 & 0.000037 \\ 
  Supremum error & 0.032658 & 0.052441 & 0.042040 \\ 
  Normalized Sup error & 0.000327 & 0.000524 & 0.000420 \\ 
  Peak bias & -0.010635 & 0.006640 & -0.002388 \\ 
  Relative Peak bias & -0.131630 & 0.082180 & -0.029551 \\ 
  Peak drift & 0.503816 & 0.753839 & 0.596590 \\ 
  Relative Peak drift & 0.071974 & 0.107691 & 0.085227 \\ 
  Centroid bias & -0.013864 & -0.009846 & -0.013099 \\ 
  Relative Centroid bias & -0.171596 & -0.121868 & -0.162127 \\ 
  Centroid drift & 0.228592 & 0.242398 & 0.247296 \\ 
  Relative Centroid drift & 0.032656 & 0.034628 & 0.035328 \\ 
   \bottomrule
\end{tabular}

    \caption{Means} 
    \end{subtable}%
    \begin{subtable}{0.5\textwidth}
    % latex table generated in R 3.4.3 by xtable 1.8-2 package
% Sun Apr 15 08:37:43 2018
\begin{tabular}{lrrr}
  \toprule
 & Oracle & Silverman & CV \\ 
  \midrule
MISE & 0.000020 & 0.000028 & 0.000032 \\ 
  Relative MISE & 0.003104 & 0.004242 & 0.004895 \\ 
  Normalized MISE & 2.026098 & 2.769325 & 3.195379 \\ 
  MIAE & 0.000674 & 0.000529 & 0.000780 \\ 
  Relative MIAE & 0.008337 & 0.006543 & 0.009657 \\ 
  Normalized MIAE & 0.000007 & 0.000005 & 0.000008 \\ 
  Supremum error & 0.014262 & 0.025486 & 0.025950 \\ 
  Normalized Sup error & 0.000143 & 0.000255 & 0.000260 \\ 
  Peak bias & 0.007732 & 0.016208 & 0.020017 \\ 
  Relative Peak bias & 0.095694 & 0.200609 & 0.247752 \\ 
  Peak drift & 0.591447 & 0.774400 & 0.669252 \\ 
  Relative Peak drift & 0.084492 & 0.110629 & 0.095607 \\ 
  Centroid bias & 0.007667 & 0.011869 & 0.010926 \\ 
  Relative Centroid bias & 0.094897 & 0.146900 & 0.135229 \\ 
  Centroid drift & 0.153294 & 0.160808 & 0.161309 \\ 
  Relative Centroid drift & 0.021899 & 0.022973 & 0.023044 \\ 
   \bottomrule
\end{tabular}

    \caption{Standard deviations} 
    \end{subtable}

\caption{Error rates for uniform population of 10,000, single peak intensity of factor 100 and decay rate 1.4, distance between population peak and risk peak is 1}
\label{tbl:mean_error_rates:p1.4_100_1_1h_1s}
\end{table}

\subsection{100 cases from 10,000 with population decay rate 1.4, distance of 2}
\begin{table}[H]
\centering
\scriptsize

    \begin{subtable}{0.5\textwidth}
    % latex table generated in R 3.4.3 by xtable 1.8-2 package
% Sun Apr 15 08:38:29 2018
\begin{tabular}{lrrr}
  \toprule
 & Oracle & Silverman & CV \\ 
  \midrule
MISE & 0.000128 & 0.000186 & 0.000171 \\ 
  Relative MISE & 0.019452 & 0.028262 & 0.026077 \\ 
  Normalized MISE & 12.791084 & 18.584174 & 17.147081 \\ 
  MIAE & 0.004923 & 0.005142 & 0.005273 \\ 
  Relative MIAE & 0.060708 & 0.063408 & 0.065032 \\ 
  Normalized MIAE & 0.000049 & 0.000051 & 0.000053 \\ 
  Supremum error & 0.064541 & 0.110936 & 0.089923 \\ 
  Normalized Sup error & 0.000645 & 0.001109 & 0.000899 \\ 
  Peak bias & 0.007356 & 0.054198 & 0.031429 \\ 
  Relative Peak bias & 0.090715 & 0.668364 & 0.387582 \\ 
  Peak drift & 1.231593 & 1.481287 & 1.435563 \\ 
  Relative Peak drift & 0.175942 & 0.211612 & 0.205080 \\ 
  Centroid bias & -0.015659 & -0.013082 & -0.013226 \\ 
  Relative Centroid bias & -0.193105 & -0.161322 & -0.163101 \\ 
  Centroid drift & 0.681121 & 0.587086 & 0.709678 \\ 
  Relative Centroid drift & 0.097303 & 0.083869 & 0.101383 \\ 
   \bottomrule
\end{tabular}

    \caption{Means} 
    \end{subtable}%
    \begin{subtable}{0.5\textwidth}
    % latex table generated in R 3.3.3 by xtable 1.8-2 package
% Mon Mar 19 11:18:35 2018
\begin{tabular}{lrrr}
  \hline
 & Oracle & Silverman & CV \\ 
  \hline
MISE & 0.000149 & 0.000228 & 0.000220 \\ 
  Relative MISE & 0.022731 & 0.034747 & 0.033399 \\ 
  Normalized MISE & 14.946905 & 22.848156 & 21.961772 \\ 
  MIAE & 0.001460 & 0.001193 & 0.001518 \\ 
  Relative MIAE & 0.018006 & 0.014709 & 0.018719 \\ 
  Normalized MIAE & 0.000015 & 0.000012 & 0.000015 \\ 
  Max Error & 0.049284 & 0.089215 & 0.082627 \\ 
  Normalized Max Error & 0.000493 & 0.000892 & 0.000826 \\ 
  Peak bias & 0.040724 & 0.079859 & 0.075417 \\ 
  Relative Peak bias & 0.502201 & 0.984812 & 0.930040 \\ 
  Peak drift & 0.920127 & 0.795791 & 0.894325 \\ 
  Relative Peak drift & 0.131447 & 0.113684 & 0.127761 \\ 
  Centroid bias & 0.015710 & 0.021420 & 0.020204 \\ 
  Relative Centroid bias & 0.193734 & 0.264149 & 0.249158 \\ 
  Centroid drift & 0.425282 & 0.325892 & 0.403521 \\ 
  Relative Centroid drift & 0.060755 & 0.046556 & 0.057646 \\ 
   \hline
\end{tabular}

    \caption{Standard deviations} 
    \end{subtable}

\caption{Error rates for uniform population of 10,000, single peak intensity of factor 100 and decay rate 1.4, distance between population peak and risk peak is 2}
\label{tbl:mean_error_rates:p1.4_100_1_1h_2s}
\end{table}

\subsection{100 cases from 10,000 with population decay rate 1.4, distance of 3}
\begin{table}[H]
\centering
\scriptsize

    \begin{subtable}{0.5\textwidth}
    % latex table generated in R 3.4.2 by xtable 1.8-2 package
% Sat Feb 17 16:36:39 2018
\begin{tabular}{lrrr}
  \hline
 & Oracle & Silverman & CV \\ 
  \hline
MISE & 0.000462 & 0.000665 & 0.000746 \\ 
  Relative MISE & 0.068257 & 0.098273 & 0.110132 \\ 
  Normalized MISE & 0.000462 & 0.000665 & 0.000746 \\ 
  MIAE & 0.007742 & 0.007766 & 0.008587 \\ 
  Relative MIAE & 0.094092 & 0.094377 & 0.104359 \\ 
  Max Error & 0.122605 & 0.206161 & 0.193819 \\ 
  Peak bias & 0.062892 & 0.153943 & 0.135246 \\ 
  Relative Peak bias & 0.764344 & 1.870910 & 1.643687 \\ 
  Peak drift & 1.408035 & 1.345539 & 1.514687 \\ 
  Relative Peak drift & 0.201148 & 0.192220 & 0.216384 \\ 
  Centroid bias & 0.010182 & 0.009211 & 0.022703 \\ 
  Relative Centroid bias & 0.123744 & 0.111939 & 0.275912 \\ 
  Centroid drift & 0.860888 & 0.660993 & 0.913689 \\ 
  Relative Centroid drift & 0.122984 & 0.094428 & 0.130527 \\ 
   \hline
\end{tabular}

    \caption{Means} 
    \end{subtable}%
    \begin{subtable}{0.5\textwidth}
    % latex table generated in R 3.3.3 by xtable 1.8-2 package
% Sun Apr  8 14:00:34 2018
\begin{tabular}{lrrr}
  \toprule
 & Oracle & Silverman & CV \\ 
  \midrule
MISE & 0.000654 & 0.000885 & 0.000987 \\ 
  Relative MISE & 0.096613 & 0.130782 & 0.145836 \\ 
  Normalized MISE & 65.410827 & 88.544119 & 98.736309 \\ 
  MIAE & 0.003100 & 0.002769 & 0.003825 \\ 
  Relative MIAE & 0.037678 & 0.033658 & 0.046481 \\ 
  Normalized MIAE & 0.000031 & 0.000028 & 0.000038 \\ 
  Supremum error & 0.098086 & 0.162229 & 0.170279 \\ 
  Normalized Sup error & 0.000981 & 0.001622 & 0.001703 \\ 
  Peak bias & 0.090873 & 0.150966 & 0.162138 \\ 
  Relative Peak bias & 1.104407 & 1.834730 & 1.970512 \\ 
  Peak drift & 0.843872 & 0.685766 & 0.753360 \\ 
  Relative Peak drift & 0.120553 & 0.097967 & 0.107623 \\ 
  Centroid bias & 0.052001 & 0.051078 & 0.058121 \\ 
  Relative Centroid bias & 0.631982 & 0.620763 & 0.706356 \\ 
  Centroid drift & 0.493068 & 0.370142 & 0.475213 \\ 
  Relative Centroid drift & 0.070438 & 0.052877 & 0.067888 \\ 
   \bottomrule
\end{tabular}

    \caption{Standard deviations} 
    \end{subtable}

\caption{Error rates for uniform population of 10,000, single peak intensity of factor 100 and decay rate 1.4, distance between population peak and risk peak is 3}
\label{tbl:mean_error_rates:p1.4_100_1_1h_3s}
\end{table}

\subsection{100 cases from 10,000 with population decay rate 1.4, distance of 4}
\begin{table}[H]
\centering
\scriptsize

    \begin{subtable}{0.5\textwidth}
    % latex table generated in R 3.4.2 by xtable 1.8-2 package
% Sat Feb 17 16:37:01 2018
\begin{tabular}{lrrr}
  \hline
 & Oracle & Silverman & CV \\ 
  \hline
MISE & 0.001774 & 0.002091 & 0.003161 \\ 
  Relative MISE & 0.240960 & 0.284080 & 0.429375 \\ 
  Normalized MISE & 0.001774 & 0.002091 & 0.003161 \\ 
  MIAE & 0.011174 & 0.011352 & 0.014069 \\ 
  Relative MIAE & 0.130234 & 0.132302 & 0.163978 \\ 
  Max Error & 0.210715 & 0.325465 & 0.379102 \\ 
  Peak bias & 0.144665 & 0.280264 & 0.322637 \\ 
  Relative Peak bias & 1.686072 & 3.266464 & 3.760320 \\ 
  Peak drift & 1.458091 & 1.115466 & 1.362937 \\ 
  Relative Peak drift & 0.208299 & 0.159352 & 0.194705 \\ 
  Centroid bias & 0.065153 & 0.069904 & 0.108514 \\ 
  Relative Centroid bias & 0.759357 & 0.814731 & 1.264726 \\ 
  Centroid drift & 0.902143 & 0.662575 & 0.876723 \\ 
  Relative Centroid drift & 0.128878 & 0.094654 & 0.125246 \\ 
   \hline
\end{tabular}

    \caption{Means} 
    \end{subtable}%
    \begin{subtable}{0.5\textwidth}
    % latex table generated in R 3.3.3 by xtable 1.8-2 package
% Mon Mar 19 11:18:46 2018
\begin{tabular}{lrrr}
  \hline
 & Oracle & Silverman & CV \\ 
  \hline
MISE & 0.006326 & 0.004932 & 0.006532 \\ 
  Relative MISE & 0.859258 & 0.669959 & 0.887313 \\ 
  Normalized MISE & 632.558660 & 493.202592 & 653.212287 \\ 
  MIAE & 0.006350 & 0.005399 & 0.008434 \\ 
  Relative MIAE & 0.074011 & 0.062929 & 0.098298 \\ 
  Normalized MIAE & 0.000064 & 0.000054 & 0.000084 \\ 
  Max Error & 0.332694 & 0.388475 & 0.446557 \\ 
  Normalized Max Error & 0.003327 & 0.003885 & 0.004466 \\ 
  Peak bias & 0.332958 & 0.384721 & 0.444374 \\ 
  Relative Peak bias & 3.880619 & 4.483908 & 5.179166 \\ 
  Peak drift & 0.831842 & 0.527184 & 0.608822 \\ 
  Relative Peak drift & 0.118835 & 0.075312 & 0.086975 \\ 
  Centroid bias & 0.122374 & 0.137106 & 0.153321 \\ 
  Relative Centroid bias & 1.426264 & 1.597966 & 1.786947 \\ 
  Centroid drift & 0.472368 & 0.317499 & 0.443602 \\ 
  Relative Centroid drift & 0.067481 & 0.045357 & 0.063372 \\ 
   \hline
\end{tabular}

    \caption{Standard deviations} 
    \end{subtable}

\caption{Error rates for uniform population of 10,000, single peak intensity of factor 100 and decay rate 1.4, distance between population peak and risk peak is 4}
\label{tbl:mean_error_rates:p1.4_100_1_1h_4s}
\end{table}

