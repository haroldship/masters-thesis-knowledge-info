% !TEX root = appendix_tables_only.tex

%%
%%
%% results tables appendix
%%
%%

%%
%% Section
\section{Uniform risk on a uniform population}

\begin{table}[H]
\centering
\scriptsize

    \begin{subtable}{0.5\textwidth}
    % latex table generated in R 3.3.3 by xtable 1.8-2 package
% Sun Mar 11 18:27:52 2018
\begin{tabular}{lrrr}
  \hline
 & Oracle & Silverman & CV \\ 
  \hline
MISE & 0.000012 & 0.000013 & 0.000013 \\ 
  Relative MISE & 0.116112 & 0.125979 & 0.128308 \\ 
  Normalized MISE & 0.000000 & 0.000000 & 0.000000 \\ 
  MIAE & 0.002786 & 0.002898 & 0.002938 \\ 
  Relative MIAE & 0.278620 & 0.289846 & 0.293819 \\ 
  Normalized MIAE & 0.000000 & 0.000000 & 0.000000 \\ 
  Max Error & 0.008551 & 0.009000 & 0.008926 \\ 
  Normalized Max Error & 0.000001 & 0.000001 & 0.000001 \\ 
   \hline
\end{tabular}

    \caption{Means} 
    \end{subtable}%
    \begin{subtable}{0.5\textwidth}
    % latex table generated in R 3.4.3 by xtable 1.8-2 package
% Tue Apr 03 12:03:44 2018
\begin{tabular}{lrrr}
  \hline
 & Oracle & Silverman & CV \\ 
  \hline
MISE & 0.000003 & 0.000003 & 0.000003 \\ 
  Relative MISE & 0.030194 & 0.032029 & 0.034292 \\ 
  Normalized MISE & 0.301940 & 0.320288 & 0.342922 \\ 
  MIAE & 0.000438 & 0.000418 & 0.000446 \\ 
  Relative MIAE & 0.043781 & 0.041769 & 0.044615 \\ 
  Normalized MIAE & 0.000004 & 0.000004 & 0.000004 \\ 
  Max Error & 0.000581 & 0.000870 & 0.001109 \\ 
  Normalized Max Error & 0.000006 & 0.000009 & 0.000011 \\ 
  Peak bias & 0.001757 & 0.002365 & 0.002891 \\ 
  Relative Peak bias & 0.175734 & 0.236474 & 0.289140 \\ 
  Peak drift & 1.467764 & 1.600954 & 1.558562 \\ 
  Relative Peak drift & 0.209681 & 0.228708 & 0.222652 \\ 
  Centroid bias & 0.002315 & 0.003475 & 0.003480 \\ 
  Relative Centroid bias & 0.231491 & 0.347454 & 0.347951 \\ 
  Centroid drift & 1.204587 & 1.197905 & 1.181971 \\ 
  Relative Centroid drift & 0.172084 & 0.171129 & 0.168853 \\ 
   \hline
\end{tabular}

    \caption{Standard deviations} 
    \end{subtable}

\caption{Error rates for uniform population of 10,000, uniform intensity of factor 100}
\label{tbl:mean_error_rates:unif_100_unif}
\end{table}


%%
%% Section
\section{Varying the number of cases for fixed population of 10,000}

\subsection{50 cases}
\begin{table}[H]
\centering
\scriptsize

    \begin{subtable}{0.5\textwidth}
    % latex table generated in R 3.4.0 by xtable 1.8-2 package
% Sat Aug  5 19:49:51 2017
\begin{table}[H]
\centering
\begin{tabular}{lrrr}
  \hline
 & Oracle & Silverman & CV \\ 
  \hline
MISE & 0.000008 & 0.000014 & 0.000014 \\ 
  Relative MISE & 0.005208 & 0.008600 & 0.008580 \\ 
  MIAE & 0.001568 & 0.001955 & 0.001953 \\ 
  Relative MIAE & 0.038833 & 0.048413 & 0.048363 \\ 
  Max Error & 0.012435 & 0.018787 & 0.018756 \\ 
  Peak bias & -0.007124 & 0.004056 & 0.004020 \\ 
  Relative Peak bias & -0.176378 & 0.100410 & 0.099522 \\ 
  Peak drift & 0.322404 & 0.476198 & 0.475222 \\ 
  Relative Peak drift & 0.046058 & 0.068028 & 0.067889 \\ 
  Centroid bias & -0.007304 & 0.000496 & 0.000487 \\ 
  Relative Centroid bias & -0.180839 & 0.012284 & 0.012057 \\ 
  Centroid drift & 0.262196 & 0.280772 & 0.281105 \\ 
  Relative Centroid drift & 0.037457 & 0.040110 & 0.040158 \\ 
   \hline
\end{tabular}
\caption{Mean error rates} 
\label{tbl:mean_error_rates}
\end{table}

    \caption{Means} 
    \end{subtable}%
    \begin{subtable}{0.5\textwidth}
    % latex table generated in R 3.4.0 by xtable 1.8-2 package
% Sat Aug  5 19:49:52 2017
\begin{table}[ht]
\centering
\begin{tabular}{rrrr}
  \hline
 & Oracle & Silverman & CV \\ 
  \hline
MISE & 0.000005 & 0.000006 & 0.000006 \\ 
  Relative MISE & 0.002759 & 0.003630 & 0.003613 \\ 
  MIAE & 0.000349 & 0.000321 & 0.000320 \\ 
  Relative MIAE & 0.008640 & 0.007935 & 0.007920 \\ 
  Max Error & 0.004095 & 0.005908 & 0.005887 \\ 
  Peak bias & 0.006126 & 0.010857 & 0.010834 \\ 
  Relative Peak bias & 0.151664 & 0.268788 & 0.268237 \\ 
  Peak drift & 0.179354 & 0.252110 & 0.252185 \\ 
  Relative Peak drift & 0.025622 & 0.036016 & 0.036026 \\ 
  Centroid bias & 0.006226 & 0.011137 & 0.011125 \\ 
  Relative Centroid bias & 0.154153 & 0.275729 & 0.275424 \\ 
  Centroid drift & 0.138102 & 0.145182 & 0.145503 \\ 
  Relative Centroid drift & 0.019729 & 0.020740 & 0.020786 \\ 
   \hline
\end{tabular}
\caption{Standard deviation of error rates} 
\label{tbl:stddev_error_rates}
\end{table}

    \caption{Standard deviations} 
    \end{subtable}

\caption{Error rates for uniform population of 10,000, single peak intensity of factor 50}
\label{tbl:mean_error_rates:unif_50_1_1h}
\end{table}

\subsection{100 cases}
\begin{table}[H]
\centering
\scriptsize

    \begin{subtable}{0.5\textwidth}
    % latex table generated in R 3.4.0 by xtable 1.8-2 package
% Sat Aug  5 21:15:22 2017
\begin{tabular}{lrrr}
  \hline
 & Oracle & Silverman & CV \\ 
  \hline
MISE & 0.000022 & 0.000035 & 0.000034 \\ 
  Relative MISE & 0.003358 & 0.005293 & 0.005266 \\ 
  MIAE & 0.002505 & 0.003100 & 0.003092 \\ 
  Relative MIAE & 0.031010 & 0.038378 & 0.038279 \\ 
  Max Error & 0.020705 & 0.030617 & 0.030498 \\ 
  Peak bias & -0.012292 & 0.006167 & 0.006026 \\ 
  Relative Peak bias & -0.152166 & 0.076337 & 0.074592 \\ 
  Peak drift & 0.265067 & 0.409888 & 0.408051 \\ 
  Relative Peak drift & 0.037867 & 0.058555 & 0.058293 \\ 
  Centroid bias & -0.012639 & -0.000422 & -0.000492 \\ 
  Relative Centroid bias & -0.156464 & -0.005226 & -0.006090 \\ 
  Centroid drift & 0.199485 & 0.216521 & 0.216495 \\ 
  Relative Centroid drift & 0.028498 & 0.030932 & 0.030928 \\ 
   \hline
\end{tabular}

    \caption{Means} 
    \end{subtable}%
    \begin{subtable}{0.5\textwidth}
    % latex table generated in R 3.4.0 by xtable 1.8-2 package
% Sat Aug  5 21:15:23 2017
\begin{tabular}{lrrr}
  \hline
 & Oracle & Silverman & CV \\ 
  \hline
MISE & 0.000012 & 0.000013 & 0.000013 \\ 
  Relative MISE & 0.001764 & 0.001964 & 0.001962 \\ 
  MIAE & 0.000525 & 0.000440 & 0.000441 \\ 
  Relative MIAE & 0.006496 & 0.005449 & 0.005460 \\ 
  Max Error & 0.006748 & 0.008658 & 0.008654 \\ 
  Peak bias & 0.009900 & 0.016313 & 0.016285 \\ 
  Relative Peak bias & 0.122549 & 0.201943 & 0.201595 \\ 
  Peak drift & 0.140602 & 0.212548 & 0.211161 \\ 
  Relative Peak drift & 0.020086 & 0.030364 & 0.030166 \\ 
  Centroid bias & 0.010031 & 0.016949 & 0.016909 \\ 
  Relative Centroid bias & 0.124173 & 0.209809 & 0.209321 \\ 
  Centroid drift & 0.106052 & 0.113541 & 0.114205 \\ 
  Relative Centroid drift & 0.015150 & 0.016220 & 0.016315 \\ 
   \hline
\end{tabular}

    \caption{Standard deviations} 
    \end{subtable}

\caption{Error rates for uniform population of 10,000, single peak intensity of factor 100}
\label{tbl:mean_error_rates:unif_100_1_1h}
\end{table}

\subsection{200 cases}
\begin{table}[H]
\centering
\scriptsize

    \begin{subtable}{0.5\textwidth}
    % latex table generated in R 3.4.0 by xtable 1.8-2 package
% Sat Aug  5 21:44:37 2017
\begin{tabular}{lrrr}
  \hline
 & Oracle & Silverman & CV \\ 
  \hline
MISE & 0.000053 & 0.000083 & 0.000082 \\ 
  Relative MISE & 0.002029 & 0.003184 & 0.003129 \\ 
  MIAE & 0.003909 & 0.004852 & 0.004810 \\ 
  Relative MIAE & 0.024196 & 0.030032 & 0.029772 \\ 
  Max Error & 0.033936 & 0.049447 & 0.048810 \\ 
  Peak bias & -0.016473 & 0.010215 & 0.009458 \\ 
  Relative Peak bias & -0.101961 & 0.063224 & 0.058539 \\ 
  Peak drift & 0.219372 & 0.357137 & 0.353058 \\ 
  Relative Peak drift & 0.031339 & 0.051020 & 0.050437 \\ 
  Centroid bias & -0.017532 & -0.001587 & -0.001859 \\ 
  Relative Centroid bias & -0.108518 & -0.009821 & -0.011508 \\ 
  Centroid drift & 0.146327 & 0.158930 & 0.158287 \\ 
  Relative Centroid drift & 0.020904 & 0.022704 & 0.022612 \\ 
   \hline
\end{tabular}

    \caption{Means} 
    \end{subtable}%
    \begin{subtable}{0.5\textwidth}
    % latex table generated in R 3.4.0 by xtable 1.8-2 package
% Sat Aug  5 21:44:37 2017
\begin{table}[H]
\centering
\begin{tabular}{lrrr}
  \hline
 & Oracle & Silverman & CV \\ 
  \hline
MISE & 0.000025 & 0.000026 & 0.000026 \\ 
  Relative MISE & 0.000965 & 0.000979 & 0.000979 \\ 
  MIAE & 0.000739 & 0.000595 & 0.000603 \\ 
  Relative MIAE & 0.004575 & 0.003682 & 0.003729 \\ 
  Max Error & 0.010130 & 0.012017 & 0.011950 \\ 
  Peak bias & 0.015575 & 0.023178 & 0.023003 \\ 
  Relative Peak bias & 0.096400 & 0.143460 & 0.142376 \\ 
  Peak drift & 0.121164 & 0.182007 & 0.180423 \\ 
  Relative Peak drift & 0.017309 & 0.026001 & 0.025775 \\ 
  Centroid bias & 0.015861 & 0.024553 & 0.024299 \\ 
  Relative Centroid bias & 0.098172 & 0.151969 & 0.150398 \\ 
  Centroid drift & 0.084649 & 0.089605 & 0.089907 \\ 
  Relative Centroid drift & 0.012093 & 0.012801 & 0.012844 \\ 
   \hline
\end{tabular}
\caption{Standard deviation of error rates} 
\label{tbl:stddev_error_rates}
\end{table}

    \caption{Standard deviations} 
    \end{subtable}

\caption{Error rates for uniform population of 10,000, single peak intensity of factor 200}
\label{tbl:mean_error_rates:unif_200_1_1h}
\end{table}

\subsection{500 cases}
\begin{table}[H]
\centering
\scriptsize

    \begin{subtable}{0.5\textwidth}
    % latex table generated in R 3.4.0 by xtable 1.8-2 package
% Sat Aug  5 22:43:32 2017
\begin{tabular}{lrrr}
  \hline
 & Oracle & Silverman & CV \\ 
  \hline
MISE & 0.000170 & 0.000265 & 0.000244 \\ 
  Relative MISE & 0.001041 & 0.001626 & 0.001496 \\ 
  MIAE & 0.007096 & 0.008775 & 0.008421 \\ 
  Relative MIAE & 0.017567 & 0.021725 & 0.020849 \\ 
  Max Error & 0.065158 & 0.092681 & 0.087429 \\ 
  Peak bias & -0.023250 & 0.018688 & 0.012450 \\ 
  Relative Peak bias & -0.057562 & 0.046269 & 0.030824 \\ 
  Peak drift & 0.176919 & 0.293212 & 0.276800 \\ 
  Relative Peak drift & 0.025274 & 0.041887 & 0.039543 \\ 
  Centroid bias & -0.026539 & -0.003605 & -0.005705 \\ 
  Relative Centroid bias & -0.065707 & -0.008925 & -0.014124 \\ 
  Centroid drift & 0.096196 & 0.104184 & 0.103456 \\ 
  Relative Centroid drift & 0.013742 & 0.014883 & 0.014779 \\ 
   \hline
\end{tabular}

    \caption{Means} 
    \end{subtable}%
    \begin{subtable}{0.5\textwidth}
    % latex table generated in R 3.4.0 by xtable 1.8-2 package
% Sat Aug  5 22:43:32 2017
\begin{table}[H]
\centering
\begin{tabular}{lrrr}
  \hline
 & Oracle & Silverman & CV \\ 
  \hline
MISE & 0.000065 & 0.000066 & 0.000068 \\ 
  Relative MISE & 0.000401 & 0.000404 & 0.000417 \\ 
  MIAE & 0.001108 & 0.000892 & 0.000963 \\ 
  Relative MIAE & 0.002744 & 0.002209 & 0.002384 \\ 
  Max Error & 0.016494 & 0.018880 & 0.019274 \\ 
  Peak bias & 0.025669 & 0.034254 & 0.033769 \\ 
  Relative Peak bias & 0.063552 & 0.084807 & 0.083606 \\ 
  Peak drift & 0.096380 & 0.147575 & 0.143744 \\ 
  Relative Peak drift & 0.013769 & 0.021082 & 0.020535 \\ 
  Centroid bias & 0.026146 & 0.037366 & 0.035905 \\ 
  Relative Centroid bias & 0.064733 & 0.092512 & 0.088895 \\ 
  Centroid drift & 0.061402 & 0.064030 & 0.063331 \\ 
  Relative Centroid drift & 0.008772 & 0.009147 & 0.009047 \\ 
   \hline
\end{tabular}
\caption{Standard deviation of error rates} 
\label{tbl:stddev_error_rates}
\end{table}

    \caption{Standard deviations} 
    \end{subtable}

\caption{Error rates for uniform population of 10,000, single peak intensity of factor 500}
\label{tbl:mean_error_rates:unif_500_1_1h}
\end{table}

\subsection{1000 cases}
\begin{table}[H]
\centering
\scriptsize

    \begin{subtable}{0.5\textwidth}
    % latex table generated in R 3.4.0 by xtable 1.8-2 package
% Sun Aug  6 00:15:05 2017
\begin{table}[H]
\centering
\begin{tabular}{lrrr}
  \hline
 & Oracle & Silverman & CV \\ 
  \hline
MISE & 0.000379 & 0.000619 & 0.000484 \\ 
  Relative MISE & 0.000581 & 0.000948 & 0.000742 \\ 
  MIAE & 0.010804 & 0.013570 & 0.012061 \\ 
  Relative MIAE & 0.013375 & 0.016798 & 0.014931 \\ 
  Max Error & 0.105433 & 0.145677 & 0.126060 \\ 
  Peak bias & -0.034129 & 0.024518 & -0.000092 \\ 
  Relative Peak bias & -0.042248 & 0.030351 & -0.000114 \\ 
  Peak drift & 0.129044 & 0.251876 & 0.185739 \\ 
  Relative Peak drift & 0.018435 & 0.035982 & 0.026534 \\ 
  Centroid bias & -0.037516 & -0.001969 & -0.014152 \\ 
  Relative Centroid bias & -0.046441 & -0.002438 & -0.017519 \\ 
  Centroid drift & 0.068889 & 0.076451 & 0.074548 \\ 
  Relative Centroid drift & 0.009841 & 0.010922 & 0.010650 \\ 
   \hline
\end{tabular}
\caption{Mean error rates} 
\label{tbl:mean_error_rates}
\end{table}

    \caption{Means} 
    \end{subtable}%
    \begin{subtable}{0.5\textwidth}
    % latex table generated in R 3.4.0 by xtable 1.8-2 package
% Sun Aug  6 00:15:05 2017
\begin{table}[H]
\centering
\begin{tabular}{lrrr}
  \hline
 & Oracle & Silverman & CV \\ 
  \hline
MISE & 0.000111 & 0.000119 & 0.000128 \\ 
  Relative MISE & 0.000170 & 0.000182 & 0.000196 \\ 
  MIAE & 0.001345 & 0.001111 & 0.001360 \\ 
  Relative MIAE & 0.001665 & 0.001375 & 0.001684 \\ 
  Max Error & 0.022746 & 0.025000 & 0.026488 \\ 
  Peak bias & 0.025524 & 0.032193 & 0.031474 \\ 
  Relative Peak bias & 0.031597 & 0.039853 & 0.038962 \\ 
  Peak drift & 0.070303 & 0.116284 & 0.103061 \\ 
  Relative Peak drift & 0.010043 & 0.016612 & 0.014723 \\ 
  Centroid bias & 0.025868 & 0.035816 & 0.031993 \\ 
  Relative Centroid bias & 0.032023 & 0.044337 & 0.039604 \\ 
  Centroid drift & 0.054288 & 0.054756 & 0.054273 \\ 
  Relative Centroid drift & 0.007755 & 0.007822 & 0.007753 \\ 
   \hline
\end{tabular}
\caption{Standard deviation of error rates} 
\label{tbl:stddev_error_rates}
\end{table}

    \caption{Standard deviations} 
    \end{subtable}

\caption{Error rates for uniform population of 10,000, single peak intensity of factor 1000}
\label{tbl:mean_error_rates:unif_1000_1_1h}
\end{table}


%%
%% Section
\section{Varying population and cases together}

\subsection{100 cases from 10,000}

This is the same as \cref{tbl:mean_error_rates:unif_100_1_1h}.

\begin{table}[H]
\centering
\scriptsize

    \begin{subtable}{0.5\textwidth}
    % latex table generated in R 3.4.0 by xtable 1.8-2 package
% Sat Aug  5 21:15:22 2017
\begin{tabular}{lrrr}
  \hline
 & Oracle & Silverman & CV \\ 
  \hline
MISE & 0.000022 & 0.000035 & 0.000034 \\ 
  Relative MISE & 0.003358 & 0.005293 & 0.005266 \\ 
  MIAE & 0.002505 & 0.003100 & 0.003092 \\ 
  Relative MIAE & 0.031010 & 0.038378 & 0.038279 \\ 
  Max Error & 0.020705 & 0.030617 & 0.030498 \\ 
  Peak bias & -0.012292 & 0.006167 & 0.006026 \\ 
  Relative Peak bias & -0.152166 & 0.076337 & 0.074592 \\ 
  Peak drift & 0.265067 & 0.409888 & 0.408051 \\ 
  Relative Peak drift & 0.037867 & 0.058555 & 0.058293 \\ 
  Centroid bias & -0.012639 & -0.000422 & -0.000492 \\ 
  Relative Centroid bias & -0.156464 & -0.005226 & -0.006090 \\ 
  Centroid drift & 0.199485 & 0.216521 & 0.216495 \\ 
  Relative Centroid drift & 0.028498 & 0.030932 & 0.030928 \\ 
   \hline
\end{tabular}

    \caption{Means} 
    \end{subtable}%
    \begin{subtable}{0.5\textwidth}
    % latex table generated in R 3.4.0 by xtable 1.8-2 package
% Sat Aug  5 21:15:23 2017
\begin{tabular}{lrrr}
  \hline
 & Oracle & Silverman & CV \\ 
  \hline
MISE & 0.000012 & 0.000013 & 0.000013 \\ 
  Relative MISE & 0.001764 & 0.001964 & 0.001962 \\ 
  MIAE & 0.000525 & 0.000440 & 0.000441 \\ 
  Relative MIAE & 0.006496 & 0.005449 & 0.005460 \\ 
  Max Error & 0.006748 & 0.008658 & 0.008654 \\ 
  Peak bias & 0.009900 & 0.016313 & 0.016285 \\ 
  Relative Peak bias & 0.122549 & 0.201943 & 0.201595 \\ 
  Peak drift & 0.140602 & 0.212548 & 0.211161 \\ 
  Relative Peak drift & 0.020086 & 0.030364 & 0.030166 \\ 
  Centroid bias & 0.010031 & 0.016949 & 0.016909 \\ 
  Relative Centroid bias & 0.124173 & 0.209809 & 0.209321 \\ 
  Centroid drift & 0.106052 & 0.113541 & 0.114205 \\ 
  Relative Centroid drift & 0.015150 & 0.016220 & 0.016315 \\ 
   \hline
\end{tabular}

    \caption{Standard deviations} 
    \end{subtable}

\caption{Error rates for uniform population of 10,000, single peak intensity of factor 100}
\label{tbl:mean_error_rates:unif_100_1_1h:2}
\end{table}

\subsection{200 cases from 20,000}
\begin{table}[H]
\centering
\scriptsize

    \begin{subtable}{0.5\textwidth}
    % latex table generated in R 3.4.0 by xtable 1.8-2 package
% Sun Aug 13 13:20:48 2017
\begin{table}[ht]
\centering
\begin{tabular}{rrrr}
  \hline
 & Oracle & Silverman & CV \\ 
  \hline
MISE & 0.000015 & 0.000023 & 0.000022 \\ 
  Relative MISE & 0.002372 & 0.003481 & 0.003425 \\ 
  MIAE & 0.002069 & 0.002496 & 0.002476 \\ 
  Relative MIAE & 0.025606 & 0.030901 & 0.030651 \\ 
  Max Error & 0.018984 & 0.026091 & 0.025789 \\ 
  Peak bias & -0.011406 & 0.001175 & 0.000799 \\ 
  Relative Peak bias & -0.141193 & 0.014544 & 0.009891 \\ 
  Peak drift & 0.291837 & 0.422926 & 0.420581 \\ 
  Relative Peak drift & 0.041691 & 0.060418 & 0.060083 \\ 
  Centroid bias & -0.012288 & -0.006236 & -0.006300 \\ 
  Relative Centroid bias & -0.152112 & -0.077190 & -0.077988 \\ 
  Centroid drift & 0.171960 & 0.190351 & 0.190355 \\ 
  Relative Centroid drift & 0.024566 & 0.027193 & 0.027194 \\ 
   \hline
\end{tabular}
\caption{Mean error rates} 
\label{tbl:mean_error_rates}
\end{table}

    \caption{Means} 
    \end{subtable}%
    \begin{subtable}{0.5\textwidth}
    % latex table generated in R 3.4.0 by xtable 1.8-2 package
% Sun Aug 13 13:20:48 2017
\begin{tabular}{rrrr}
  \hline
 & Oracle & Silverman & CV \\ 
  \hline
MISE & 0.000007 & 0.000007 & 0.000007 \\ 
  Relative MISE & 0.001051 & 0.001092 & 0.001093 \\ 
  MIAE & 0.000370 & 0.000314 & 0.000318 \\ 
  Relative MIAE & 0.004574 & 0.003892 & 0.003931 \\ 
  Max Error & 0.005609 & 0.005995 & 0.006009 \\ 
  Peak bias & 0.007725 & 0.011211 & 0.011137 \\ 
  Relative Peak bias & 0.095629 & 0.138785 & 0.137868 \\ 
  Peak drift & 0.155419 & 0.211609 & 0.209799 \\ 
  Relative Peak drift & 0.022203 & 0.030230 & 0.029971 \\ 
  Centroid bias & 0.007989 & 0.012125 & 0.012026 \\ 
  Relative Centroid bias & 0.098899 & 0.150096 & 0.148867 \\ 
  Centroid drift & 0.097875 & 0.103910 & 0.104528 \\ 
  Relative Centroid drift & 0.013982 & 0.014844 & 0.014933 \\ 
   \hline
\end{tabular}

    \caption{Standard deviations} 
    \end{subtable}

\caption{Error rates for uniform population of 20,000, single peak intensity of factor 200}
\label{tbl:mean_error_rates:unif20k_200_1_1h}
\end{table}

\subsection{400 cases from 40,000}
\begin{table}[H]
\centering
\scriptsize

    \begin{subtable}{0.5\textwidth}
    % latex table generated in R 3.4.0 by xtable 1.8-2 package
% Sun Aug 13 14:16:06 2017
\begin{table}[ht]
\centering
\begin{tabular}{rrrr}
  \hline
 & Oracle & Silverman & CV \\ 
  \hline
MISE & 0.000010 & 0.000014 & 0.000013 \\ 
  Relative MISE & 0.001506 & 0.002142 & 0.002051 \\ 
  MIAE & 0.001642 & 0.001969 & 0.001926 \\ 
  Relative MIAE & 0.020323 & 0.024376 & 0.023837 \\ 
  Max Error & 0.015691 & 0.021412 & 0.020759 \\ 
  Peak bias & -0.009361 & 0.001818 & 0.000969 \\ 
  Relative Peak bias & -0.115878 & 0.022501 & 0.011994 \\ 
  Peak drift & 0.224460 & 0.348102 & 0.343680 \\ 
  Relative Peak drift & 0.032066 & 0.049729 & 0.049097 \\ 
  Centroid bias & -0.010033 & -0.004142 & -0.004456 \\ 
  Relative Centroid bias & -0.124197 & -0.051270 & -0.055156 \\ 
  Centroid drift & 0.125798 & 0.138387 & 0.137610 \\ 
  Relative Centroid drift & 0.017971 & 0.019770 & 0.019659 \\ 
   \hline
\end{tabular}
\caption{Mean error rates} 
\label{tbl:mean_error_rates}
\end{table}

    \caption{Means} 
    \end{subtable}%
    \begin{subtable}{0.5\textwidth}
    % latex table generated in R 3.4.0 by xtable 1.8-2 package
% Sun Aug 13 14:16:06 2017
\begin{table}[ht]
\centering
\begin{tabular}{rrrr}
  \hline
 & Oracle & Silverman & CV \\ 
  \hline
MISE & 0.000004 & 0.000004 & 0.000004 \\ 
  Relative MISE & 0.000659 & 0.000601 & 0.000609 \\ 
  MIAE & 0.000278 & 0.000220 & 0.000229 \\ 
  Relative MIAE & 0.003442 & 0.002723 & 0.002830 \\ 
  Max Error & 0.004435 & 0.004672 & 0.004730 \\ 
  Peak bias & 0.006121 & 0.008856 & 0.008732 \\ 
  Relative Peak bias & 0.075776 & 0.109626 & 0.108095 \\ 
  Peak drift & 0.116624 & 0.171068 & 0.166589 \\ 
  Relative Peak drift & 0.016661 & 0.024438 & 0.023798 \\ 
  Centroid bias & 0.006271 & 0.009763 & 0.009491 \\ 
  Relative Centroid bias & 0.077629 & 0.120854 & 0.117487 \\ 
  Centroid drift & 0.073622 & 0.078391 & 0.079245 \\ 
  Relative Centroid drift & 0.010517 & 0.011199 & 0.011321 \\ 
   \hline
\end{tabular}
\caption{Standard deviation of error rates} 
\label{tbl:stddev_error_rates}
\end{table}

    \caption{Standard deviations} 
    \end{subtable}

\caption{Error rates for uniform population of 40,000, single peak intensity of factor 400}
\label{tbl:mean_error_rates:unif40k_400_1_1h}
\end{table}

\subsection{600 cases from 60,000}
\begin{table}[H]
\centering
\scriptsize

    \begin{subtable}{0.5\textwidth}
    % latex table generated in R 3.4.0 by xtable 1.8-2 package
% Sun Aug 13 15:09:40 2017
\begin{table}[ht]
\centering
\begin{tabular}{rrrr}
  \hline
 & Oracle & Silverman & CV \\ 
  \hline
MISE & 0.000007 & 0.000011 & 0.000010 \\ 
  Relative MISE & 0.001127 & 0.001630 & 0.001509 \\ 
  MIAE & 0.001426 & 0.001718 & 0.001650 \\ 
  Relative MIAE & 0.017653 & 0.021262 & 0.020430 \\ 
  Max Error & 0.013768 & 0.019194 & 0.018128 \\ 
  Peak bias & -0.007019 & 0.002555 & 0.001164 \\ 
  Relative Peak bias & -0.086889 & 0.031633 & 0.014413 \\ 
  Peak drift & 0.183350 & 0.302506 & 0.287383 \\ 
  Relative Peak drift & 0.026193 & 0.043215 & 0.041055 \\ 
  Centroid bias & -0.007706 & -0.003067 & -0.003522 \\ 
  Relative Centroid bias & -0.095392 & -0.037962 & -0.043602 \\ 
  Centroid drift & 0.097719 & 0.112315 & 0.111326 \\ 
  Relative Centroid drift & 0.013960 & 0.016045 & 0.015904 \\ 
   \hline
\end{tabular}
\caption{Mean error rates} 
\label{tbl:mean_error_rates}
\end{table}

    \caption{Means} 
    \end{subtable}%
    \begin{subtable}{0.5\textwidth}
    % latex table generated in R 3.4.0 by xtable 1.8-2 package
% Sun Aug 13 15:09:40 2017
\begin{tabular}{rrrr}
  \hline
 & Oracle & Silverman & CV \\ 
  \hline
MISE & 0.000003 & 0.000003 & 0.000003 \\ 
  Relative MISE & 0.000484 & 0.000431 & 0.000453 \\ 
  MIAE & 0.000231 & 0.000177 & 0.000195 \\ 
  Relative MIAE & 0.002861 & 0.002194 & 0.002409 \\ 
  Max Error & 0.003864 & 0.004055 & 0.004263 \\ 
  Peak bias & 0.005921 & 0.008327 & 0.008273 \\ 
  Relative Peak bias & 0.073291 & 0.103080 & 0.102413 \\ 
  Peak drift & 0.106179 & 0.150642 & 0.148199 \\ 
  Relative Peak drift & 0.015168 & 0.021520 & 0.021171 \\ 
  Centroid bias & 0.006052 & 0.008976 & 0.008647 \\ 
  Relative Centroid bias & 0.074912 & 0.111120 & 0.107043 \\ 
  Centroid drift & 0.064924 & 0.069500 & 0.068346 \\ 
  Relative Centroid drift & 0.009275 & 0.009929 & 0.009764 \\ 
   \hline
\end{tabular}

    \caption{Standard deviations} 
    \end{subtable}

\caption{Error rates for uniform population of 60,000, single peak intensity of factor 600}
\label{tbl:mean_error_rates:unif60k_600_1_1h}
\end{table}

\subsection{800 cases from 80,000}
\begin{table}[H]
\centering
\scriptsize

    \begin{subtable}{0.5\textwidth}
    % latex table generated in R 3.4.0 by xtable 1.8-2 package
% Sun Aug 13 16:07:13 2017
\begin{tabular}{rrrr}
  \hline
 & Oracle & Silverman & CV \\ 
  \hline
MISE & 0.000006 & 0.000009 & 0.000008 \\ 
  Relative MISE & 0.000891 & 0.001335 & 0.001173 \\ 
  MIAE & 0.001278 & 0.001556 & 0.001458 \\ 
  Relative MIAE & 0.015821 & 0.019260 & 0.018043 \\ 
  Max Error & 0.012651 & 0.017620 & 0.016067 \\ 
  Peak bias & -0.006430 & 0.002183 & 0.000162 \\ 
  Relative Peak bias & -0.079594 & 0.027022 & 0.002006 \\ 
  Peak drift & 0.189578 & 0.304958 & 0.282499 \\ 
  Relative Peak drift & 0.027083 & 0.043565 & 0.040357 \\ 
  Centroid bias & -0.007286 & -0.003685 & -0.004249 \\ 
  Relative Centroid bias & -0.090199 & -0.045614 & -0.052603 \\ 
  Centroid drift & 0.083142 & 0.094548 & 0.091757 \\ 
  Relative Centroid drift & 0.011877 & 0.013507 & 0.013108 \\ 
   \hline
\end{tabular}

    \caption{Means} 
    \end{subtable}%
    \begin{subtable}{0.5\textwidth}
    % latex table generated in R 3.4.0 by xtable 1.8-2 package
% Sun Aug 13 16:07:13 2017
\begin{tabular}{rrrr}
  \hline
 & Oracle & Silverman & CV \\ 
  \hline
MISE & 0.000002 & 0.000002 & 0.000002 \\ 
  Relative MISE & 0.000342 & 0.000328 & 0.000353 \\ 
  MIAE & 0.000186 & 0.000146 & 0.000173 \\ 
  Relative MIAE & 0.002300 & 0.001813 & 0.002141 \\ 
  Max Error & 0.003414 & 0.003799 & 0.003941 \\ 
  Peak bias & 0.005078 & 0.007028 & 0.006866 \\ 
  Relative Peak bias & 0.062864 & 0.086995 & 0.084993 \\ 
  Peak drift & 0.106356 & 0.140820 & 0.133500 \\ 
  Relative Peak drift & 0.015194 & 0.020117 & 0.019071 \\ 
  Centroid bias & 0.005216 & 0.007790 & 0.007253 \\ 
  Relative Centroid bias & 0.064570 & 0.096434 & 0.089789 \\ 
  Centroid drift & 0.061492 & 0.062965 & 0.062084 \\ 
  Relative Centroid drift & 0.008785 & 0.008995 & 0.008869 \\ 
   \hline
\end{tabular}

    \caption{Standard deviations} 
    \end{subtable}

\caption{Error rates for uniform population of 80,000, single peak intensity of factor 800}
\label{tbl:mean_error_rates:unif80k_800_1_1h}
\end{table}

\subsection{1000 cases from 100,000}
\begin{table}[H]
\centering
\scriptsize

    \begin{subtable}{0.5\textwidth}
    % latex table generated in R 3.4.0 by xtable 1.8-2 package
% Sun Jul 30 00:05:45 2017
\begin{table}[ht]
\centering
\begin{tabular}{rrrr}
  \hline
 & Oracle & Silverman & CV \\ 
  \hline
MISE & 0.000002 & 0.000002 & 0.000002 \\ 
  Relative MISE & 0.000296 & 0.000275 & 0.000312 \\ 
  MIAE & 0.000172 & 0.000134 & 0.000169 \\ 
  Relative MIAE & 0.002123 & 0.001663 & 0.002093 \\ 
  Max Error & 0.003190 & 0.003425 & 0.003799 \\ 
  Peak bias & 0.004776 & 0.006378 & 0.006163 \\ 
  Relative Peak bias & 0.059119 & 0.078949 & 0.076294 \\ 
  Peak drift & 0.106483 & 0.145200 & 0.134117 \\ 
  Relative Peak drift & 0.015212 & 0.020743 & 0.019160 \\ 
  Centroid bias & 0.004937 & 0.007348 & 0.006476 \\ 
  Relative Centroid bias & 0.061119 & 0.090964 & 0.080164 \\ 
  Centroid drift & 0.059317 & 0.062619 & 0.061337 \\ 
  Relative Centroid drift & 0.008474 & 0.008946 & 0.008762 \\ 
   \hline
\end{tabular}
\caption{Standard deviation of error rates} 
\label{tbl:stddev_error_rates}
\end{table}

    \caption{Means} 
    \end{subtable}%
    \begin{subtable}{0.5\textwidth}
    % latex table generated in R 3.4.2 by xtable 1.8-2 package
% Thu Feb 15 19:57:40 2018
\begin{tabular}{lrrr}
  \hline
 & Oracle & Silverman & CV \\ 
  \hline
MISE & 0.000089 & 0.000116 & 0.000228 \\ 
  Relative MISE & 0.000131 & 0.000171 & 0.000335 \\ 
  Normalized MISE & 0.000018 & 0.000023 & 0.000046 \\ 
  MIAE & 0.000911 & 0.000820 & 0.001681 \\ 
  Relative MIAE & 0.001105 & 0.000994 & 0.002040 \\ 
  Max Error & 0.024919 & 0.036302 & 0.058950 \\ 
  Peak bias & 0.031372 & 0.045576 & 0.062499 \\ 
  Relative Peak bias & 0.038067 & 0.055302 & 0.075836 \\ 
  Peak drift & 0.060193 & 0.086240 & 0.111909 \\ 
  Relative Peak drift & 0.008599 & 0.012320 & 0.015987 \\ 
  Centroid bias & 0.031457 & 0.047196 & 0.046071 \\ 
  Relative Centroid bias & 0.038170 & 0.057267 & 0.055902 \\ 
  Centroid drift & 0.051758 & 0.054538 & 0.052729 \\ 
  Relative Centroid drift & 0.007394 & 0.007791 & 0.007533 \\ 
   \hline
\end{tabular}

    \caption{Standard deviations} 
    \end{subtable}

\caption{Error rates for uniform population of 100,000, single peak intensity of factor 1000}
\label{tbl:mean_error_rates:unif100k_1000_1_1h}
\end{table}


%%
%% Section
\section{Varying the decay of the risk function}


\subsection{100 cases from 10,000 with no decay (uniform)}

This is the same as \Cref{tbl:mean_error_rates:unif_100_unif}.

\begin{table}[H]
\centering
\scriptsize

    \begin{subtable}{0.5\textwidth}
    % latex table generated in R 3.3.3 by xtable 1.8-2 package
% Sun Mar 11 18:27:52 2018
\begin{tabular}{lrrr}
  \hline
 & Oracle & Silverman & CV \\ 
  \hline
MISE & 0.000012 & 0.000013 & 0.000013 \\ 
  Relative MISE & 0.116112 & 0.125979 & 0.128308 \\ 
  Normalized MISE & 0.000000 & 0.000000 & 0.000000 \\ 
  MIAE & 0.002786 & 0.002898 & 0.002938 \\ 
  Relative MIAE & 0.278620 & 0.289846 & 0.293819 \\ 
  Normalized MIAE & 0.000000 & 0.000000 & 0.000000 \\ 
  Max Error & 0.008551 & 0.009000 & 0.008926 \\ 
  Normalized Max Error & 0.000001 & 0.000001 & 0.000001 \\ 
   \hline
\end{tabular}

    \caption{Means} 
    \end{subtable}%
    \begin{subtable}{0.5\textwidth}
    % latex table generated in R 3.4.3 by xtable 1.8-2 package
% Tue Apr 03 12:03:44 2018
\begin{tabular}{lrrr}
  \hline
 & Oracle & Silverman & CV \\ 
  \hline
MISE & 0.000003 & 0.000003 & 0.000003 \\ 
  Relative MISE & 0.030194 & 0.032029 & 0.034292 \\ 
  Normalized MISE & 0.301940 & 0.320288 & 0.342922 \\ 
  MIAE & 0.000438 & 0.000418 & 0.000446 \\ 
  Relative MIAE & 0.043781 & 0.041769 & 0.044615 \\ 
  Normalized MIAE & 0.000004 & 0.000004 & 0.000004 \\ 
  Max Error & 0.000581 & 0.000870 & 0.001109 \\ 
  Normalized Max Error & 0.000006 & 0.000009 & 0.000011 \\ 
  Peak bias & 0.001757 & 0.002365 & 0.002891 \\ 
  Relative Peak bias & 0.175734 & 0.236474 & 0.289140 \\ 
  Peak drift & 1.467764 & 1.600954 & 1.558562 \\ 
  Relative Peak drift & 0.209681 & 0.228708 & 0.222652 \\ 
  Centroid bias & 0.002315 & 0.003475 & 0.003480 \\ 
  Relative Centroid bias & 0.231491 & 0.347454 & 0.347951 \\ 
  Centroid drift & 1.204587 & 1.197905 & 1.181971 \\ 
  Relative Centroid drift & 0.172084 & 0.171129 & 0.168853 \\ 
   \hline
\end{tabular}

    \caption{Standard deviations} 
    \end{subtable}

\caption{Error rates for uniform population of 10,000, single peak intensity of factor 100 and no decay (uniform)}
\label{tbl:mean_error_rates:unif_100_unif:2}
\end{table}

\subsection{100 cases from 10,000 with decay rate 2.0}
\begin{table}[H]
\centering
\scriptsize

    \begin{subtable}{0.5\textwidth}
    % latex table generated in R 3.4.0 by xtable 1.8-2 package
% Sat Aug  5 21:31:40 2017
\begin{table}[ht]
\centering
\begin{tabular}{lrrr}
  \hline
 & Oracle & Silverman & CV \\ 
  \hline
MISE & 0.000006 & 0.000012 & 0.000012 \\ 
  Relative MISE & 0.011308 & 0.021333 & 0.021316 \\ 
  MIAE & 0.001867 & 0.002638 & 0.002637 \\ 
  Relative MIAE & 0.079514 & 0.112328 & 0.112282 \\ 
  Max Error & 0.006153 & 0.010407 & 0.010401 \\ 
  Peak bias & -0.003225 & 0.003258 & 0.003251 \\ 
  Relative Peak bias & -0.137322 & 0.138713 & 0.138426 \\ 
  Peak drift & 0.615800 & 0.900437 & 0.900381 \\ 
  Relative Peak drift & 0.087971 & 0.128634 & 0.128626 \\ 
  Centroid bias & -0.003271 & 0.001927 & 0.001923 \\ 
  Relative Centroid bias & -0.139292 & 0.082052 & 0.081892 \\ 
  Centroid drift & 0.534199 & 0.668770 & 0.668461 \\ 
  Relative Centroid drift & 0.076314 & 0.095539 & 0.095494 \\ 
   \hline
\end{tabular}
\caption{Mean error rates} 
\label{tbl:mean_error_rates}
\end{table}

    \caption{Means} 
    \end{subtable}%
    \begin{subtable}{0.5\textwidth}
    % latex table generated in R 3.4.0 by xtable 1.8-2 package
% Sat Aug  5 21:31:40 2017
\begin{tabular}{lrrr}
  \hline
 & Oracle & Silverman & CV \\ 
  \hline
MISE & 0.000004 & 0.000004 & 0.000004 \\ 
  Relative MISE & 0.007039 & 0.007532 & 0.007529 \\ 
  MIAE & 0.000542 & 0.000431 & 0.000431 \\ 
  Relative MIAE & 0.023067 & 0.018346 & 0.018348 \\ 
  Max Error & 0.001995 & 0.002548 & 0.002547 \\ 
  Peak bias & 0.002817 & 0.004825 & 0.004823 \\ 
  Relative Peak bias & 0.119960 & 0.205429 & 0.205337 \\ 
  Peak drift & 0.333560 & 0.493778 & 0.493829 \\ 
  Relative Peak drift & 0.047651 & 0.070540 & 0.070547 \\ 
  Centroid bias & 0.002842 & 0.005453 & 0.005452 \\ 
  Relative Centroid bias & 0.121027 & 0.232194 & 0.232131 \\ 
  Centroid drift & 0.287704 & 0.353528 & 0.353490 \\ 
  Relative Centroid drift & 0.041101 & 0.050504 & 0.050499 \\ 
   \hline
\end{tabular}

    \caption{Standard deviations} 
    \end{subtable}

\caption{Error rates for uniform population of 10,000, single peak intensity of factor 100 and decay rate 2.0}
\label{tbl:mean_error_rates:unif_100_2_1h}
\end{table}

\subsection{100 cases from 10,000 with decay rate 1.4}
\begin{table}[H]
\centering
\scriptsize

    \begin{subtable}{0.5\textwidth}
    % latex table generated in R 3.4.2 by xtable 1.8-2 package
% Thu Dec  7 17:09:22 2017
\begin{tabular}{lrrr}
  \hline
 & Oracle & Silverman & CV \\ 
  \hline
MISE & 0.000012 & 0.000019 & 0.000019 \\ 
  Relative MISE & 0.006627 & 0.010652 & 0.010635 \\ 
  MIAE & 0.002291 & 0.002949 & 0.002946 \\ 
  Relative MIAE & 0.054683 & 0.070397 & 0.070338 \\ 
  Max Error & 0.011293 & 0.016460 & 0.016441 \\ 
  Peak bias & -0.005739 & 0.003331 & 0.003308 \\ 
  Relative Peak bias & -0.137007 & 0.079526 & 0.078980 \\ 
  Peak drift & 0.407527 & 0.576641 & 0.576732 \\ 
  Relative Peak drift & 0.058218 & 0.082377 & 0.082390 \\ 
  Centroid bias & -0.005876 & 0.001197 & 0.001182 \\ 
  Relative Centroid bias & -0.140265 & 0.028578 & 0.028217 \\ 
  Centroid drift & 0.337723 & 0.384568 & 0.383883 \\ 
  Relative Centroid drift & 0.048246 & 0.054938 & 0.054840 \\ 
   \hline
\end{tabular}

    \caption{Means} 
    \end{subtable}%
    \begin{subtable}{0.5\textwidth}
    % latex table generated in R 3.4.2 by xtable 1.8-2 package
% Thu Dec  7 17:09:22 2017
\begin{tabular}{lrrr}
  \hline
 & Oracle & Silverman & CV \\ 
  \hline
MISE & 0.000006 & 0.000007 & 0.000007 \\ 
  Relative MISE & 0.003137 & 0.003786 & 0.003783 \\ 
  MIAE & 0.000459 & 0.000437 & 0.000437 \\ 
  Relative MIAE & 0.010946 & 0.010433 & 0.010433 \\ 
  Max Error & 0.003239 & 0.004242 & 0.004240 \\ 
  Peak bias & 0.005135 & 0.008388 & 0.008383 \\ 
  Relative Peak bias & 0.122587 & 0.200238 & 0.200114 \\ 
  Peak drift & 0.224036 & 0.304266 & 0.304044 \\ 
  Relative Peak drift & 0.032005 & 0.043467 & 0.043435 \\ 
  Centroid bias & 0.005205 & 0.009014 & 0.009006 \\ 
  Relative Centroid bias & 0.124269 & 0.215189 & 0.215006 \\ 
  Centroid drift & 0.178480 & 0.198033 & 0.197719 \\ 
  Relative Centroid drift & 0.025497 & 0.028290 & 0.028246 \\ 
   \hline
\end{tabular}

    \caption{Standard deviations} 
    \end{subtable}

\caption{Error rates for uniform population of 10,000, single peak intensity of factor 100 and decay rate 1.4}
\label{tbl:mean_error_rates:unif_100_1.4_1h}
\end{table}

\subsection{100 cases from 10,000 with decay rate 1.4}
\begin{table}[H]
\centering
\scriptsize

    \begin{subtable}{0.5\textwidth}
    % latex table generated in R 3.4.2 by xtable 1.8-2 package
% Thu Dec  7 17:09:22 2017
\begin{tabular}{lrrr}
  \hline
 & Oracle & Silverman & CV \\ 
  \hline
MISE & 0.000012 & 0.000019 & 0.000019 \\ 
  Relative MISE & 0.006627 & 0.010652 & 0.010635 \\ 
  MIAE & 0.002291 & 0.002949 & 0.002946 \\ 
  Relative MIAE & 0.054683 & 0.070397 & 0.070338 \\ 
  Max Error & 0.011293 & 0.016460 & 0.016441 \\ 
  Peak bias & -0.005739 & 0.003331 & 0.003308 \\ 
  Relative Peak bias & -0.137007 & 0.079526 & 0.078980 \\ 
  Peak drift & 0.407527 & 0.576641 & 0.576732 \\ 
  Relative Peak drift & 0.058218 & 0.082377 & 0.082390 \\ 
  Centroid bias & -0.005876 & 0.001197 & 0.001182 \\ 
  Relative Centroid bias & -0.140265 & 0.028578 & 0.028217 \\ 
  Centroid drift & 0.337723 & 0.384568 & 0.383883 \\ 
  Relative Centroid drift & 0.048246 & 0.054938 & 0.054840 \\ 
   \hline
\end{tabular}

    \caption{Means} 
    \end{subtable}%
    \begin{subtable}{0.5\textwidth}
    % latex table generated in R 3.4.2 by xtable 1.8-2 package
% Thu Dec  7 17:09:22 2017
\begin{tabular}{lrrr}
  \hline
 & Oracle & Silverman & CV \\ 
  \hline
MISE & 0.000006 & 0.000007 & 0.000007 \\ 
  Relative MISE & 0.003137 & 0.003786 & 0.003783 \\ 
  MIAE & 0.000459 & 0.000437 & 0.000437 \\ 
  Relative MIAE & 0.010946 & 0.010433 & 0.010433 \\ 
  Max Error & 0.003239 & 0.004242 & 0.004240 \\ 
  Peak bias & 0.005135 & 0.008388 & 0.008383 \\ 
  Relative Peak bias & 0.122587 & 0.200238 & 0.200114 \\ 
  Peak drift & 0.224036 & 0.304266 & 0.304044 \\ 
  Relative Peak drift & 0.032005 & 0.043467 & 0.043435 \\ 
  Centroid bias & 0.005205 & 0.009014 & 0.009006 \\ 
  Relative Centroid bias & 0.124269 & 0.215189 & 0.215006 \\ 
  Centroid drift & 0.178480 & 0.198033 & 0.197719 \\ 
  Relative Centroid drift & 0.025497 & 0.028290 & 0.028246 \\ 
   \hline
\end{tabular}

    \caption{Standard deviations} 
    \end{subtable}

\caption{Error rates for uniform population of 10,000, single peak intensity of factor 100 and decay rate 1.4}
\label{tbl:mean_error_rates:unif_100_1.4_1h}
\end{table}

\subsection{100 cases from 10,000 with decay rate 1.0}

This is the same as \cref{tbl:mean_error_rates:unif_100_1_1h}
\begin{table}[H]
\centering
\scriptsize

    \begin{subtable}{0.5\textwidth}
    % latex table generated in R 3.4.0 by xtable 1.8-2 package
% Sat Aug  5 21:15:22 2017
\begin{tabular}{lrrr}
  \hline
 & Oracle & Silverman & CV \\ 
  \hline
MISE & 0.000022 & 0.000035 & 0.000034 \\ 
  Relative MISE & 0.003358 & 0.005293 & 0.005266 \\ 
  MIAE & 0.002505 & 0.003100 & 0.003092 \\ 
  Relative MIAE & 0.031010 & 0.038378 & 0.038279 \\ 
  Max Error & 0.020705 & 0.030617 & 0.030498 \\ 
  Peak bias & -0.012292 & 0.006167 & 0.006026 \\ 
  Relative Peak bias & -0.152166 & 0.076337 & 0.074592 \\ 
  Peak drift & 0.265067 & 0.409888 & 0.408051 \\ 
  Relative Peak drift & 0.037867 & 0.058555 & 0.058293 \\ 
  Centroid bias & -0.012639 & -0.000422 & -0.000492 \\ 
  Relative Centroid bias & -0.156464 & -0.005226 & -0.006090 \\ 
  Centroid drift & 0.199485 & 0.216521 & 0.216495 \\ 
  Relative Centroid drift & 0.028498 & 0.030932 & 0.030928 \\ 
   \hline
\end{tabular}

    \caption{Means} 
    \end{subtable}%
    \begin{subtable}{0.5\textwidth}
    % latex table generated in R 3.4.0 by xtable 1.8-2 package
% Sat Aug  5 21:15:23 2017
\begin{tabular}{lrrr}
  \hline
 & Oracle & Silverman & CV \\ 
  \hline
MISE & 0.000012 & 0.000013 & 0.000013 \\ 
  Relative MISE & 0.001764 & 0.001964 & 0.001962 \\ 
  MIAE & 0.000525 & 0.000440 & 0.000441 \\ 
  Relative MIAE & 0.006496 & 0.005449 & 0.005460 \\ 
  Max Error & 0.006748 & 0.008658 & 0.008654 \\ 
  Peak bias & 0.009900 & 0.016313 & 0.016285 \\ 
  Relative Peak bias & 0.122549 & 0.201943 & 0.201595 \\ 
  Peak drift & 0.140602 & 0.212548 & 0.211161 \\ 
  Relative Peak drift & 0.020086 & 0.030364 & 0.030166 \\ 
  Centroid bias & 0.010031 & 0.016949 & 0.016909 \\ 
  Relative Centroid bias & 0.124173 & 0.209809 & 0.209321 \\ 
  Centroid drift & 0.106052 & 0.113541 & 0.114205 \\ 
  Relative Centroid drift & 0.015150 & 0.016220 & 0.016315 \\ 
   \hline
\end{tabular}

    \caption{Standard deviations} 
    \end{subtable}

\caption{Error rates for uniform population of 10,000, single peak intensity of factor 100 and decay rate 1.0}
\label{tbl:mean_error_rates:unif_100_1_1h:3}
\end{table}

\subsection{100 cases from 10,000 with decay rate 0.7}
\begin{table}[H]
\centering
\scriptsize

    \begin{subtable}{0.5\textwidth}
    % latex table generated in R 3.4.2 by xtable 1.8-2 package
% Thu Dec  7 16:57:46 2017
\begin{tabular}{lrrr}
  \hline
 & Oracle & Silverman & CV \\ 
  \hline
MISE & 0.000042 & 0.000070 & 0.000067 \\ 
  Relative MISE & 0.001528 & 0.002568 & 0.002483 \\ 
  MIAE & 0.002425 & 0.003070 & 0.003019 \\ 
  Relative MIAE & 0.014713 & 0.018625 & 0.018318 \\ 
  Max Error & 0.041189 & 0.062268 & 0.060801 \\ 
  Peak bias & -0.020965 & 0.013460 & 0.011707 \\ 
  Relative Peak bias & -0.127194 & 0.081663 & 0.071024 \\ 
  Peak drift & 0.187770 & 0.306194 & 0.300844 \\ 
  Relative Peak drift & 0.026824 & 0.043742 & 0.042978 \\ 
  Centroid bias & -0.023222 & -0.008659 & -0.009076 \\ 
  Relative Centroid bias & -0.140888 & -0.052531 & -0.055064 \\ 
  Centroid drift & 0.114011 & 0.119838 & 0.119153 \\ 
  Relative Centroid drift & 0.016287 & 0.017120 & 0.017022 \\ 
   \hline
\end{tabular}

    \caption{Means} 
    \end{subtable}%
    \begin{subtable}{0.5\textwidth}
    % latex table generated in R 3.4.2 by xtable 1.8-2 package
% Sat Feb 17 16:38:14 2018
\begin{tabular}{lrrr}
  \hline
 & Oracle & Silverman & CV \\ 
  \hline
MISE & 0.000020 & 0.000027 & 0.000039 \\ 
  Relative MISE & 0.000739 & 0.000989 & 0.001432 \\ 
  Normalized MISE & 0.000020 & 0.000027 & 0.000039 \\ 
  MIAE & 0.000484 & 0.000443 & 0.000714 \\ 
  Relative MIAE & 0.002936 & 0.002687 & 0.004331 \\ 
  Max Error & 0.012449 & 0.016956 & 0.024634 \\ 
  Peak bias & 0.019842 & 0.030937 & 0.038487 \\ 
  Relative Peak bias & 0.120383 & 0.187693 & 0.233502 \\ 
  Peak drift & 0.105135 & 0.157797 & 0.148019 \\ 
  Relative Peak drift & 0.015019 & 0.022542 & 0.021146 \\ 
  Centroid bias & 0.020297 & 0.031820 & 0.031438 \\ 
  Relative Centroid bias & 0.123139 & 0.193052 & 0.190736 \\ 
  Centroid drift & 0.067499 & 0.069453 & 0.068564 \\ 
  Relative Centroid drift & 0.009643 & 0.009922 & 0.009795 \\ 
   \hline
\end{tabular}

    \caption{Standard deviations} 
    \end{subtable}

\caption{Error rates for uniform population of 10,000, single peak intensity of factor 100 and decay rate 0.7}
\label{tbl:mean_error_rates:unif_100_0.7_1h}
\end{table}

\subsection{100 cases from 10,000 with decay rate 0.5}
\begin{table}[H]
\centering
\scriptsize

    \begin{subtable}{0.5\textwidth}
    % latex table generated in R 3.4.0 by xtable 1.8-2 package
% Sat Aug  5 20:27:22 2017
\begin{table}[ht]
\centering
\begin{tabular}{rrrr}
  \hline
 & Oracle & Silverman & CV \\ 
  \hline
MISE & 0.000077 & 0.000131 & 0.000119 \\ 
  Relative MISE & 0.000734 & 0.001254 & 0.001144 \\ 
  MIAE & 0.002409 & 0.003038 & 0.002876 \\ 
  Relative MIAE & 0.007454 & 0.009399 & 0.008899 \\ 
  Max Error & 0.076835 & 0.117578 & 0.108564 \\ 
  Peak bias & -0.046725 & 0.020911 & 0.009059 \\ 
  Relative Peak bias & -0.144576 & 0.064704 & 0.028031 \\ 
  Peak drift & 0.129820 & 0.216138 & 0.203513 \\ 
  Relative Peak drift & 0.018546 & 0.030877 & 0.029073 \\ 
  Centroid bias & -0.054373 & -0.022933 & -0.029030 \\ 
  Relative Centroid bias & -0.168240 & -0.070958 & -0.089824 \\ 
  Centroid drift & 0.064940 & 0.071866 & 0.070546 \\ 
  Relative Centroid drift & 0.009277 & 0.010267 & 0.010078 \\ 
   \hline
\end{tabular}
\caption{Mean error rates} 
\label{tbl:mean_error_rates}
\end{table}

    \caption{Means} 
    \end{subtable}%
    \begin{subtable}{0.5\textwidth}
    % latex table generated in R 3.4.0 by xtable 1.8-2 package
% Sat Aug  5 20:27:22 2017
\begin{tabular}{lrrr}
  \hline
 & Oracle & Silverman & CV \\ 
  \hline
MISE & 0.000036 & 0.000046 & 0.000063 \\ 
  Relative MISE & 0.000340 & 0.000438 & 0.000607 \\ 
  MIAE & 0.000466 & 0.000425 & 0.000563 \\ 
  Relative MIAE & 0.001442 & 0.001314 & 0.001741 \\ 
  Max Error & 0.023200 & 0.030300 & 0.039341 \\ 
  Peak bias & 0.032038 & 0.050895 & 0.055081 \\ 
  Relative Peak bias & 0.099132 & 0.157480 & 0.170430 \\ 
  Peak drift & 0.080767 & 0.118526 & 0.114042 \\ 
  Relative Peak drift & 0.011538 & 0.016932 & 0.016292 \\ 
  Centroid bias & 0.033350 & 0.057284 & 0.053665 \\ 
  Relative Centroid bias & 0.103192 & 0.177250 & 0.166049 \\ 
  Centroid drift & 0.056974 & 0.058126 & 0.057936 \\ 
  Relative Centroid drift & 0.008139 & 0.008304 & 0.008277 \\ 
   \hline
\end{tabular}

    \caption{Standard deviations} 
    \end{subtable}

\caption{Error rates for uniform population of 10,000, single peak intensity of factor 100 and decay rate 0.5}
\label{tbl:mean_error_rates:unif_100_0.5_1h}
\end{table}


%%
%% Section
\section{Varying the decay of the population density}


%%
%% Section
\section{Varying the distance between two peaks}


%%
%% Section
\section{Varying the distance between the population and risk function peaks}


