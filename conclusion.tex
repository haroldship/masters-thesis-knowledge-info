% !TEX root = thesis.tex

%%
%%
%% Conclusion chapter
%%
%%

%%%%%%%%%%%%%%%%%%%%%%%%%%%%%%%%
%% How good is Silverman?
%%%%%%%%%%%%%%%%%%%%%%%%%%%%%%%%

We compared the performance of the \acrlong{dkd} in a variety of test scenarios
under two different bandwidth selection schemes:
\gls{silverman} Rule of Thumb and \acrlong{cv}.
We did this using several measures of accuracy,
each of which reflects a real-world concern related to epidemiological studies.
The global measures \gls{mise} and \gls{miae} showed no significant performance difference
for the \gls{dkd} using both schemes,
leading us to consider preference for the \gls{silverman} method due to its greater ease of computation.
This was true for the \gls{supremum error} as well.
Furthermore,
the chosen bandwidths and the \gls{mise} in its absolute,
relative and normalized forms performed in line with the theory of
\Cref{sec:method:accuracy,sec:theory:bandwidthselection}.
However,
for finding the mode,
or the peak of the risk function,
the \gls{cv} bandwidths were superior in finding the location of the peak
while maintaining similar performance for the magnitude.
This was true for both the simple \gls{peak bias} and \gls{peak drift} as well as for the
\gls{centroid bias} and \gls{centroid drift}.

When we examine the behavior under different population distributions,
we can conclude that the \gls{dkd} is sensitive to large variations in population density.

\textbf{bottom line:}

Under the right conditions,
\gls{dkd} can give a good approximation of a true \gls{incidence rate} function.
This is so in terms of the general accuracy of the rate at any given point,
as well as for the location of the peak.
Because the accuracy improves with the number of observations,
it is prudent to increase the time frame of the study when the number of observations in a given year is small.
In cases where the population density varies greatly over a study area,
the \gls{dkd} is less accurate.
Therefore in such cases,
it may be beneficial to divide the study area into sub-regions and compute separate \glspl{dkd}.

In general, the \gls{oracle} performed best, followed by \gls{cv}, then \gls{silverman}.
On average, the \gls{mise} of \gls{silverman} was between 0\% and 200\% of the \gls{mise} of \gls{cv}.
However, there were a few cases where \gls{silverman} slightly outperformed \gls{cv}.
When we looked at a uniform risk function on a uniformly distributed population (\Cref{sec:app:results_unif_unif}), \gls{silverman} was at least as good as \gls{cv} for every \gls{factor}.
The ratio of the \gls{mise} of \gls{silverman} to that of \gls{cv} for these experiments ranged from 0.94 to 1.0.
For some values of \gls{mu} (200, 500 but not 50, 100, or 1000) the \gls{miae} and \gls{supremum error} were better than \gls{cv} as well.

We saw this phenomenon in a few other cases as well.
When there were two peaks, but they were close together, \gls{cv} was outperformed by \gls{silverman}.
In particular, when the distance between the peaks was 1.0 and 2.0, but not for larger distances of 3.0 and 4.0.
Also when there are peaks in both the population and the risk function, sometimes the \gls{cv} does not perform as well as \gls{silverman}.

Some other cases where this occurs include \Cref{tab:mean_error_rates:unif_100_0.7_1h,tab:mean_error_rates:unif5k_50_1.0_1h,tab:mean_error_rates:p0.7_100_1.0_1h}.