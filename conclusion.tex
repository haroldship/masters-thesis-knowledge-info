% !TEX root = thesis.tex

%%
%%
%% Conclusion chapter
%%
%%

%%%%%%%%%%%%%%%%%%%%%%%%%%%%%%%%
%% How good is Silverman?
%%%%%%%%%%%%%%%%%%%%%%%%%%%%%%%%

In this study,
we examined the performance of the \acrfull{dkd} as an estimator of an incidence rate function in a variety of test scenarios.
Our simulation results show that for the \gls{incidence rate} functions we chose,
the estimation error of the \gls{dkd} is reasonable.
We find that this is so in two ways:
\begin{enumerate}
    \item In answer to \Cref{thm:accuracy-scale:global},
    the normalized global accuracy
    of the rate over the study area as indicated by \gls{nmise}
    is less than 5\% for uniform populations,
    and less than 5\% for most of the peaked populations.
    Furthermore, \gls{nmise} improves when the number of incidents increases.
    This is so regardless of whether this is due to an increase in the rate or in the population size.
    Also, the global accuracy increases when the spread of the rate increases.
    \item In answer to \Cref{thm:accuracy-scale:peaks},
    the magnitude of the point with the highest rate (the peak) can be estimated with the \gls{dkd}
    within 10\% for uniform populations,
    but overestimates the peak for peaked populations.
    \item In answer to \Cref{thm:accuracy-scale:peaks-location},
    the location of the point with the highest rate (the peak) can be estimated with the \gls{dkd}
    within 10\% of the size of the study area for uniform populations,
    and withing 15\% for peaked populations.
\end{enumerate}

For each experiment,
we used two different bandwidth selection schemes:
\gls{silverman}'s Rule of Thumb and \acrfull{cv}.
For most experiments,
we did not see a significant difference in any accuracy measure between these two schemes.
Additionally,
while both schemes were outperformed by the \gls{oracle bandwidth},
the difference was not too great;
and, since the \gls{oracle} will not be available in real-world studies based on data samples,
we consider both to be good approximations.
The range of selected bandwidths for all of these techniques was between $0.5$ and $2.0$ in a study area of dimensions $10.0 \times 10.0$.
This translates to 5\% -- 20\% of the size, assuming a square.
\begin{rec}
    \label{rec:bandwidth}
    For computing the \gls{dkd}, we recommend a bandwidth of between 5\% and 20\% of the width or height of the study area.
\end{rec}

We measured the accuracy of the \gls{dkd} using several accuracy measures,
each of which reflects a real-world concern related to epidemiological studies.
The global measures \gls{mise} and \gls{miae},
which represent the average accuracy of the estimator over the study area,
showed no significant performance difference
for the \gls{dkd} using both bandwidth selection schemes,
leading us to consider preference for the \gls{silverman} method due to its greater ease of computation.
While the \gls{mise} increased as the expected number of incidents \gls{mu} grew,
the \acrfull{nmise} decreased at a similar rate
in the examples we considered
for \gls{silverman} $(n^{-0.741})$ and \gls{cv} $(n^{-0.775})$ in answer to \Cref{thm:accuracy-affected:duration}.
In other words,
one can expect that using two years worth of data
to result in the global error \gls{mise} to be about 60\% of a single year.
Conversely,
to reduce the \gls{mise} to half of a given value
one needs 2.5 times as many incidents.
As far as small samples are concerned,
in the examples we studied,
we measured the \acrfull{rmise} at around $10\%$ for each bandwidth at 50 incidents.
Therefore,
the \gls{dkd} may provide reasonable estimates,
on average,
for samples of 50 incidents or more,
with increasing average accuracy as the number of incidents grows.
\begin{rec}
    \label{rec:small-sample}
    We recommend a sample size of at least 50 incidents for the \gls{dkd}.
    For a population of size 10,000,
    this is an overall rate of 0.5\%,
    and for a population of size 5,000,
    this is an overall rate of 1.0\%.
\end{rec}

In order to answer \Cref{thm:accuracy-affected:rates} we studied a range of spreads for different single-peak incident rate functions.
Our results using Gaussian shaped incidence functions show that the \gls{dkd} accuracy was affected by the spread,
in that the \gls{mise} decreased sharply
as the incidence rate spread increased and the height of the peak was lower.
In this case,
\gls{cv} performed better at $\gls{sigma_i}^{-1.576}$ than $\gls{sigma_i}^{-1.421}$ for \gls{silverman}.
In a similar manner,
our examples of increasing the population spread saw a sharp decrease in the \gls{mise},
in answer to \Cref{thm:accuracy-affected:popdist}.

In the examples we studied
in contrast to the average accuracy,
the \gls{cv} bandwidths were superior to \gls{silverman}
in finding the location of the peak,
while maintaining similar performance for the magnitude.
This was true for both the simple \gls{peak bias} and \gls{peak drift} as well as for the
\gls{centroid bias} and \gls{centroid drift}.
The centroid measures were much more accurate than the simple peak.
\begin{rec}
    \label{rec:bandwidth-scheme}
    For overall accuracy, we recommend the \gls{silverman} Rule of Thumb for computing the bandwidth due to its simplicity.
    However, if the location of the peak is important, we recommend least squares cross-validation.
\end{rec}

When we examine the behavior under different population distributions in consideration of \Cref{thm:accuracy-affected:popsize},
we observed that the \gls{dkd} is sensitive to large variations in population density.
This was true for all types of measures.
What we found was that for population densities with a very sharp peak,
so that the population was concentrated roughly 10\% of the study area or less,
the \gls{dkd} accuracy was 2-10 times worse than all other cases.
When the population density was not so concentrated,
the accuracy measures were stable.
However,
the estimation errors observed for the uniform populations were much lower than the peaked populations.
\begin{rec}
    \label{rec:pop-density}
    When the population density is highly variable,
    we do not recommend computing the \gls{dkd} over the entire study area.
\end{rec}
As we stated in \Cref{ch:further},
it is possible that for such cases,
an adaptive bandwidth selection technique would improve the estimate.

In summary,
under the conditions we studied,
the \gls{dkd} can give a good approximation of a true \gls{risk} or \gls{incidence rate} function.
This is so in terms of the general accuracy of the rate at any given point,
as well as for the location of the peak.
Our results did not show a large difference between the \gls{silverman} and \gls{cv} bandwidth selection schemes,
except for in determining the location of the peak.
Both schemes chose bandwidths between 5\% and 20\% of the size of the study area.
Because the accuracy improves with the number of observations,
it is prudent to increase the time frame of the study when the number of observations in a given year is smaller than 50.
In cases where the population density varies greatly over a study area,
the \gls{dkd} is less accurate than for smoother varying populations.

