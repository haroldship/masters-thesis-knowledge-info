% !TEX root = thesis.tex

%%
%%
%% Conclusion chapter
%%
%%

%%%%%%%%%%%%%%%%%%%%%%%%%%%%%%%%
%% How good is Silverman?
%%%%%%%%%%%%%%%%%%%%%%%%%%%%%%%%

We compared the performance of the \gls{dkd} under three different bandwidth schemes:
the \gls{oracle}, \gls{silverman}, and \gls{cv}.
In general, the \gls{oracle} performed best, followed by \gls{cv}, then \gls{silverman}.
On average, the \gls{mise} of \gls{silverman} was between 0\% and 200\% of the \gls{mise} of \gls{cv}.
However, there were a few cases where \gls{silverman} slightly outperformed \gls{cv}.
When we looked at a uniform risk function on a uniformly distributed population (\cref{sec:app:results_unif_unif}), \gls{silverman} was at least as good as \gls{cv} for every \gls{factor}.
The ratio of the \gls{mise} of \gls{silverman} to that of \gls{cv} for these experiments ranged from 0.94 to 1.0.
For some values of \gls{mu} (200, 500 but not 50, 100, or 1000) the \gls{miae} and \gls{supremum error} were better than \gls{cv} as well.

We saw this phenomenon in a few other cases as well.
When there were two peaks, but they were close together, \gls{cv} was outperformed by \gls{silverman}.
In particular, when the distance between the peaks was 1.0 and 2.0, but not for larger distances of 3.0 and 4.0.
Also when there are peaks in both the population and the risk function, sometimes the \gls{cv} does not perform as well as \gls{silverman}.

Some other cases where this occurs include \Cref{tab:mean_error_rates:unif_100_0.7_1h,tab:mean_error_rates:unif5k_50_1.0_1h,tab:mean_error_rates:p0.7_100_1.0_1h}.