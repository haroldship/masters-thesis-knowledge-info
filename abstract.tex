% !TEX root = thesis.tex

%%
%%
%% Abstract
%%
%%


\begin{onehalfspace}

\begin{center}
    \vspace*{0.5cm}

    \textbf{\Large Investigating the Statistical Properties of the Double Kernel Density Estimator}

    \vspace*{1.0cm}

    \textbf{Harold Ship}

    \vspace*{1.0cm}

\end{center}

\noindent\textbf{\LARGE Abstract}
\phantomsection\addcontentsline{toc}{chapter}{Abstract}

\paragraph*{Motivation:}

There are many examples in epidemiology of data that occur as points distributed in the plane.
One example is
the home addresses of persons in a neighborhood and,
in particular,
whether or not exposed individuals have developed a particular medical condition during a specific time period.
These data allow one to compute the \textit{incidence rate} of this condition.
Recently, a technique known as the \acrlong{dkd} has been used for estimating \acrlong{incidence rate} functions
$\gls{lambda} := \gls{lambda}\dotdot$.
The function \gls{lambda} is the ratio of two other functions \gls{lambda_I} and \gls{lambda_P}.
Each one of the above functions \gls{lambda_I} and \gls{lambda_P}
is itself estimated by smoothing a set of point observations into a continuous,
smooth function using \gls{kernel intensity estimation},
and the value of \gls{lambda} at any point is the estimate of \gls{lambda_I} at that point
divided by the estimate of \gls{lambda_P} at the same point.

Researchers have recently been motivated to use the \acrlong{dkd} due to information loss due to data aggregation of incidents into geographical areas such as cities or neighborhoods.
This aggregation has been deemed necessary both
to reduce the effects of variance on the relative errors of the results,
and to preserve the privacy of individuals who have been stricken with disease.
Use of the \acrlong{dkd} also allows for the linking of location data with other data sources,
such as welfare levels, air pollution and other attributes not available for individuals.

As far as we know,
the statistical properties of the \acrlong{dkd} are not yet fully understood.
We would like to know whether or not the estimates for \gls{risk} or \gls{incidence rate}
obtained using the \acrlong{dkd} are accurate or misleading under different conditions.
Of particular concern are the peaks of the true risk function.
These peaks indicate a point around which the risk of contracting a disease is at its highest.
While the accuracy of the estimate of the magnitude of these peaks is important,
it is critical that the location estimate be accurate so as not create any misleading associations
with other, nearby points of interest.

This study examines some of the statistical properties of the \acrlong{dkd}.
In particular,
we empirically analyze how different factors of the population and incidence distributions
affect the \textit{accuracy} of the \acrlong{dkd}.

\paragraph*{Theory and method:}

The type of study we are interested in involves incidents of a chronic disease.
The common measure of frequency used in the literature is called the incidence rate.
For a given population, time period, and set of incidents,
we can compute the overall incidence rate by taking the total number of incidents and dividing by the total population.
However, we may wish to compare the incidence rate at different locations within an area.
To do this, we compute the incidence rate point-wise.

The basic unit of our study is the \textit{experiment},
a set of monte carlo simulations run with the same initial setup with the same measurements taken.
Each experiment is run with a fixed set of parameters.
We run a set of experiments,
varying one parameter at a time, or in some cases two parameters in tandem,
in order to observe the effect of this parameter on the accuracy of the \acrlong{dkd}.

In order to compute the \acrlong{dkd},
a parameter known as the \textit{bandwidth} must be set,
and the accuracy of the \acrlong{dkd} is highly dependent upon it.
In each experiment,
we used two techniques,
Silverman Rule of Thumb and Least Squares Cross Validation,
to select the bandwidth.
We compare the results of these two techniques.

In order to describe the accuracy of the \acrlong{dkd} as a method of estimating the true risk function $\lambda$,
we use several accuracy measures.
In particular,
for each experiment we measure \acrlong{mise},
\acrlong{miae},
\glsentryname{supremum error},
\glsentryname{peak bias},
\glsentryname{peak drift},
\glsentryname{centroid bias},
and \glsentryname{centroid drift}.

The contribution of this study will be to answer the following research questions:
\begin{itemize}
    \item How is the accuracy of the \acrlong{dkd} affected by:
    \begin{itemize}
        \item the duration of the study,%
        \item different rate functions,%
        \item the size of the population,%
        \item and different population distributions.%
    \end{itemize}
    \item When looking at the the above factors,
    how accurate is the \acrlong{dkd}:
    \begin{itemize}
        \item globally, over the study area as a whole,%
        \item in magnitude, point-wise at the peaks,%
        \item location of the peaks.%
    \end{itemize}
\end{itemize}

\paragraph*{Results:}

We obtained results that show that increasing the expected number of incidents $N$,
such as what would occur when increasing the duration of the study,
reduces the selected bandwidth by a factor of around $N^{-1/6}$.
It also reduces the estimation error of the \acrlong{dkd} in the relative sense
for global errors by a factor of around $N^{-3/4}$ and also for point-wise magnitude and peak errors.
However, in this case the absolute errors increased with the expected number of incidents.
We also observed that increases in the spread $\sigma$ of the risk (rate) function also results in reduced estimation error
in both absolute and relative terms of approximately $\sigma^{-1.4}$.
We also observed that increasing the expected number of incidents in tandem with the population size reduced the estimation error,
at a rate of around $N^{-2.7}$
and that having a non-uniform population increased the estimation error slightly, 
except when the population spread was extremely narrow.

\paragraph*{Conclusion:}

In the examples we studied,
we found that statistically,
the \acrlong{dkd} can give a good approximation of a true risk or incidence rate function.
This is so in terms of the general accuracy of the rate at any given point,
where the \acrlong{nmise} was nearly always less than 5\% of the truth except for cases with extreme variation of the population density.
It was also a good estimator for the location of the peak,
where on average it was within 5-7\% of the size of the study area for uniform populations and 15\% for peaked populations.
For the most part,
the \acrlong{dkd} underestimated the magnitude of the peak,
in particular when using \acrlong{cv} to select the bandwidth. 
Our results did not show a large difference between the Silverman and \acrlong{cv} bandwidth selection schemes,
except for in determining the location of the peak.
Both schemes chose bandwidths between 5\% and 20\% of the size of the study area.
Because the accuracy improves with the number of observations,
it is prudent to increase the time frame of the study when the number of observations in a given year is smaller than 50.
In our experiments this was an incidence rate of between 0.1\% and 0.5\%.
In cases where the population density varies greatly over a study area,
the \acrlong{dkd} is less accurate than for smoother varying populations.

\paragraph*{Limitations:}

This study uses simulated data,
sampled from a known ``true'' incidence rate function in order to measure the accuracy of the \acrlong{dkd} estimator.
However,
our population and incidence rate functions,
which allow us to compute the population and incidents at points in the plane,
do not completely represent any actual population or incidents.
This means that our results describe the sensitivity of the \acrlong{dkd} to the factors we are interested in,
but do not provide any guarantees about the accuracy of the \acrlong{dkd} in any specific study where it has been used.

Another limitation of this study is that we compared only two bandwidth selection techniques.
There are several techniques available that we did not consider,
including adaptive bandwidth selectors which may give better results especially under highly variable population distributions.

Also, due to computational issues,
we assume that we have enough data for accurately estimating the population density and therefore used a fixed function for that in the denominator of the \acrlong{dkd}.
This means that we are working under the ideal scenario where there is no uncertainty in estimating the population density.


\end{onehalfspace}