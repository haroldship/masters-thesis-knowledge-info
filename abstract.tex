% !TEX root = thesis.tex

%%
%%
%% Abstract
%%
%%


\begin{onehalfspace}

\begin{center}
    \vspace*{0.5cm}

    \textbf{\Large Investigating the Statistical Properties of the Double Kernel Density Estimator}

    \vspace*{1.0cm}

    \textbf{Harold Ship}

    \vspace*{1.0cm}

\end{center}

\noindent\textbf{\LARGE Abstract}
\phantomsection\addcontentsline{toc}{chapter}{Abstract}

\paragraph*{Motivation:}

There are many examples in epidemiology of data that occur as points distributed in the plane.
One example is
the home addresses of persons in a neighborhood and,
in particular, whether or not they have a particular medical condition.
These data allow one to compute the \textit{incidence rate} of this condition.
Recently,
a technique known as the \acrlong{dkd} has been used for estimating \glsentryname{incidence rate} functions.

Researchers have recently been motivated to use the \acrlong{dkd} due to information loss due to data aggregation of incidents into geographical areas such as cities or neighborhoods.
This aggregation has been deemed necessary both
to reduce the effects of variance on the relative errors of the results,
and to preserve the privacy of individuals who have been stricken with disease.

This study examines some of the statistical properties of the \acrlong{dkd}.
In particular,
we empirically analyse how different factors of the population and incidence distributions
affect the \textit{accuracy} of the \acrlong{dkd}.

The contribution of this study will be to answer the following research questions:
\begin{enumerate}
    \item How is the accuracy of the \acrlong{dkd} affected by different factors?
    \item How accurate is the \acrlong{dkd} \textit{globally},
        over the study area as a whole,
        and \textit{pointwise}, at the peaks?
\end{enumerate}

\paragraph*{Theory and method:}

The type of study we are interested in involves incidents of a chronic disease.
The common measure of frequency used in the literature is called the incidence rate.
For a given population, time period, and set of incidents,
we can compute the overall incidence rate by taking the total number of incidents and dividing by the total population.
However, we may wish to compare the incidence rate at different locations within an area.
To do this, we compute the incidence rate point-wise.

The basic unit of our study is the \textit{experiment},
a set of monte carlo simulations run with the same initial setup with the same measurements taken.
Each experiment is run with a fixed set of parameters.
We run a set of experiments,
varying one parameter at a time, or in some cases two parameters in tandem,
in order to observe the effect of this parameter on the accuracy of the \acrlong{dkd}.

In order to compute the \acrlong{dkd},
a parameter known as the \textit{bandwidth} must be set,
and the accuracy of the \acrlong{dkd} is highly dependent upon it.
In each experiment,
we used two techniques,
Silverman Rule of Thumb and Least Squares Cross Validation,
to select the bandwidth.
We compare the results of these two techniques.

In order to describe the accuracy of the \acrlong{dkd} as a method of estimating the true risk function $\lambda$,
we use several accuracy measures.
In particular,
for each experiment we measure \acrlong{mise},
\acrlong{miae},
\glsentryname{supremum error},
\glsentryname{peak bias},
\glsentryname{peak drift},
\glsentryname{centroid bias},
and \glsentryname{centroid drift}.

\paragraph*{Results:}

We obtained results that show that increasing the expected number of incidents
reduces the estimation error of the \acrlong{dkd},
for both global and point-wise errors.
We also observed that increases in the spread of the risk (rate) function also results in reduced estimation error.
We also observed that increasing the expected number of incidents in tandem with the population size reduced the estimation error,
and that having a non-uniform population increased the estimation error.
In most cases,
our results were similar for both Silverman and CV selected bandwidths.

\paragraph*{Conclusion:}

In the examples we studied,
we found that statistically,
the \acrlong{dkd} can give a good approximation of a true risk or incidence rate function.
This is so in terms of the general accuracy of the rate at any given point,
as well as for the location of the peak.
Our results did not show a large difference between the Silverman and \acrlong{cv} bandwidth selection schemes,
except for in determining the location of the peak.
Both schemes chose bandwidths between 5\% and 20\% of the size of the study area.
Because the accuracy improves with the number of observations,
it is prudent to increase the time frame of the study when the number of observations in a given year is smaller than 50.
In cases where the population density varies greatly over a study area,
the \acrlong{dkd} is less accurate than for smoother varying populations.

\end{onehalfspace}